\Para{S�tze aus der Funktionalanalysis}{}
\Satz {Rademacher}{\RADEMACHER}
{\it Ist $U\subset \RR^n$ offen und $f\colon U\longrightarrow \RR$ 
Lipschitz-stetig, so ist $f$ ${\cal H}^n$-fast �berall differenzierbar
in $U$.}
\Beweis
Siehe [SL] Seite 30.
\Satz {Einbettungssatz von Sobolev}{\SOBOLEV}
{\it Seien $m\geq 1$, $1\leq p < \infty$ und $0\leq \alpha \leq 1$ so, 
da� $m-{n\over p}\geq k + \alpha$. Dann existiert eine stetige Einbettung
$$J\colon \Hnull(\N M)\longrightarrow C^{k,\alpha}(\N M)$$
F�r $m-{n\over p} > k + \alpha$ ist diese Einbettung sogar kompakt.}
\Beweis
Siehe [A] p. 44.
\medskip
Aus den Schauderschranken und den $L^p$-Absch�tzungen f�r Gebiete in 
$\RR^n$ folgen die entsprechenden Absch�tzungen f�r eine  kompakte 
Mannigfaltigkeit $M$. Man betrachtet dazu die Differentialoperatoren in 
lokalen  Koordinaten. Wegen der Kompaktheit reicht es, endlich viele 
lokale Koordinatensysteme zu betrachten. Die Konstanten in den Absch�tzungen 
sind dann das Maximum der Konstanten in den lokalen Koordinaten.
Man erh�lt also die folgenden S�tze.

\Satz {Schauderschranken}{\SCHAUDERSCHRANKEN}
{\it Sei $A$ linearer elliptischer Differentialoperator zweiter Ordnung,
 dann gilt 
$$|u|_{C^{2,\alpha}(M,\N M)} \leq C\bigl ( 
          |A(u)|_{C^{0,\alpha}(M,\N M)} + |u|_{C^{0,\alpha}(M,\N M)} \bigl )$$
}
\Satz{$L^p$-Theorie}{\LPAB}
{\it F�r einen linearen elliptischen Differentialoperator   zweiter 
Ordnung $A$ gilt
$$|u|_{H^{2,p}(M,\N M)} 
         \leq C \bigl ( |A(u)|_{L^p(M,\N M)} +|u|_{L^p(M,\N M)}\bigl )$$
}
\Satz{Maximumsprinzip}{\MAXPRINZ}
{\it
Sei $M$ eine kompakte Riemannsche Mannigfaltigkeit mit Rand.
$A(u)=a\nabla\nabla u + b\nabla u + cu$ sei ein linearer gleichm��ig 
elliptischer Differentialoperator mit beschr�nkten Koeffizienten und 
$c\leq 0$. Eine $\C^2$-L�sung der Gleichung $A(u)=0$ ist dann entweder 
konstant oder sie nimmt ihr Maximum auf $\partial M$ an.}
\Beweis
F�r jeden Punkt $p\in M^\circ$ gibt es lokale Koordinaten 
$\phi \colon U\longrightarrow V$ mit $p\in U^\circ$. $A_\phi$ ist dann ein 
linearer gleichm��ig elliptischer Differentialoperator �ber $V$ mit 
$c_\phi=c\leq0$.
Nach dem Hopfschen Maximumsprinzip (siehe z. B. [GT])
ist $u\circ\phi^{-1}$ konstant auf $V$ 
oder nimmt das Maximum in $\partial U$, also nicht in $p$, an.\kasten
\Satz{Spektralsatz f�r selbstadjungierte Operatoren}{\SPEKSATZ}
{\it Ist $X$ ein separabler Hilbertraum und $T\ne 0$ ein selbstadjungierter 
Operator auf $X$, so hat
$T$ die Gestalt
$$T(x)=\sum_{k\in N}{\lambda_k<x,e_k>_Xe_k}$$
mit $N\subset \NN$, einem Orthonormal-System $(e_k)_{k\in N}$ von $X$ und 
$\lambda_k\in\RR$ f�r alle $k\in N$.}
\Bem{}{}
Ist 0 kein Eigenwert von $T$, so ist $T$ invertierbar.

