Ist $x\in M$, $y\in \widetilde M$, $V^*\in T^*M$ und 
$W\in T\widetilde M$ so ist
$$\eqalign{F(x,y,V^*\otimes W)&\in \RR                              \cr
           F_x(x,y,F^*\otimes W)\colon =\nabla^MF(x,y,V^*\otimes W)
                              &\in T^*M                           \cr
           F_y(x,y,V^*\otimes W)\colon =\nabla^{\widetilde M}F(x,y,V^*\otimes W)
                              &\in T^*_y\widetilde M                \cr
           F_{V^*\otimes W}(x,y,V^*\otimes W)                       
                              &\in T^*_{V^*\otimes W}\Hom
                                   (T^*M\otimes T\widetilde M)      \cr
             }$$
\Bem{}{}
$F$, $F_y$ und $F_{V^*\otimes W}$ sind global auf 
$M\times\widetilde M\times(T^*M\otimes T\widetilde M)$ definierte 
(1,0)-Tensorfelder.
$\phi\in \T^*\widetilde M$ ist ein (1,0)-Tensorfeld.\par
In einer lokalen Basis
$\{Y_\alpha\colon={\partial\over\partial y_\alpha}\}$ von $\T M$ ist 
$$\eqalign{
        F_y(x,y,V^*\otimes W)&={\partial F\over\partial y_\alpha}dY_\alpha\cr
        \phi                 &=\phi^\alpha Y_\alpha\;,\cr
             }$$
und damit 
$$C_{y,\phi}(F_y(x,y,V^*\otimes W)\otimes\phi)=
       {\partial F\over\partial y_\alpha}\phi^\alpha\;.$$
\Lemma{Eulergleichungen}{\EULERGL}
{\it Ein kritischer Punkt $f$von $\F$ erf�llt die Gleichung
$$F_y\klamf-C_{X,V^*}(\nabla_XF_{V^*\otimes W}\klamf)=0.$$}
\Beweis Wir rechnen diese Gleichung in lokalen Koordinaten nach.\par
Ist $x$ lokale Karte auf $M$ in einer Umgebung von $p$ und
$y$ auf $\widetilde M$ in einer Umgebung von $f(p)$, so ist 
$$\{p^\alpha_i\colon ={dx_i}\otimes{\partial\over\partial y_\alpha}\}$$
eine Basis von $\Hom_{\partial f|_p}(TM,T\widetilde M)
                \cong T^*M\otimes T\widetilde M$.\par
$$\eqalign{\partial\F(f,\phi)
         = &\int_U{{\partial\over\partial t}F\klamft\dvol}              \cr
         = &\int_U{ {\partial(y_\alpha\circ \exp_ft\phi)\over\partial t}
                    {\partial F\over\partial y_\alpha}
                   +{\partial(p^\alpha_i\circ\partial\exp_ft\phi)\over\partial t}
                    {\partial F\over\partial p^\alpha_i}}              \cr
         = &\int_U{ F_{y_\alpha}\klamf\phi^\alpha
                   +F_{p^\alpha_i}\klamf(\partial \phi)^\alpha_i}      \cr
             }$$
Wegen der Existenz einer Teilung der Eins auf $M$ k�nnen wir ohne 
Einschr�nkung annehmen, da� $\phi$ kompakten Tr�ger in $U$ hat. Damit
ist dann
$$\partial\F(f,\phi)= \int_U{ (F_{y_\alpha}\klamf\phi^\alpha
                      +{\partial\over\partial x_i}
                       F_{p^\alpha_i}\klamf)\phi^\alpha}\;.   $$  
Aufgrund des Fundamentallemmas der Variationsrechnung ist aber schon
$$F_{y_\alpha}\klamf\phi^\alpha
            +{\partial\over\partial x_i}F_{p^\alpha_i}\klamf)=0.    $$
Dies ist aber genau die Gleichung in der Behauptung in lokalen Koordinaten.
\kasten  

