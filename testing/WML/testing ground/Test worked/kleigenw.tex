\Lemma{}{\PHIUNGL}
Es gibt Konstanten $c_i,c_2>0$, so da� f�r die quadratische Form
$Q_f(\phi):=\partial^2I(f,\phi)$ gilt,
$$Q_f(\phi)\geq C_0|\phi|_{H^{1,2}(M,NM)}-|\phi|_{L^2(M,NM)}$$
\Beweis
F�r $v\in T_pM$ ist
$$\eqalign{&<\widetilde R(v,\phi)v,\phi>                     \cr
         = &<\nab_v\nab_\phi v,\phi>-<\nab_\phi\nab_vv>
           -\underbrace{<\nab_{[v,\phi]}v,\phi>}_{\quad =0}   \cr
         = &-<\nab_\phi v,\nab_v\phi>+v<\nab_\phi v,\phi>
           -\underbrace{<\nab_vv,\nab_\phi\phi>}_{\quad =0}       
           +\phi<\nab_vv,\phi>                               \cr
         = &-|\nab_v\phi|^2
           -\underbrace{v\phi<v,\phi>}_{\quad =0}       
           -\underbrace{v<v,\nab_\phi\phi>}_{\quad =0}        
           +\phi<\B(v,v),\phi>                                \cr
      }$$
Deswegen ist

$$\eqalign{Q_f(\phi)
              =&\int_M{<\triangle\phi,\phi>}  
              + \int_M{<{\cal K}\phi,\phi>} - \int_M{<B\phi,\phi>}    \cr        
              =&\int_M{<\nabla\phi,\nabla\phi>} 
              + \int_M{|\nabla\phi|^2} + \int_M{\phi<H,\phi>}
              - \int_M{<{\cal B}^t\phi,{\cal B}^t\phi>}    \cr   
       }$$     
\Para{Eigenschaften des kleinsten Eigenwerts}{}
\Lemma{}{\EIGENWERT}
$$\lambda_f:=\inf\{Q_f(\phi):\phi\in\Hnull^{1,2}(M,NM),
                     |\phi|_{L^2(M,NM)}=1\}$$
ist der kleinste Eigenwert von $L_f$, das hei�t, es gibt ein 
$\phi\in \Hnull^{1,2}(M,NM)$ mit
$$\lambda_f\phi_f=L_f(\phi_f).$$
\Beweis
Wir betrachten die Minimalfolge $\phi_n$ in $\Hnull^{1,2}(M,NM)$
mit $|\phi_n|_{L^2}=1$. 
Wegen {\PHIUNGL} und der Beschr�nktheit der Folge $\partial^2I(f,\phi_n)$
gilt dann
$$|\phi|_{\Hnull^{1,2}(M,NM)}\leq C(\partial^2I(f,\phi_n)+1)\leq C$$
$\Hnull^{1,2}(M,NM)$ ist reflexiv, deswegen kann man ohne Einschr�nkung 
annehmen, da� $\phi_n\rightharpoondown \phi\in \Hnull^{1,2}(M,NM).$\par
Ist $\lambda_f>0$, so ist $Q_f(\phi)$ eine positiv definite
quadratische Form in $\phi$ und als solche schwach unterhalbstetig.
Damit erreicht man
$$\lambda_f\leq Q_f(\phi)
         \leq\lim_{n\to\infty}Q_f(\phi)\leq\lambda_f.$$
Im allgemeinen Fall betrachtet man die quadratische Form
$$\widetilde Q(\phi):=Q(\phi)+c_1|\phi|_{L^2(M,NM)}$$
Sie ist wegen {\PHIUNGL} beschr�nkt und nichtnegativ, es gibt also ein\par
$\widetilde\phi\in\Hnull^{1,2}(M,NM)$ mit $|\phi|_{L^2(M,NM)}=1$ und
$\widetilde Q(\widetilde\phi)=\widetilde\lambda_f$.
$$\eqalign{Q(\widetilde\phi)+c_1=\widetilde Q(\widetilde\phi)
       =&\inf\{\widetilde Q(\phi):
             \phi\in\Hnull^{1,2}(M,NM),|\phi|_{L^2(M,NM)}=1\}        \cr
       =&\inf\{Q(\phi):
             \phi\in\Hnull^{1,2}(M,NM),|\phi|_{L^2(M,NM)}=1\} +c_1   \cr
       =& \lambda_f+c_1                                              \cr
    }$$
Daraus folgt die Behauptung.\kasten
\Bem{}{}
Ist $\lambda_f>0$, so ist $Q_f(\phi)>0$ f�r alle $\phi\in\C^\infty_0(M,TM)$,
das hei�t, $f$ ist ein realtives Minimum des Fl�cheninhalts.\par
Ist umgekehrt $f$ ein relatives Minimum, so ist offensichtlich 
$\lambda_f\geq0$. \par
Ist also $\lambda_f<0$, so ist $f$ kein relatives, und insbesondere kein
starkes Minimum. 
\Lemma{}{\LUMKEHR}
{\it Ist $\lambda_f>0$, so ist $L_f$ invertierbar.}
\Lemma{}{}
{\it $L_f$ h�ngt stetig von der Immersion $f$ ab}
\Bem{}{}
Sei $f:M\longrightarrow\widetilde M$, $r>0$ wie in {\TUBENUM},
dann ist f�r $u\in B_r(0)\subset \C^{2,\alpha}(M,NM)$ die Abbildung
$f_u:=\exp_fu:M\longrightarrow\widetilde M$ eine 
$\C^{2,\alpha}$-Immersion.
Wir betrachten den Differentialoperator
$$L_u:={\cal L}_{f_u}$$
$L_u$ ist offensichtlich linerar und stetig in $u$.
\Lemma {}{\LAMBDASTET}
{\it Ist $\lambda(u)$ der kleinste Eigenwert des Jacobi-Operators $L_u$ von
$M_u$, so ist  das Funktional
$$\eqalign{\lambda  : \C^1(M,NM) & \longrightarrow \RR           \cr
                               u & \mapsto         \lambda (u)   \cr
     }$$
stetig in 0.}


