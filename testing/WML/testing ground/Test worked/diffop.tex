\Para{Differentialoperatoren}{}
\Def{}{}
Wir definieren die 'musikalischen` Isomorphismen
$$\eqalign{(\cdot)^\flat\colon \T M&\longrightarrow \T^* M   \cr
           (\cdot)^\sharp\colon \T^*&M\longrightarrow \T M   \cr 
            }$$
durch
$g(X,\omega^\#)=\omega(X)$ und $X^\flat\colon=g(X,\cdot)$.
\Def{Kontraktion}{}
Ist $A$ ein (1,1)-Tensorfeld auf $M$, das hei�t, $A\in\T^*M\otimes\T M$,
so definieren wir
$$C_{1,2}(A)_p\colon=\Spur(x\longmapsto A(x))$$
$C_{1,2}(A)$ hei�t die Kontraktion von $A$ und ist offensichtlich ein auf 
$M$ global definierte Funktion. 
Im Folgenden betrachen wir die Kontraktion $C_{i,j}$ des $i$-ten 
(kovarianten) Arguments, mit dem $j$-ten (kontravarianten), das einem
$(k,l)$-Tensorfeld ein $(k-1,l-1)$-Tensorfeld zuordnet.
\Def{}{}
Sei $M$ Riemannsche Mannigfaltigkeit, $0\ne a \in \C^0(\T ^*M\otimes \T ^*M)$,
$b\in \C^0(\T ^*M)$ und $c\in \C^0(M)$.\par
Ein {\bf linearer Differentialoperator der Ordnung 2} ist eine 
Abbildung der Form
$$u\mapsto A(u)=C_{1,3}C_{2,4}(a\nabla^2)u + C_{1,2}(b\nabla)u + cu.$$
Dabei ist $C_{i,j}$ die Kontraktion des $i$-ten mit dem $j$-ten 
Elements.\par
$A$ hei�t {\bf elliptisch}, falls es f�r alle $x\in M$ ein $\lambda (x)>0$
gibt, so da� 
$$ \lambda(x) ^{-1}|\xi|^2\leq a(\xi,\xi)\leq \lambda(x)|\xi|^2.$$
$A$ hei�t {\bf gleichm��ig elliptisch} falls $\lambda (x)>\lambda $ 
f�r ein festes $\lambda>0$.
\Bem{}{}
Wegen der Unabh�ngigkeit der Kontraktion von der Kartenwahl ist der obige
Ausdruck wohldefiniert.
\def\X#1{{\scriptstyle{\partial\over\partial x_#1}}}
\Lemma{}{}
{\it In einer lokalen Karte $\phi \colon U\to V$ von $M$ hat $A$ die Form
$$A_\phi (u\circ\phi^{-1}) = a^{ij}_\phi D_iD_j(u\circ \phi^{-1}) 
        + b^i_\phi D_i(u\circ\phi^{-1})
        + c_\phi (u\circ \phi^{-1}). $$}
\Beweis
$$\eqalign{A(u) = 
      & a(\X{i},\X{j}) \nabla_\X{i}\nabla_\X{j} u 
        + b(\X{i})\nabla_\X{i} u + cu                               \cr
    = & a(\X{i},\X{j}){\partial^2\over\partial x_i\partial x_j}u
        + a(\X{i},\X{j}) \nabla_\X{i}\X{j} u                        \cr
      & + b(\X{i})\nabla_\X{i} u + cu                               \cr
    = & a(\X{i}\X{j}){\partial^2\over \partial x_i\partial x_j} u                                   
        + (a(\X{k},\X{l})\Gamma_{kl}^i \X{i}u + b(\X{i}) ) \X{i}u  
        + cu .                                                      \cr
    }$$
W�hlt man jetzt 
$$\eqalign{a_\phi^{ij}&\colon =a(\X{i},\X{j})                                \cr
              b_\phi^i&\colon =b(\X{i})                       
                         +\sum _{k,l=1}^n{a(\X{k},\X{l})\Gamma_{kl}^i} \cr
                c_\phi&\colon =c  ,                                          \cr
       }$$
so erh�lt man die Behauptung. \kasten
\Bem {}{}
$A$ ist (gleichm��ig) elliptisch, wenn $A$ als Differentialoperator 
�ber $V$ (gleichm�s\-sig) elliptisch ist.
\par Man kann alle global definierten linearen Abbildungen, die in lokalen 
Koordinaten 
Differentialoperatoren zweiter Ordnung sind, in dieser Form schreiben.
\Def{}{}
Der durch $\triangle \colon =\Spur \nabla\nabla$ gegebene
Differentialoperator $\triangle \colon \C^2(M)\longrightarrow \C^0(M)$, 
hei�t der Laplace-Beltrami-Operator auf $M$. Er ist offensichtlich 
gleichm��ig elliptisch.
\Lemma{}{\LAPLACELOC}
{\it In lokalen Koordinaten $(x_1,\ldots ,x_n) $ in einer Ungebung um $p$ 
ist 
$$\triangle f = {1\over \sqrt{g}}\sum_{i,j=1}^n
     {\partial \over \partial x_i}
     \bigl(\sqrt{g}\,g^{ij}{\partial f\over \partial x_j}\bigr ).$$}
\Beweis
Siehe [B] p. 45.
\Lemma{}{\HLAPLACE}
{\it Ist $f\colon M\longrightarrow \RR^{n+1}$ isometrische Immersion, so ist
$$H=\triangle f=(\triangle f_1,\ldots,\triangle f_{n+1}).$$}
\Beweis
Seien $v_i$ ein lokaler orthonormaler  Rahmen auf einer Kartenumgebung 
$U$ und $\widetilde \nabla$ der 
euklidische Zusammenhang auf $\RR^{n+1}$. 
Es gilt dann
$$\eqalign{\triangle f_j&=\nabla_{v_i}(\nabla_{v_i}f_j)            \cr
                        &=v_iv_i(f_j)-\nabla_{v_i}v_i(f_j)         \cr
                        &=df(v_i)df(v_i)(y_j)-\nabla_{v_i}v_i(f_j) \cr
                        &=\widetilde\nabla_{df(v_i)}df(v_i)y_j
                          -\widetilde\nabla^T_{df(v_i)}df(v_i)y_j  \cr   
                        &=\widetilde\nabla^N_{df(v_i)}df(v_i)y_j   \cr
                        &=H(y_j)=H_j   \;.                         \cr
          }$$\kasten

