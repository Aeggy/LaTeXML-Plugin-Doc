\Para{Zweite Variation und der Jacobi-Operator}{}
\Def{}{}
Der {\bf Jacobi-Operator zu $\F$} $J_f$ ist die Linearisierung des
Euler-Operators $L$ an der Stelle $f$, das hei�t
$$J_f(\phi)={\partial\over\partial t}L(f_t)$$ 
f�r eine Variation $f_t$ von $f$ in Richtung $\phi$.
\Lemma{}{}
$$\eqalign{J_f(\phi)= & \Bigl[C_{2,3}(F_{yy}\klamf\otimes\phi)\cr
         & - 2C_{4,5}C_{1,2}(\nabla_XF_{y,V^*\otimes W}\klamf\otimes\phi)\cr
         & - C_{6,7}C_{1,5}C_{2,3}(\nabla_X\nabla_X
                          F_{V^*\otimes WV^*\otimes W}\klamf\otimes\phi)
                 \Bigr]^\sharp\;.\cr
}$$
\Beweis
$$\eqalign{J_f(\phi)^\flat=&{\partial\over\partial t}L(f_t)\bigl|_{t=0}   \cr
         = & {\partial\over\partial t}\left[F_y\klamft
             - C_{1,2}(\nabla_XF_{V^*\otimes W}\klamft)\right]            \cr
         = & C_{2,3}(F_{yy}\klamf\otimes\phi)
             - C_{2,3}(F_{yD}\otimes\partial\phi)                         \cr
           & - C_{4,5}C_{1,2}(\nabla_XF_{V^*\otimes W}\klamf\otimes\phi)  \cr
           & - C_{4,5}C_{1,2}(\nabla_XF_{V^*\otimes WD}\klamf
                             \otimes\partial\phi)  \;.                    \cr
       }$$
Mit partieller Integration folgt nun die Behauptung
\Bem{}{}
$J_f$ ist ein linearer Differentialoperator der zweiter Ordnung.
\par Wir betrachten in diesem Paragraphen die quadratische Form
$$Q_f(\phi)\colon =\partial^2\F(f,\phi)$$
\Lemma{}{}
{\it $J_f$ ist der Euler-Operator von ${1\over2}Q_f$ }
\Beweis
$$\eqalign{\int_M{<L_f(\phi),\phi>} 
    = & \partial^2\F(f,\phi)                                             \cr
    = & {d\over dt}{\partial \F(f_t,\phi)\bigl |}_{t=0}                  \cr  
    = & { \int_M{{d\over dt}<L(f_t),\phi>}\Bigl|}_{t=0}             \cr
    = & \int_M{<{\partial\over \partial t}L(f_t),\phi>}\Bigl |_{t=0}
  \int_M{<L(f),\underbrace{\nabla_{\partial\over\partial t}\phi}_{=0}>}\cr 
      }$$
Mit dem Fundamental-Lemma der Variationsrechnung folgt jetzt die Behauptung.
\kasten

