\Para{Funktionenr�ume}{}
\Def{}{}
Ein Schnitt $f$ in einem normierten B�ndel $\bi{E}$ �ber $M$ hei�t 
{\bf integrabel}, falls $\| f\|$ als Funktion auf $M$ integrabel ist.\par
F�r $0\leq p \leq \infty$, $0\leq\alpha\leq  1$ und $k\in\NN$ definieren wir
folgende Normen
$$\eqalign{|f|_{L^p(M,\bi{E})}^p & \colon =\int_M{\| f\|^p }              \cr
           |f|_{\C^0(M,\bi{E})} & \colon =\sup_{x\in M}\|f(x)\|           \cr 
           |f|_{\C^k(M,\bi{E})} & \colon =\sum_{i=1}^k{|\nabla^if|_{\C^0}}\cr
          }$$
F�r $x,y\in M$ und $V\in\pi^{-1}(x)$ sei $p_y(V)$ die Parallelverschiebung
von $V$ in den Punkt $y$ l�ngs einer Kurve von $x$ nach $y$.
$$   |f|_{\C^{k,\alpha}(M,\bi{E})}\colon =|f|_{\C^k(M,\bi{E})} 
                   + \sup_{x,y\in M}{\|p_y(f(x))-f(y)\|\over d(x,y)}$$
\Lemma{}{}
W�hlt man einen endlichen Atlas 
${\cal A}=\{x_i\colon U_i\longrightarrow V_i,i\in I\}$ f�r $M$, so ist 
$\bi{E}\bigl|_{U_i}$ diffeomorph zu $V_i\times F$, wobei 
$F\colon=\pi^{-1}(p)$ f�r ein $p\in U_i$. Man kann dann die obigen 
Normen folgenderweise �quivalent definieren:
$$\eqalign{
    |f|_{L^p(M,\bi{E})}^p & \colon =\sup_i|F_{ij}|_{L^p(V_i,F)}          \cr
    |f|_{\C^0(M,\bi{E})} & \colon =\sup_i|F_i|_{\C^0(V_i,F)}             \cr 
    |f|_{\C^k(M,\bi{E})} & \colon =\sup_j|F_i|_{\C^k(V_i,F)}             \cr
    |f|_{\C^{k,\alpha}(M,\bi{E})} & \colon =|f|_{\C^{k,\alpha}(V_i,F)}   \cr
             }$$
\Beweis Siehe [BC].
\Lemma{}{}
Die Vektorr�ume
$$\eqalign{ L^p(M,\bi{E}) & \colon =\{ \hbox{f ist integrabel und }
                            |f|_{L^p(M,\bi{E})}<\infty\}                      \cr
           \C^k(M,\bi{E}) & \colon =\{|f|_{\C^k(M,\bi{E})}<\infty \}                     \cr
         \C^k_0(M,\bi{E}) & \colon =\{|f|_{\C^k(M,\bi{E})}<\infty ,
                           f\bigl |_{\partial M}=0\}                     \cr
  \C^{k,\alpha}(M,\bi{E}) & \colon =\{|f|_{\C^{k,\alpha}(M,\bi{E})}<\infty\}             \cr
\C^{k,\alpha}_0(M,\bi{E}) & \colon =\{|f|_{\C^{k,\alpha}(M,\bi{E})}<\infty,
                           f\bigl |_{\partial M}=0\}                     \cr
         }$$
sind mit den oben definierten Normen separable Banachr�ume.
\Def{}{}
$H^{k,p}(M,\bi{E})$ ist die Vervollst�ndigung von $\C^\infty(M,\bi{E})$ und
$\Hnull^{k,p}(M,\bi{E})$ die von $\C^\infty_0(M,\bi{E})$ bez�glich der Norm
$$|f|_{H^{k,p}(M,\bi{E})}\colon =\sum_{i=0}^k{|\nabla^i f|_{L^p(M,\bi{E})}}.$$
\Bem{}{}
Wie im Fall von Funktionenr�umen �ber Gebieten $\Omega\subset\RR^n$ ist
$H^{k,p}(M,\bi{E})=L^p(M,\bi{E})$\par
Wenn die Fasern von $\bi{E}$ ein Skalarprodukt tragen, etwa eine 
Riemannsche Metrik, so ist $L^2(M,\bi{E})$ ein Hilbertraum in dem Skalarprodukt
$$(u,v)_{L^2}\colon =\int_M{<u(x),v(x)>_E}.$$
F�r $p>1$ sind $H^{k,p}(M,\bi{E})$ und $\Hnull^{k,p}(M,\bi{E})$ separable Banachr�ume.

\Para{Tubenumgebung und induzierte Mannigfaltigkeit}{}
\Lemma{}{\TUBENUM}
{\it Ist $f\colon M\longrightarrow \widetilde M$ immersierte 
Untermannigfaltigkeit der Kodimension $k$, so gibt es ein $r>0$ und eine 
Immersion
$$\tau\colon M\times B_r(0)\longrightarrow {\cal T}_r(f)\subset \widetilde M$$
auf eine Umgebung  von $f(M)=\tau(M\times\{0\}$.
Dabei ist $B_r(0)$ der offene Einheitsball in $\RR^k$. 
Ist $M$ eingebettet, so ist 
${\cal T}_r(f)$ verm�ge $\tau$ diffeomorph zu $M\times B_r(0)$.}
\Def{}{}
${\cal T}_r(f) $ hei�t {\bf Tubenumgebung von $M$}.
\Beweis
Das Normalenb�ndel ist lokal trivial, das hei�t in einer Umgebung 
$U\subset M$ von $p\in M$ ist $\N U$ b�ndelisomorph zu 
$U\times \RR^k$. Wir betrachten deswegen die Abbildung 
$\Gamma \colon \N U\to \widetilde M$ mit 
$\Gamma (p,v)\colon =\exp_pv$. 
Wegen der Vollst�ndigkeit von $\widetilde M$ und der Differenzierbarkeit 
von $\exp$ ist $\Gamma $ wohldefiniert und differenzierbar auf 
$M\times \RR^k$. Offensichtlich ist 
$\partial\Gamma\bigl |_{(p,s)}=\Id_{\T _p\widetilde M}$, deswegen ist $\Gamma$ 
auf $M\times B_r(0)$ lokal umkehrbar.
\par Angenommen, $M$ ist eingebettet aber 
$\tau\colon  M\times B_r(0)\to {\cal T}_r(f)\colon =\Im(\tau)$ 
ist kein Diffeomorphismus. Dann g�be es 
$Q_i=\tau(\widetilde p_i,\widetilde s_i)=\tau( p_i,s_i)$ mit 
$\lim\widetilde s_i=\lim s_i=0$. Dann konvergieren, nachdem man eventuell
zu einer Teilfolge �bergegangen ist, die $Q_i$ gegen ein $Q\in f(M)$, also 
ist auch $\lim \widetilde p_i=\lim p_i =Q$. $\Gamma $ ist aber auf einer 
ganzen Umbebung von $Q$ injektiv. \kasten
\Kor {}{}
{\it Sei $f\colon M\to \widetilde M$ kompakte, immersierte, orientierbare 
Untermannigfaltigkeit. 
F�r alle Schnitte $u \in \C^{k,\alpha}(M,\N M)$  mit 
$|u|_{\C^0}<r$, ist  $f_u(x)\colon =\tau\circ u(x)=\exp_{f(x)}u(x)$ eine 
$\C^{k,\alpha}$-Immersion.}
\Bem{}{}
F�r $\widetilde M=\RR^{n+1}$ ist $f_u=f+u\circ f$, wenn man $\T _p\RR^{n+1}$
mit $\RR^{n+1}$ identifiziert..
\Def {}{}
$M_u\colon =f_u(M)$ hei�t die {\bf durch $u$ induzierte Mannigfaltig\-keit.}
\Bem{}{}
Ist $M$ eingebettet, so ist auch $M_u$ eingebettet.
F�r $u\in \C^1_0(M,\N M)$ ist $\partial M_u = \partial M$.
\Def {}{} 
Die durch den konstanten Schnitt $s\nu \in \C^\infty(M,\N M)$ 
induzierte Mannigfaltigkeit $M_s\colon =M_{s\nu}$ hei�t die {\bf Parallelfl�che 
zur Entfernung $s$ von $M$}.

