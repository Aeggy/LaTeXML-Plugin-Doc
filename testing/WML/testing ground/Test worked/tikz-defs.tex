%%%%%%%%%%%%%%%%%%%%%%%%%%%%%%%%%%%%%%%%%%%%%%%%%%%%%%%%%%%%%%%%%%%%%%%%%%%%%
%%%%%%%%%%%%%%%%%%%%%%%%%%%%%%%%% tikz-defs %%%%%%%%%%%%%%%%%%%%%%%%%%%%%%%%%
%%%%%%%%%%%%%%%%%%%%%%%%%%%%%%%%%%%%%%%%%%%%%%%%%%%%%%%%%%%%%%%%%%%%%%%%%%%%%
% version: 6
% last edited: 02.10.2013

%%%%%%%%%%%%%%%%%%%%%%%%%%%% necessary includes %%%%%%%%%%%%%%%%%%%%%%%%%%%%
\usepackage{etoolbox}
\usepackage{tikz}
\usetikzlibrary{positioning}
\usetikzlibrary{patterns}
\usetikzlibrary{arrows}
\usetikzlibrary{scopes}
\usetikzlibrary{backgrounds,fit}
\usetikzlibrary{calc}
\usetikzlibrary{decorations.markings}
\usetikzlibrary{shapes}

%%%%%%%%%%%%%%%%%%%%%%%%%%%% text commands %%%%%%%%%%%%%%%%%%%%%%%%%%%%%%%%%

\newcommand{\tikzArrowToText}[1]{%
\mathop{\begin{tikzpicture}[baseline={([yshift=-0.5ex]current 
bounding box.south)}]%
\draw[#1] (0,0) -- (1.1em,0);%
\end{tikzpicture}}}

%%%%%%%%%%%%%%%%%%%%%%%%%%%% format definitions %%%%%%%%%%%%%%%%%%%%%%%%%%%%

\tikzset{nodelabel/.style={font=\footnotesize}}
\tikzset{edgelabel/.style={font=\footnotesize}}
\tikzset{henodenumber/.style={font=\footnotesize}}
\tikzset{undiredge/.style={shorten >=0pt, shorten <=0pt}}
\tikzset{partialedge/.style={-left to, thick}}
\tikzset{dotline/.style={dotted, shorten <=5pt, shorten >=5pt}}

\tikzset{graphedge/.style={->, >=latex}}
\tikzset{bigraphedge/.style={<->, >=latex}}
\tikzset{invgraphedge/.style={<-, >=latex}}
\tikzset{henode/.style={draw, rectangle, rounded corners=2.5pt}}
\tikzset{stdnode/.style={draw, circle}}
\tikzset{mapedge/.style={->, >=stealth', dashed, color=orange}}

\tikzset{ruleappedge/.style={double, ->}}
\tikzset{mor-tot-inj/.style={catt-catt}}
\tikzset{mor-tot/.style={-catt}}
\tikzset{mor-parr/.style={-catpr}}
\tikzset{mor-parl/.style={-catpl}}
\tikzset{mor-minor/.style={|-catt}}
\tikzset{mor-subgraph/.style={)-catt}}
\tikzset{mor-indsubgraph/.style={open triangle 45 reversed-catt}}
\tikzset{mor-genorder/.style={]-catt}}
\tikzset{noedge/.style={decoration={markings,mark=at position 0.5 with 
{\arrow[scale=1.6,color=red]{|}}}, postaction={decorate}}}
\tikzset{closure/.style={postaction={decorate, decoration={raise=4pt, 
markings, mark=at position 1 with {\node[scale=0.65]{\textbf{*}};}}}}}

\tikzset{backstepedge/.style={-open triangle 45, shorten <=0pt, shorten >=0pt}}

\tikzset{graphbox/.style={draw, dashed, rounded corners=2mm, outer sep=5pt, on 
background layer}}
\tikzset{graphboxgrey/.style={draw, dashed, black!50, rounded corners=2mm, 
outer sep=5pt, on background layer}}

%%%%% definition of arrow tips for total category edges %%%%%

\newdimen\arrowsize
\pgfarrowsdeclare{recatt}{catt}
{
%\pgfsetlinewidth{0.4pt}
\arrowsize=1pt
\advance\arrowsize by .5\pgflinewidth
\pgfarrowsleftextend{-2.5\arrowsize-.5\pgflinewidth}
\pgfarrowsrightextend{.5\pgflinewidth}
}
{
\pgfsetlinewidth{0.4pt}
\arrowsize=0.8pt
\advance\arrowsize by .5\pgflinewidth
\pgfsetdash{}{0pt} % do not dash
\pgfsetroundjoin % fix join
\pgfsetroundcap % fix cap
\pgfpathmoveto{\pgfpointorigin}
\pgfpatharc{270}{190}{2.8\arrowsize}
\pgfusepathqstroke
\pgfpathmoveto{\pgfpointorigin}
\pgfpatharc{90}{170}{2.8\arrowsize}
\pgfusepathqstroke
}

\pgfarrowsdeclarereversed{catt}{recatt}{recatt}{catt}
\pgfarrowsdeclaredouble{recatcatt}{catcatt}{recatt}{catt}
\pgfarrowsdeclaredouble{catcatt}{recatcatt}{catt}{recatt}

%%%%% definition of arrow tips for partial category edges %%%%%

\pgfarrowsdeclare{helperpr}{helperpr}
%\pgfarrowsdeclare{recatpr}{catpr}
{
%\pgfsetlinewidth{0.4pt}
\arrowsize=1pt
\advance\arrowsize by .5\pgflinewidth
\pgfarrowsleftextend{-2.5\arrowsize-.5\pgflinewidth}
\pgfarrowsrightextend{.5\pgflinewidth}
}
{
\pgfsetlinewidth{0.55pt}
\arrowsize=0.8pt
\advance\arrowsize by .5\pgflinewidth
\pgfsetdash{}{0pt} % do not dash
\pgfsetroundjoin % fix join
\pgfsetroundcap % fix cap
\pgfpathmoveto{\pgfpointorigin}
\pgfpatharc{90}{170}{3\arrowsize}
\pgfusepathqstroke
}

\pgfarrowsdeclare{helperpl}{helperpl}
%\pgfarrowsdeclare{recatpl}{catpl}
{
%\pgfsetlinewidth{0.4pt}
\arrowsize=1pt
\advance\arrowsize by .5\pgflinewidth
\pgfarrowsleftextend{-2.5\arrowsize-.5\pgflinewidth}
\pgfarrowsrightextend{.5\pgflinewidth}
}
{
\pgfsetlinewidth{0.55pt}
\arrowsize=0.8pt
\advance\arrowsize by .5\pgflinewidth
\pgfsetdash{}{0pt} % do not dash
\pgfsetroundjoin % fix join
\pgfsetroundcap % fix cap
\pgfpathmoveto{\pgfpointorigin}
\pgfpatharc{270}{190}{3\arrowsize}
\pgfusepathqstroke
}

\pgfarrowsdeclarereversed{rehelperpr}{rehelperpr}{helperpr}{helperpr}
\pgfarrowsdeclarereversed{rehelperpl}{rehelperpl}{helperpl}{helperpl}
\pgfarrowsdeclarealias{catpl}{catpl}{helperpl}{helperpl}
\pgfarrowsdeclarealias{catpr}{catpr}{helperpr}{helperpr}
%\pgfarrowsdeclarereversed{catpr}{recatpl}{recatpl}{catpl}
%\pgfarrowsdeclarereversed{catpl}{recatpr}{recatpr}{catpr}
%\pgfarrowsdeclarereversed{catpr}{recatpl}{recatpl}{catpl}
%\pgfarrowsdeclaredouble{recatcatpr}{catcatpr}{recatpr}{catpr}
%\pgfarrowsdeclaredouble{catcatpr}{recatcatpr}{catpr}{recatpr}
%\pgfarrowsdeclaredouble{recatcatpl}{catcatpl}{recatpl}{catpl}
%\pgfarrowsdeclaredouble{catcatpl}{recatcatpl}{catpl}{recatpl}

%%%%%%%%%%%%%%%%%%%%%%%%%%%%%% box definitions %%%%%%%%%%%%%%%%%%%%%%%%%%%%%%

% places a dashed box around a center with using the specified height and width
% usage: \DynamicGraphbox[name of box]{center of box}{half of boxwidth}{half 
%of boxheight}
\newcommand\DynamicGraphbox[4][grbox]{%
	\begin{pgfonlayer}{background}%
	\node[circle, above left= #4 and #3 of #2] (top_left) {};%
	\node[circle, below right= #4 and #3 of #2] (bottom_right) {};%
	\node[fit=(top_left) (bottom_right), outer sep=5pt] (#1) {};%
	\draw[dashed, black!50,rounded corners=2mm] 
		($(#1.north west) + (5pt,-5pt)$) -- ($(#1.north east) + (-5pt,-5pt)$) --
		($(#1.south east) + (-5pt,5pt)$) -- ($(#1.south west) + (5pt,5pt)$) -- 
		cycle;%
	\end{pgfonlayer}%
}%

% places a dashed box around a collection of nodes with the specified height 
% and width as minimum size
% usage: \BoundedGraphbox[name of box]{collection of nodes}{width}{height}
\newcommand\BoundedGraphbox[4][grbox]{%
	\begin{pgfonlayer}{background}%
	  \node[graphboxgrey, fit=#2, minimum height=#4, minimum width=#3] (#1) {};%
	\end{pgfonlayer}%
}%

% places a dashed box around a list of nodes
% usage: \Interfacebox[name of box]{list of names or coordinates of nodes (if 
%necessary in brackets)}
\newcommand\Interfacebox[2][ifbox]{%
  \begin{pgfonlayer}{background}%
    \node[fit=#2] (#1) {};%
    \draw[dashed,black,rounded corners=2mm] 
          (#1.north west) -- (#1.north east) --
          (#1.south east) -- (#1.south west) -- cycle;%
  \end{pgfonlayer}%
}%

% places a normal lined box around a list of nodes
% usage: \Boldbox[name of box]{list of names or coordinates of nodes (if 
%necessary in brackets)}
\newcommand\Boldbox[2][boldbox]{%
  \begin{pgfonlayer}{background}%
    \node[fit=#2] (#1) {};%
    \draw[black,rounded corners=2mm] 
          (#1.north west) -- (#1.north east) --
          (#1.south east) -- (#1.south west) -- cycle;%
  \end{pgfonlayer}%
}%

% places a dashed box around a center with a specified heigth and width
% usage: \StaticGraphbox[name of box]{center node position}{width of 
%box}{height of box}{alignment and displayed name (komma seperated)}
\makeatletter
\@ifclassloaded{beamer}{%
\newcommand<>{\StaticGraphbox}[5][grbox]{%
\def\hAlignCJS{right}%
\def\vAlignCJS{top}%
\def\nameCJS{}%
\renewcommand{\do}[1]{%
  \ifthenelse{\equal{##1}{top}}{\def\vAlignCJS{top}}{%
  \ifthenelse{\equal{##1}{bottom}}{\def\vAlignCJS{bottom}}{%
  \ifthenelse{\equal{##1}{right}}{\def\hAlignCJS{right}}{%
  \ifthenelse{\equal{##1}{left}}{\def\hAlignCJS{left}}{\def\nameCJS{##1}}}}}}%
  \docsvlist{#5}%
  \begin{pgfonlayer}{background}
    \node[circle, above left= #4 and #3 of #2] (top_left) {};
    \node[circle, above right= #4 and #3 of #2] (top_right) {};
    \node[circle, below left= #4 and #3 of #2] (bottom_left) {};
    \node[circle, below right= #4 and #3 of #2] (bottom_right) {};
    \node[fit=(top_left) (bottom_right), outer sep=5pt] (#1) {} ;
    \draw#6[dashed, black!50,rounded corners=2mm] 
    ($(#1.north west) + (5pt,-5pt)$) -- ($(#1.north east) + (-5pt,-5pt)$) --
    ($(#1.south east) + (-5pt,5pt)$) -- ($(#1.south west) + (5pt,5pt)$) -- 
    cycle ;
    \node#6[left] (boxname) at (\vAlignCJS_\hAlignCJS.east) {\nameCJS};
  \end{pgfonlayer}
}}{%
\newcommand{\StaticGraphbox}[5][grbox]{%
\def\hAlignCJS{right}%
\def\vAlignCJS{top}%
\def\nameCJS{}%
\renewcommand{\do}[1]{%
  \ifthenelse{\equal{##1}{top}}{\def\vAlignCJS{top}}{%
  \ifthenelse{\equal{##1}{bottom}}{\def\vAlignCJS{bottom}}{%
  \ifthenelse{\equal{##1}{right}}{\def\hAlignCJS{right}}{%
  \ifthenelse{\equal{##1}{left}}{\def\hAlignCJS{left}}{\def\nameCJS{##1}}}}}}%
  \docsvlist{#5}%
  \begin{pgfonlayer}{background}
    \node[circle, above left= #4 and #3 of #2] (top_left) {};
    \node[circle, above right= #4 and #3 of #2] (top_right) {};
    \node[circle, below left= #4 and #3 of #2] (bottom_left) {};
    \node[circle, below right= #4 and #3 of #2] (bottom_right) {};
    \node[fit=(top_left) (bottom_right), outer sep=5pt] (#1) {} ;
    \draw[dashed, black!50,rounded corners=2mm] 
    ($(#1.north west) + (5pt,-5pt)$) -- ($(#1.north east) + (-5pt,-5pt)$) --
    ($(#1.south east) + (-5pt,5pt)$) -- ($(#1.south west) + (5pt,5pt)$) -- 
    cycle ;
    \node[left] (boxname) at (\vAlignCJS_\hAlignCJS.east) {\nameCJS};
  \end{pgfonlayer}
}}
\makeatother

%%%%%%%%%%%%%%%%%%%%%%%%%%%%% commands for dots %%%%%%%%%%%%%%%%%%%%%%%%%%%%%

% places the dots in a vertical line at the specified position
% usage: \verticaldots{position of the middle dot}{width and height of a dot}
\providecommand{\verticaldots[2]}{
	\path (#1) node[fill=black, circle, minimum width=#2, minimum height=#2,inner sep=0] {}
	+(0,3*#2) node[fill=black, circle, minimum width=#2, minimum height=#2,inner sep=0] {}
	+(0,-3*#2) node[fill=black, circle, minimum width=#2, minimum 
	height=#2,inner sep=0] {};}