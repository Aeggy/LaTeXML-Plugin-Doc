\newcommand{\filespath}{Jancl_style_V2.1.3/Styles/}
\documentclass{\filespath jancl}
%\documentclass[a4paper]{llncs}
%Version 19.9.1997 10:15
%Send off 24.6.1998
%Last Changed 1.10.1998 10:52
%Last Changed 25.9.2008 by Manfred
%\usepackage{rotating}
%\usepackage{epsfig}
%\pagestyle{plain}

\journal{??}{??}{??}{-}
\title[Reasoning with Presuppositions]{Reasoning without Believing: On the Mechanization of Presuppositions and Partiality}
\author{Manfred Kerber$^1$ and Michael Kohlhase$^2$}
\address{\begin{minipage}[t]{0.45\textwidth}\begin{center}$^1$School of Computer Science\\
University of Birmingham\\
%Birmingham B15 2TT, England\\
{\tt \sperrklein{http://www.cs.bham.ac.uk}/\~{}\sperrklein{mmk}}\end{center}\end{minipage}
\qquad \begin{minipage}[t]{0.45\textwidth}\begin{center}$^2$Computer Science\\
Jacobs University Bremen\\
%66041 Saarbr\"ucken, Germany\\
{\tt
  \sperrklein{http://kwarc.info}/\sperrklein{kohlhase}}\end{center}\end{minipage}}
\abstract{ It is well-known that many relevant aspects of everyday
  reasoning based on natural language cannot be adequately expressed
  in classical first-order logic.  In this paper we address two of the
  problems, firstly that of so-called {\em presuppositions\/},
  expressions from which it is possible to draw implicit conclusions,
  which classical logic normally does not warrant, and secondly the
  problem of {\em partiality\/} and the adequate treatment of
  undefined expressions.  In natural language, presuppositions are
  quite common, they can, however, only insufficiently be modeled in
  classical first-order logic.  For instance, in the case of universal
  restricted quantification in natural language it is typically
  presupposed that these restrictions are non-empty, whereas in
  classical logic it is only assumed that the whole universe is
  non-empty.  Conversely, all constants mentioned in classical
  logic are presupposed to denote something, while it makes no problems to speak
  about hypothetical objects in everyday language.  Similarly,
  undefined expressions can be handled in natural language discourses
  and utterances are not only classified into the two categories
  `true' and `false'.  This has led to the development of various
  better-suited many-valued logics.
  
  By combining different approaches we can now give a {\em static\/}
  description of presuppositions and undefinedness within the same
  framework. Additionally, we have developed an efficient
  mechanization of the induced consequence relation by combining
  methods from many-valued truth-functional logics and sort techniques
  developed for search control in automated theorem proving.  }

\keywords{presuppositions, many-valued logics, undefinedness, resolution calculus}

\newcommand\COMP{{\sf TheCompany}}
\usepackage{amssymb}
\def\sperr#1{\def\sperrdist{0.3ex}\SperrRest#1\endlistxxx}
\def\sperrklein#1{\def\sperrdist{-0.12ex}\SperrRest#1\endlistxxx}
\def\endlistxxx{\endlistxxx}
\def\SperrRest{\afterassignment\SperrZeichen\let\next= }
\def\SperrZeichen{\ifx\next\endlistxxx \let\next\relax\kern-\sperrdist
                  \else\next\kern\sperrdist\let\next\SperrRest\fi\next}
\mathchardef\Gamma="7000 \mathchardef\Delta="7001 \mathchardef\Theta="7002
\mathchardef\Lambda="7003 \mathchardef\Xi="7004   \mathchardef\Pi="7005
\mathchardef\Sigma="7006  \mathchardef\Upsilon="7007 \mathchardef\Phi="7008
\mathchardef\Psi="7009 \mathchardef\Omega="700A

\def\Sorts{{\cS}}                     %set of sort symbols
\def\Vars{{\cV}}                      %set of variables
\def\Signat{{\Sigma}}                 %Signature
\def\Signats{{\Sigma_\Sorts}}         %sorted Signature
\def\Cnf{\hbox{\bf CNF}}               %clause normal form
\def\cwst{{\bf wsT}}
\def\Funcs{{\mbox{$\cF$}}}
\def\PFuncs{{\mbox{$\cF_{\kern -0.4ex p}$}}}
\def\Box{\hbox{\vrule\vbox{\hrule width 1.3ex\vskip 1.3ex\hrule}\vrule}}

\def\Var{\hbox{\bf Var}}
\def\free{\hbox{\bf Free}}
\def\Dom{\hbox{\bf Dom}}
\def\Rel{\hbox{\bf Rel}}
\def\Rels{\Rel^{\Sorts}}
\def\Reld{\Rel^{\Defined}}
\def\restrict#1#2{{#1\bigr|_{#2}}}
\def\to{\longrightarrow}
\def\hat{\widehat}
\def\setdivider{\;\bigl|\;}
\def\setminus{\,\raisebox{0.4ex}{$\scriptscriptstyle\backslash$}\,}
\def\deq{{\;\colon\kern-.2em=\;}}
\def\eqd{{\;=\kern-.1em\colon\;}}

\def\phi{\varphi}
\def\cA{{\cal A}}\def\cB{{\cal B}}\def\cC{{\cal C}}
\def\cD{{\cal D}}\def\cE{{\cal E}}\def\cF{{\cal F}}
\def\cG{{\cal G}}\def\cH{{\cal H}}\def\cI{{\cal I}}
\def\cJ{{\cal J}}\def\cK{{\cal K}}\def\cL{{\cal L}}
\def\cM{{\cal M}}\def\cN{{\cal N}}\def\cO{{\cal O}}
\def\cP{{\cal P}}\def\cQ{{\cal Q}}\def\cR{{\cal R}}
\def\cS{{\cal S}}\def\cT{{\cal T}}\def\cU{{\cal U}}
\def\cV{{\cal V}}\def\cW{{\cal W}}\def\cX{{\cal X}}
\def\cY{{\cal Y}}\def\cZ{{\cal Z}}

%first retain the original versions of \S and \P
\def\A{{\bf A}}\def\B{{\bf B}}\def\C{{\bf C}}\def\D{{\bf D}}
\def\E{{\bf E}}\def\F{{\bf F}}\def\G{{\bf G}}\def\H{{\bf H}}
\def\I{{\bf I}}\def\J{{\bf J}}\def\K{{\bf K}}\def\L{{\bf L}}
\def\M{{\bf M}}\def\N{{\bf N}}\def\O{{\bf O}}\def\P{{\bf P}}
\def\Q{{\bf Q}}\def\R{{\bf R}}\def\S{{\bf S}}\def\T{{\bf T}}
\def\U{{\bf U}}\def\V{{\bf V}}\def\W{{\bf W}}\def\X{{\bf X}}
\def\Y{{\bf Y}}\def\Z{{\bf Z}}


%%%%%%%%%%%%%%%%%%%%%%%%%%%%%%%%%%%%%%%%%%%%%%%%%%%%%%%%%%%%%%%%%%%%%%%%%%%%%%%%%%%%%%%%%%%%%%%%%%
% Some stuff related to the syntax of Logic

\def\skolem#1{{\cal E}_{{#1}(o{#1})}}
\def\Sortpred{\cP}                  
\def\AH{\cQ}                        % Andrews-Henkin formulation of simple TT

\def\sdot{\rule{.3ex}{.3ex}\hspace{-0.4ex}\;} %%%% was \hskip.1ex\;
\def\lnot{\neg}
\def\fol{\Rightarrow}                % Logische Folgerung


%%%%%%%%%%%%%%%%%%%%%%%%%%%%%%%%%%%%%%%%%%%%%%%%%%%%%%%%%%%%%%%%%%%%%%%%%%%%%%%%%%%%%%%%%%%%%%%%%%%
% The symbols and their positioning rules for the end of proofs in the proof environment


\renewenvironment{proof}{{\bf Proof: }}{\kasten}
\newenvironment{sketch}{{\bf Proof sketch: }}{\kasten}

\def\kasten{\hfil\null\nobreak\hfill
            \hbox{\vrule\vbox{\hrule width 6 pt\vskip 6pt\hrule}\vrule}
            \par\smallskip}
\def\nocubekasten{\hfil\null\nobreak\hfill
            \nocubechar
            \par\smallskip}


%%%%%%%%%%%%%%%%%%%%%%%%%%%%%%%%%%%%%%%%%%%%%%%%%%%%%%%%%%%%%%%%%%%%%%%%%%%%%%%%%%%%%%%%%%%%%%%%%%
% Typesetting Deductions (obtained from F. Pfenning)
%
% There are two basic constructors of proof trees \ian and \ibn
% for aplications of unary and binary inference rules.
% There are versions \ianc and \ibnc that center the names of the rules.
% For representing subtrees we have a macro \subtree that has 2 lines 
% and a name for the subtre as of yet is only supports linear subtrees.


\def\ian#1#2#3{{\lineskip 4pt\inrulean{#1}{#2}{{\it #3}}}} %premise, conc, name
\def\ibn#1#2#3#4{{\lineskip 4pt\inrulebn{#1}{#2}{#3}{{\it #4}}}}  %prem1, prem2, conc, name
\def\ianc#1#2#3{{\lineskip 4pt\lowerhalf{\inruleanhelp{#1}{#2}{{\it #3}}%
                   \box\tempb\hskip\wd\tempd}}}
\def\ibnc#1#2#3#4{{\lineskip 4pt\ianc{#1\quad\qquad #2}{#3}{#4}}}
\def\subtree#1#2#3{{\lineskip 4pt\doubleinrulean{#1}{#2}{#3}}} %prem, conc, name

\def\lowerhalf#1{\hbox{\raise -0.8\baselineskip\hbox{#1}}}


\newbox\tempa
\newbox\tempb
\newdimen\tempc
\newbox\tempd

\def\mud#1{\hfil $\displaystyle{#1}$\hfil}
\def\rig#1{\hfil $\displaystyle{#1}$}

\def\inruleanhelp#1#2#3{\setbox\tempa=\hbox{$\displaystyle{\mathstrut #2}$}%
                        \setbox\tempd=\hbox{$\; #3$}%
                        \setbox\tempb=\vbox{\halign{##\cr
        \mud{#1}\cr
        \noalign{\vskip\the\lineskip}%
        \noalign{\hrule height 0pt}%
        \rig{\vbox to 0pt{\vss\hbox to 0pt{\copy\tempd \hss}\vss}}\cr
        \noalign{\hrule}%
        \noalign{\vskip\the\lineskip}%
        \mud{\copy\tempa}\cr}}%
                      \tempc=\wd\tempb
                      \advance\tempc by \wd\tempa
                      \divide\tempc by 2 }

\def\inrulean#1#2#3{{\inruleanhelp{#1}{#2}{#3}%
                     \hbox to \wd\tempa{\hss \box\tempb \hss}}}
\def\inrulebn#1#2#3#4{\inrulean{#1\quad\qquad #2}{#3}{#4}}


\def\rulespacing{\renewcommand{\arraystretch}{3} \arraycolsep 0em}
\def\Rulespacing{\renewcommand{\arraystretch}{4} \arraycolsep 0em}

\def\doubleinruleanhelp#1#2#3{\setbox\tempa=\hbox{$\displaystyle{\mathstrut #2}$}%
                        \setbox\tempd=\hbox{$\; #3$}%
                        \setbox\tempb=\vbox{\halign{##\cr
        \mud{#1}\cr
        \noalign{\vskip\the\lineskip}%
        \noalign{\hrule height 0pt}%
        \rig{\vbox to 0pt{\vss\hbox to 0pt{\copy\tempd \hss}\vss}}\cr
        \noalign{\hrule}%
        \noalign{\vskip\the\lineskip}%
        \noalign{\hrule}%
        \noalign{\vskip\the\lineskip}%
        \mud{\copy\tempa}\cr}}%
                      \tempc=\wd\tempb
                      \advance\tempc by \wd\tempa
                      \divide\tempc by 2 }

\def\doubleinrulean#1#2#3{{\doubleinruleanhelp{#1}{#2}{#3}%
                     \hbox to \wd\tempa{\hss \box\tempb \hss}}}



%%%%%%%%%%%%%%%%%%%%%%%%%%%%%%%%%%%%%%%%%%%%%%%%%%%%%%%%%%%%%%%%%%%%%%
% Editorials
%%%%%%%%%%%%%%%%%%%%%%%%%%%%%%%%%%%%%%%%%%%%%%%%%%%%%%%%%%%%%%%%%%%%%%

\long\def\ednote#1{\marginpar{EDNOTE}\footnote{\it #1}\message{ednote!}}
%\long\def\ednote#1{}
\newenvironment{newpart}[1]{\begin{quote}\marginpar{BNP}\footnote{\it #1}\message{note!}\it}{\end{quote}\marginpar{ENP}}
\long\def\ignore#1{}



%%%%%%%%%%%%%%%%%%%%%%%%%%%%%%%%%%%%%%%%%%%%%%%%%%%%%%%%%%%%%%%%%%%%%%
% Judgments
%%%%%%%%%%%%%%%%%%%%%%%%%%%%%%%%%%%%%%%%%%%%%%%%%%%%%%%%%%%%%%%%%%%%%%

\def\vd{\null\mathrel{\vdash}}
\def\gvd{\Gamma \vd}

\def\th{\null\mathrel{\vdash\!\!\!\vdash}}
\def\gth{\Gamma \th}

\let\phi\varphi
\newcommand{\allin}[2]{\forall #1_#2\,\,}
\newcommand{\exin}[2]{\exists #1_#2\,\,}
\newcommand{\Defined}{{\mathfrak D}}
\newcommand{\Secure}{{\mathfrak S}}
\newcommand{\Undefined}{{\mathfrak U}}
\newcommand{\true}{{\sf t}}
\newcommand{\undefined}{{\sf u}}
\newcommand{\false}{{\sf f}}
\newcommand{\twts}{{{\sf t}^+}}
\newcommand{\twfs}{{{\sf f}^+}}
\newcommand{\twti}{{{\sf t}^-}}
\newcommand{\twfi}{{{\sf f}^-}}
\newcommand{\twu}{{\sf u}}
\newcommand{\Tval}{{\cal B}}

\newcommand{\sortpair}[2]{\lambda{#1}{#2}}
\newcommand{\sortel}{\mbox{$<\kern-1.0em -$}}
\newcommand{\notsortel}{\mbox{$\not<\kern-1.0em -$}}
\newcommand{\set}[1]{\left\{#1\right\}}
%\newcommand{\case}[1]{\left\{\begin{array}{l@{\hspace{2ex}}l}#1\end{array}\right.}
\newcommand{\Real}{{\mathfrak R}}

\newcommand{\cond}{{\bf COND}}

\renewcommand{\Real}{{\mathchoice{\displaystyle\rm I\hskip-0.22em R}%
{\textstyle\rm I\hskip-0.22em R}%
{\scriptstyle\rm I\hskip-0.15em R}%
{\scriptscriptstyle\rm I\hskip-0.15em R}}}
\newcommand{\Realnull}{{\Real^*}}

\headheight15pt
\newcommand{\comment}[1]{}

\def\restconst{{\cal R\kern-.4ex C}}
\def\scapply#1#2{{\left.#1\right\downarrow(#2)}}
%\def\cond{\sortconst}
\def\presuppl{{\cal P\kern-.4ex L}}
\def\presuppres{{\cal R\kern-.3ex P\kern-.4ex L}}
\def\Preds{{\cal P}}
%\def\tf{!}
\def\tf{{\bf D}}
\def\holds{{\bf T}}
\def\deft{{\bf S}}
\def\Cnfh{\Cnf_H}
\def\Herbrandbase{{\cH}}
\def\TermDecs{\cal T\kern-.3ex D}
\setcounter{secnumdepth}{2}

\def\True{{\sf T}}
\def\updiv{{\bigl|\kern-1pt\bigl|}}  %unification problem divider

\newcommand{\all}[3]{\forall {#1}_{#2}\sdot #3}
\newcommand{\ex}[3]{\exists {#1}_{#2}\sdot #3}
\newcommand{\und}{\vee}\def\implies{\rightarrow}

\def\sf{\tt}

\newcommand{\bd}{\begin{tabular}{l@{$\;\;$}l}}
\newcommand{\ed}{\end{tabular}}
\def\impl{\implies}\def\non{\neg}



%\date{}

\begin{document}
\maketitle

\section{Introduction}
\label{sec:intro}

Classical first-order logic (FOL) has often been used as a basis for
natural language semantics. But there are considerable discrepancies
between the meaning of everyday language and the formal semantics of
its naive translation into FOL.  In this paper we address two
important aspects, the first is about so-called
{\em presuppositions\/}\footnote{For instance, albeit in FOL typically there is the
  assumption that the universe of discourse is non-empty, it is not
  assumed that restrictions as usually expressed by implications are
  non-empty.  A direct translation of ``All humans are mortal'' into
  FOL does not presuppose that there are any humans at all but only that the
  universe of discourse is non-empty.}, that is,
implicit assumptions in everyday reasoning about what exists and what
not, the second is about the treatment of {\em undefined expressions}.

These phenomena have been widely studied in the philosophy of language from a semantic
point of view, see~\cite{Beaver:p97,BeaZee:a07} for surveys. An
important view is that presuppositions form a common ground of
knowledge between the sender and the receiver of an utterance. This
view has been refined to a description that the speaker and the hearer
are both cooperative. Presuppositions capture in this model
uncontroversial knowledge, which is not necessarily already known to
the hearer, see~\cite{Simons06}. 

One of the more logic-oriented ways to model presuppositions
is to use a four-valued logic~\cite{Bergmann:patdl81}. We
take this logic as a starting point, extend it to a five-valued
version that is capable of capturing undefinedness as well, and
mechanize this logic by a resolution calculus.


We will use the remainder of the introduction to pinpoint the
contributions of this paper, review the problem and some related work
which is most relevant to our work, that is, presupposition
logics\footnote{Since we are mainly interested in the inference aspect
  of presuppositions, we will not go into the large body of linguistic
  literature on presuppositions.}, the logical treatment of
undefinedness, and automated theorem in many-valued and sorted
logics. This provides the background information for the formal
development of our logical system, presented in
Section~\ref{sec:presuppl}. Then we will present a resolution calculus
and briefly sketch the soundness and completeness results
(Section~\ref{sec:presupplres}). We will also discuss which aspect of
natural language (presupposition, undefinedness) is captured by which
aspect of the formalism. Finally, we explain the method by two simple
examples (Section~\ref{sec:example}) before we conclude with a
discussion of further work.


\subsection{Contribution of this paper}
\label{sec:contrib}

We present a five-valued logic for the formalization of everyday reasoning with
presuppositions. This system enhances the four-valued system proposed by Berg\-mann
in~\cite{Bergmann:patdl81} and mechanized by the authors in
\cite{KeKo:arcfp96} by providing a fifth truth value for syntactical
undefinedness
(for instance, application of functions to objects outside their domain). It adds a
further degree of partiality to the system and is treated in a Kleene way as described
in~\cite{KeKo94,KeKo:atcfpf96}. Furthermore we present a sound and complete
resolution calculus for our system which uses a sort mechanism to capture
Bergmann's restricted quantifications.

This paper is {\em not\/} intended as a linguistic account of the semantics of
natural language sentences involving presuppositions. Such accounts -- see for
instance~\cite{Sandt:ppaar92,Krahmer:dap95,Beaver:p97,BeaZee:a07} -- have
nowadays mostly turned away from multi-valued logics towards so-called dynamic
approaches such as discourse representation theories (DRT), see
e.g.~\cite{KamRey:fdtl93} or~\cite{EijckUnger07}, where
presuppositions are described via
public announcement updates and common knowledge in a multimodal
epistemic logic. Rather than a linguistic account, our work is an
investigation of the underlying logical principles and the status of
partiality, which also plays a major role in the linguistic theories.
See cf.~\cite{Krahmer:dap95} for a discussion. In particular, this paper does
not make any claim about the process of constructing logical representations
from natural language utterances.

Since the discourse representation theories used in the linguistic accounts are
essentially dynamic versions of FOL, we are convinced that the methods and principles
presented in this paper can be transported to the linguistic theories to give them more
leverage.  Our logic $\presuppl$ is a generalization of FOL in a direction that is
orthogonal to dynamic aspects so that the ideas from both systems can be combined into a
joint logical system. In fact Krahmer's work already takes a step in that direction: it
uses a version of discourse representation theory that models partiality using ideas from
Kleene's weak three-valued logic, which has well-known disadvantages for modeling
presuppositions. Since direct inferences in dynamic discourse logics seem to be a
difficult problem, the present paper studies the static case in isolation.

\subsection{Presuppositions}\label{subsec:presupp}
There are two different kinds of presuppositions: the {\em
  quantificational} presuppositions which mean that the domain of quantifications
is non-empty and the {\em existential} presuppositions which entail the existence of
denotations of constants.  In natural language, the first ones are mandatory, whereas
the second kind is defeasible (it is possible to talk about
non-existing entities in natural language).  Surprisingly enough, the
standard semantics of FOL treats the two kinds almost the
opposite way: constants must always have denotations, that is, just
speaking about an object means that it must exist (for instance,
speaking about a dragon, means that there is one), while
quantifications are unrestricted and therefore range always over the
whole (albeit non-empty) universe.  In classical logic the standard way to
restrict a quantification is the use of an implication, which may,
however, have an antecedent with empty domain.

This problem has been recognized quite early in the field of philosophical logics and
different kinds of utterances are distinguished.  For
instance Strawson~\cite[p.203 ff]{Strawson73} analyses the sentence ``The king of
France is wise.'' and relates his view to that of
Russell. Russell~\cite{Russell05} says:\vspace{-0.5ex}
\begin{quote}
\sl  Firstly the sentence is significant, that is, its meaning can be understood.
  Secondly it is true only if there existed in fact one and only one king of France,
  and if he were wise.
\end{quote}\vspace{-0.5ex}
Strawson argues that -- in contrast to Russell's view -- the sentence is not
necessarily true or false.  He underpins this view as follows:\vspace{-0.5ex}
\begin{quote}
  {\sl Now suppose someone were in fact to say to you with a
    perfectly serious air: `The king of France is wise.'  Would you
    say, `That's untrue' ...  suppose he went on to\sperr{ ask }you
    whether you thought that what he had just said was true, or was
    false ... You might ... say something like: `I'm afraid you must
    be under a misapprehension.  France is not a monarchy. There is no
    king of France.'}
\end{quote}
Later on Strawson explains, why the sentence is significant, namely, since
\begin{quote}
  {\sl the sentence\sperr{ could }be used, in certain circumstances, to say
    something true or false, that the expression\sperr{ could }be used, in certain
    circumstances, to mention a particular person\ldots}
\end{quote}\vspace{-0.5ex}

In order to cope with such kinds of phenomena, many different logics
have been developed.  They can be roughly categorized according to the
truth values they use. There is classical two-valued logic, which is
not particularly well-suited according to the discussion of Strawson.
In order to overcome such problems three-valued approaches and
approaches with truth-value gaps have been adapted and been developed,
these are in particular the logics developed by Kleene~\cite{Kleene52}
and van Fraassen~\cite{Fraassen66}.  These logics are quite
well-suited to indicate that something is wrong with a certain kind of
statement (like ``The king of France is wise'') and assign to it
either the truth value {\sf undefined} or no truth value at
all. Partiality has been studied in different contexts and there is a
significant body of work. Burmeister~\cite{Burmeister86} gives a very
detailed account of the semantics of partiality by
corresponding partial algebras. They are of particular
interest in order to model the semantics of non-termination in
programming languages. Our approach differs insofar that the logical
conjunction ``false'' and ``undefined'' evaluates to ``false'', but in
a programming language without lazy evaluation would evaluate to
``undefined''.

Logics for reasoning in the presence of partiality do not
enable any kind of hypothetical reasoning (like ``The king of France
is wise.'', ``If somebody is wise he is gracious'', hence ``The king
of France is gracious.'')  In order to cope with corresponding
phenomena Herzberger~\cite{Herzberger73} developed a two-dimensional
approach, in which essentially the truth values consist of pairs of
values, the first value is a traditional truth value, that is, either
{\sf true} or {\sf false} and the second is a kind of presupposition
value, this value is {\sf 1} if all presuppositions are fulfilled and
{\sf 0} otherwise.  This approach was modified and further developed
by Bergmann~\cite{Bergmann:patdl81}, who kept the two-dimensional
approach, but interpreted the values slightly differently so that she
got different truth tables.  While the first component in Bergmann's
approach represents the classical two values, the second expresses
whether the sentence is secure or insecure.  For instance, in an
interpretation ``The king of France is wise'' would be insecure, but
may be true or false.

The presupposition that all mentioned objects exist is adequate unless
the opposite is explicitly said.  However, in everyday language it is
possible to explicitly disable certain presuppositions. This aspect
makes everyday language well-suited, for instance, for a discussion
between a theist and an atheist about the properties of God, {\em
  assuming he exists}.  Classical logic, however, is not well-suited
as basis for formalizing such a dispute, since the fact that all
constants denote something, means if the atheist only used the word
``God'', he would admit the existence of God. In the formal system we
are going to present, however, the status of statements about
constants can be insecure and in particular no existence is assumed,
unless otherwise specified. In Section~\ref{sec:expres} we will look
at a simplified version of Leibniz's argument that God cannot be
almighty, an argument that can be agreed upon by theists and atheists
since it is agnostic with respect to the existence of God. The status
secure/insecure is left open in the argument, that is, the
corresponding formulae will be labelled with truth values
`true-secure/true-insecure'.

While this kind of presuppositions (existence of denotations of constants) is reasonable for most cases,
quantificational presuppositions of everyday languages fundamentally differ from those in
classical logic.  For instance, an everyday sentence like ``All children of John are
sleeping'' presupposes that John really has children. Therefore the representation in FOL
$\all{x}{}{{\sf\!  child\_of}(x,{\sf John}) \implies {\sf sleeps}(x)}$ is not adequate,
since this sentence is true even when John has no children at all.

To overcome this problem Bergmann proposes a restricted quantification
operator of the syntactic form $\all{x}{{\sf child\_of}(x,{\sf
    John})}{{\sf sleeps}(x)}$, as is used in generalized quantifier
theory~\cite{BarCoo:gqnl81,Mostowski57}.  In Section~\ref{sec:presuppl}, the
semantics of a quantified expression will be defined so that the restriction
expression, that is, ${\sf child\_of}(x,{\sf John})$ in the example, is always
non-empty.

\subsection{Undefinedness}
The richer semantics, like those developed by Herzberger and Bergmann,
allow for hypothetical reasoning. However, they do not -- at least
when they are compared to the Kleene approach -- deal appropriately
with utterances that do not make any sense at all. Utterances may
be not understandable, e.g.\ since certain things are not defined, or
because the speech is not serious at all.  Sentences like ``The
present king of France is wise'' or ``God is almighty'' should be
distinguished from sentences like ``Madagascar is east of the
equator'' or even ``Kraba bla shle''.  For the first category,
Strawson's criterion that it could be true (or false) in certain
circumstances holds, while sentences of the second category do not
make any sense.  So we would like to assign to those a fifth truth
value, marking them as {\sf undefined}. 

Again, if we consider the semantics of a quantification in the presence of
undefined values we have to be precise what an expression like
$\forall x\sdot P(x)$ should mean.  In general we do not want this
sentence to tell us anything about undefined terms (since there
is nothing to tell about them, except that they are undefined), so we
would like to interpret this formula as ``for all defined expressions
$x$, $P(x)$ holds.''  In order to be precise about this, here again,
we use only restricted quantifications, that is, expressions of the
kind $\all{x}{{\sf human}(x)} {\sf mortal}(x)$.  If we do not want to
make any restriction on $x$, that is, to quantify over the whole
universe, we write $\all{x}{\Defined(x)} \ldots$, where $\Defined$
stands for ``defined''.  In the same spirit we will use a predicate
$\Secure$ that stands for ``secure''.

For modeling undefinedness, we have to choose among various
approaches.  We advocate here Kleene's approach, in which
undefinedness is contagious for all terms and atomic formulae. That
is, whenever a term contains an undefined expression (e.g., the form
``North of east of the equator'' contains the undefined expression
``east of the equator'') then it is undefined too (and in particular
never covered by a quantification).  Any atomic formula containing
such a term (e.g., ``Madagascar is east of the equator'') is mapped to
a fifth truth value $\undefined$, the truth value ``undefined''.

Whereas atomic formulae containing undefined expressions always
evaluate to ``undefined'', this is not the case for formulae in
general. They may evaluate to true even if they do contain an
undefined expression.  For instance, ``Madagascar is east of the
equator {\em or\/} south of the equator'' is true.  The first part
evaluates to ``undefined'', the second to ``true'' (more precisely to
``true and secure''), hence the disjunction evaluates to true too
(actually to ``true and insecure'' to indicate that something is
strange about the sentence). This is in line with Strawson's argument
that when we speak about the king of France that then we would
constructively react with `France has no king, but a president as head
of state'. Likewise here we might say `east of the equator does not
make sense.' and not only say `yes'.


\subsection{Mechanization} %% was "Automated Theorem Proving"
\label{sec:atp}

% One of the cornerstones in the development of logic in the twentieth century is the
% development of sound and complete calculi, that is, of calculi which exactly correspond to
% a given semantics.  This introduces a fully syntactical notion of proof which represents,
% but does not have to refer to the semantics. By this, a mechanical treatment of the
% semantical consequence relation and in particular of theorem-hood becomes possible at
% least in principle.  The simplest calculi are Hilbert calculi, which derive true
% statements from axiom schemata by sound calculus rules.  While this is quite convenient
% for proving theoretical results about the calculus, it does not correspond to usual
% mathematical practice.  Therefore, Gentzen~\cite{Gentzen:uudlsiii35} developed the calculi
% of Natural Deduction, which also allow to make temporary assumptions.  Variants of this
% calculus are much better suited for practical theorem proving than Hilbert calculi and are
% very popular in advanced {\em interactive\/} proof development environments.

% The most important breakthrough in the development of {\em automated\/} theorem proving
% was Robinson's invention of the resolution principle in 1965~\cite{Robinson:amolbo65}.
% Its three important features compared to the calculi above are: It works with formulae in
% a conjunctive normal form (clause normal form CNF), which simplifies the calculus. It is a
% refutation calculus, which tries to prove a theorem by refuting its negation; this is the
% basis for automating the theorem process as goal-oriented search for the (unsatisfiable)
% empty clause. Finally, but most importantly, it uses a procedure called unification, which
% essentially computes for terms the most general substitution that makes formulae equal;
% this makes it possible to avoid instantiation of universal quantifiers and thus make the
% search space finitely branching.

% In the sequel the idea of unification was transferred to other calculi, furthermore the
% basic calculi were very much refined in order to get very efficient implementations, with
% theorem provers such as Otter~\cite{McCune:otter90}, its successor
% Prover9~\cite{McCune:p9manual06}, Vampire~\cite{RiaVor:Vampire01} and
% Spass~\cite{WeiSchHil:Spass07}.

The mechanization we are going to present in this paper builts up on
the one hand on the mechanization of \textit{sorted logics} by which
the universe of discourse can be structured.  For instance, in a
sorted logic it is possible to say $\all{x}{{\sf human}}{\sf
  mortal}(x)$, instead of $\forall x\sdot {\sf human}(x) \implies {\sf
  mortal}(x)$. This results in a significant improvement of the search
complexity in many cases.  For certain problem classes the
refutation/unification-based framework provides a very powerful tool
for automated theorem proving and in fact it has even been possible to
prove mechanically several mathematical theorems that had been open
previously.  See~\cite{WeGaRo:sfv96} for the developments important to
this paper. On the other hand the mechanization builts up on the
operationalization of \textit{many-valued} first-order logics as
developed by Carnielli~\cite{Carnielli87}, H\"ahnle
\cite{Haehnle:adimvl94}, Baaz and Ferm\"uller~\cite{BaFe92}.

We do not give a full comparison to other work on the mechanization of
partiality since it is only one aspect of the work presented here. We
compare it to the mechanization investigated
in~\cite{GaSoWa06}. Ganzinger et al.\ firstly use a modular
superposition calculus (we a resolution calulus) and secondly build
their work on a different semantic treatment of partiality. In
particular they extend the concept of ``undefinedness'' to that of
``irrelevance'' so that expressions involving undefined terms such
`east of the equator' would be considered as irrelevant rather than
meaningless. In a quantification meaningless terms do
not change the semantic status of the formulae.  In contrast we use
restricted quantification, which firstly allows to exclude undefined
expressions and secondly is well suited to deal with quantification
presuppositions.  We think that the approach in~\cite{GaSoWa06} is
very elegant for the specification of programs but less well suited
for modelling linguistic phenomena.

% In this
% paper we will advocate a many-valued logic that is much better suited to describe
% certain phenomena of presuppositions and undefinedness than classical logic is.
Our operationalization is different from all above-mentioned
operationalizations since these quantifications are unrestricted, that
is, quantified formulae obtain their truth values from those of {\em
  all\/} instances of the scope, whereas we use a restricted form of
quantification.  This is reflected in the use of restriction
constraints in the calculus presented in Section~\ref{sec:presupplres}.

Once the problem of restricted quantification is mastered, it is
relatively straightforward to build a suitable resolution calculus by
employing standard techniques from multi-valued theorem proving: Only
the standard first-order translation from formulae to a clause normal
form has to be adapted so that it respects the many-valued semantics
of the connectives and quantifiers. In order to master the restriction
problem, we employ a sorted version of the multi-valued logic.  For
our purpose the method developed by
Weidenbach~\cite{Weidenbach91,Weidenbach:cafols96} is most
appropriate, since it allows arbitrary sort expressions that may occur
anywhere in formulae.

\section{Presupposition Logic}
\label{sec:presuppl}

In this section, we develop a logic system $\presuppl$ by combining ideas from the
presupposition logic of Bergmann~\cite{Bergmann:patdl81} with methods from our
mechanization of partial functions~\cite{KeKo94} by a sorted variant of three-valued
Kleene logic.

The main problem is to give a proper treatment of restricted
quantification and their presuppositions. But let us first fix the notation
for the logical language. 

\begin{definition}[Signature]
  A {\bf signature} $\Sigma\colon=(\Vars,\Funcs,\Preds)$ consists of the
  following disjoint sets: $\Vars$ is a countably infinite set of {\bf
    variable symbols}, $\Funcs$ is a set of {\bf function symbols}, and
  $\Preds$ is a set of {\bf predicate symbols} that contains special predicates
  $\Defined$ and $\Secure$, called the {\bf definedness} and {\bf security}
  predicates.
  
  The sets $\Funcs$ and $\Preds$ are subdivided into the sets $\Funcs^k$ of
  {\bf function symbols of arity $k$} and $\Preds^k$ of {\bf predicate symbols
    of arity $k$}. Note that individual constants are just nullary functions.
\end{definition}

{}From this signature the terms and formulae are built up in the
usual way with the following two exceptions. Firstly quantification is
generalized to {\em restricted\/} quantification, where the quantifier involves
not only a variable binding, but also a formula $S$ for restricting the domain
of quantification. Secondly, in addition to negation there are three further unary connectives.

\begin{definition}[Terms and Formulae]
  We define the set of {\bf terms} to be the set of variables together
  with {\bf compound terms} $f(t^1,\ldots,t^k)$ for terms
  $t^1,\ldots,t^k$ and $f\in\Funcs^k$. The set of\/ {\bf formulae}
  consists of\/ {\bf atoms} ($P(t^1,\ldots,t^k)$, where
  $P\in\Preds^k$) and of\/ {\bf compound formulae} $A\land B$, $A\lor
  B$, $A\implies B$, $\lnot A$, $\deft A$, $\holds A$, $\tf A$,
  $\forall x_S\sdot A$, and $\exists x_S\sdot A$ ({\bf restricted
    quantification}), where $A$, $B$, and $S$ are formulae.
\end{definition}

  The intended meaning of the restricted quantification $\forall
  x_S\sdot A$ is that $A$ holds for the set of all $x$ for which $S$
  holds, and that furthermore this set is nonempty. The meaning of
  $\deft A$ is that $A$ is secure, the meaning of $\holds A$ that $A$ holds,
  but may be insecure, and the meaning of $\tf A$ that $A$ is defined.

Note that the concept of restricted quantification is a generalization of
quantification in sorted logics, where variables are restricted by so-called sorts, i.e., unary
predicates: For any unary predicate $P\in\Preds$ the restricted quantification
$\forall x_{P(x)}\sdot A$ is equivalent to the sorted quantification $\forall
x_P\sdot A$, which is generally used in the field of sorted logics.

Now we will define the five-valued, two-dimensional semantics for $\presuppl$.
The main feature of Bergmann's logic for presuppositions is a two-dimensional
set of truth values, where the classical two are replaced by four truth-values
which are represented by pairs, where the first component consists of the
values true and false, and the second of the values secure and insecure.
Intuitively, the truth values are augmented by decorating the truth value of a
formula with a ``security value''. Thus the set of truth values contains
$\twts$ and $\twfs$ for secure truth and falsity and $\twti,\twfi$ for the
insecure ones.  Furthermore, we add a value $\undefined$ well-known from the
Kleene-type logics.  In the following we denote the set of {\bf truth values}
by $\Tval=\{\twts,\twfs,\twti,\twfi,\undefined\}$. 
This set serves as the semantical domain for formulae. The meaning of terms
can best be formalized with a more algebraic notion for the universe of
individuals, which we define now.

In defining the semantics we have different options. First we assume
strictness for all function and predicate symbols, that is, we assume
that undefinedness is infectious in the sense that an undefined
subterm of a term makes the whole term undefined.  Whereas strictness
with respect to undefinedness makes sense in almost all cases, the
strictness assumption for security may be too strong for some
applications. For instance, we might want to model the fans of
Sherlock Holmes as secure even if the detective is purely fictional.
Although we assume strictness for security too in the following, this
assumption can be lifted by simply skipping part~\ref{alg:strict} of
the following definition.  In such a case the burden of specifying
where strictness is wanted would be put on the user who has to make
corresponding term declarations (cf.
Definition~\ref{Def:termdec}). Obviously, the change in the semantics
would be reflected in the calculus as well: the second strictness rule
would have to be dropped from the resolution calculus (see
Definition~\ref{def:res}).


\begin{definition}[Strict $\Sigma$-Algebra]
  Let $\Signat$ be a signature, then a pair $(\cA,\cI)$ is called a {\bf strict
    $\Sigma$-algebra} iff\vspace{-1.5ex}
   \begin{enumerate}
     \item the {\bf carrier set} $\cA$ is an arbitrary set that contains a
       special element $\bot$,
     \item\label{alg:int} the {\bf interpretation function $\cI$} obeys the following
       restrictions: 
        \begin{enumerate}
           \item $\cI(\Defined)(\bot)=\twfs$ and  $\cI(\Defined)(a)=\twts$
             in all other cases. Furthermore $\cI(\Secure)(a)\in\{\twfs\!,\twts\}$, but
             $\cI(\Secure)(\bot)=\undefined$. 
             
             We call elements $a\in\cA$ {\bf secure} if $\cI(\Secure)(a)=\twts$
             and {\bf insecure} if $\cI(\Secure)(a)=\twfs$, and {\bf
             undefined} in all other cases. Thus we can partition $\cA$ into subsets $\cA^+$ of
             secure, $\cA^-$ of insecure elements, and $\{\bot\}$. In
             particular, we have
             $\cA=\cA^+\cup\cA^-\cup\{\bot\}$ and $\cA^+\cap\cA^-=\emptyset$.

           \item For all function symbols $f$, the function
             $\cI(f)\colon\cA^k\to\cA$ is strict for $\bot$, that is,
             $\cI(f)(a_1,\ldots,a_k)=\bot$ if $a_i=\bot$ for at least one
             $i$.
           \item If $p\ne\Defined$ is a predicate symbol then the five-valued
             relation $\cI(p)$ is a mapping from $\cA^k$ into $\cB$, the set of
             truth values (see also Example~\ref{ex:literature}).  It is strict for
             $\bot$, that is, $\cI(p)(a_1,\ldots,a_k)=\undefined$ if $a_i=\bot$
             for at least one $i$.
           \item\label{alg:strict} All predicates and functions are strict for $\Secure$,
             i.e., $\cI(f)(a_1,\ldots,a_k)\in\cA^-\cup\{\bot\}$ if $a_{i}\in\cA^{-}$ and 
              $\cI(p)(a_1,\ldots,a_k)\notin\{\twts,\twfs\}$ if
             $a_{i}\in\cA^{-}$ for at least one $i$.\footnote{Note that
             $\cI(p)(a_1,\ldots,a_k)$ may be $\undefined$.  This is in particular
             the case when one of the arguments evaluates to $\bot$.
             If, however, an argument of an expression is insecure the whole
             expression cannot be secure.  For instance, the expression ``mother of
             Sherlock Holmes'' would not be evaluated in $\cA^+$. We do not
             require an ``if and only if'' in the definition, that
             is, for a real person like ``Conan Doyle'', the
             term ``guardian angel of Conan Doyle'' can be
             interpreted in $\cA^-$.}
        \end{enumerate}
  \end{enumerate}
\end{definition}
  
  Note that this treatment differs from the Kleene approach taken
  in~\cite{KeKo94}, where all undefined elements are identified
  with the semantic object $\bot$, since here we distinguish between
  secure, insecure, and undefined terms.  In consequence we are able
  to reason about insecure terms, but not about undefined ones.

The meaning of material from the signature $\Sigma$ is given by the
interpretation function of a strict $\Sigma$-algebra, that of the variables is
locally given by an assignment function $\phi$. It can be combined to
a value function $\cI_{\phi}$, which gives the value of a term or formula
(where the variables are evaluated as defined by $\phi$). 

\begin{definition}[Value of Terms and $\!$Atoms]
  Let $(\cA,\cI)$ be a $\Sigma$-algebra, then we call a total mapping
  $\phi\colon\Vars\to\cA$ a {\bf $\Sigma$-assignment}. We denote the
  $\Sigma$-as\-sign\-ment that coincides with $\phi$ away from $x$ and
  maps $x$ to $a$ with $\phi[a/x]$.
  Now we can inductively define the {\bf value function} $\cI_\phi$ for terms
  and atoms to be
    \begin{enumerate}
  \item $\cI_\phi(f)\deq\cI(f)$ if $f$ is a function symbol or a predicate symbol;
  \item $\cI_\phi(x)\deq\phi(x)$ if $x$ is a variable;
  \item
    $\cI_\phi(f(t^1,\ldots,t^k))\deq\cI(f)(\cI_\phi(t^1),\ldots,\cI_\phi(t^k))\in\cA$,
    for $f\in\Funcs^k$;
  \item
    $\cI_\phi(p(t^1,\ldots,t^k))\deq\cI(p)(\cI_\phi(t^1),\ldots,\cI_\phi(t^k))\in\Tval$
    if $p\in\Preds^k$.
    \end{enumerate}
\end{definition}


\begin{example}\label{ex:literature}
  As a simple example let us look at some atomic formulae with the predicate
  symbol `male'. The set of term declarations
  \[\begin{array}{l}male(arthur\_conan\_doyle)^{\twts},male(sherlock\_holmes)^{\twti},\\
                    male(louisa\_hawkins)^{\twfs},male(mrs\_hudson)^{\twfi},male(baker\_street)^{\undefined}
  \end{array}\]
  tells us about our belief about famous fictional personalities, the
  author who invented them and his wife. In particular these tell us that the
  name ``Arthur Conan Doyle'' stands for a real person
  and that the predicate $\textit{male}$ does not make sense for a street, irrespective
  of whether it exists or not. This shows
  us that five-valued predicates subdivide the universe of individuals into
  five regions: one for every defined truth value (we have given examples for
  each of them) and the undefined region, which only contains the value $\bot$.
\end{example}

In the next definition we define the semantics of the connectives in
such a way that for the classical connectives, $\land, \lor, \non,
\impl$, the behavior in the true-false part is as usual, but
insecureness is infectious. The treatment of undefinedness is
analogous to the case of three-valued Kleene logic. Note, however,
that an expressions involving undefinedness never results in a secure
truth value (e.g., $\twfs\land\twu$ evaluates to $\twfi$ and not to
$\twfs$). $\lor$ and $\impl$ can be defined in $\land$ and $\non$ in
the usual way. The semantics of $\lor$ is of a kind that an expression
in which one of the disjuncts is undefined and the other true and
secure or insecure, the whole formula is true, but insecure (to
indicate that something is odd with the expression). For instance,
``Madagascar is east of the equator or is south of the equator'' will
get the truth value $\twti$, with ``Madagascar is east of the
equator'' as $\twu$ and ``Madagascar is south of the equator'' as
$\twts$. The semantics of the connectives $\holds$, $\deft$, and $\tf$
is defined so that they single out the truth values which are true
(i.e., $\twts$ or $\twti$), secure (i.e., $\twts$ or $\twfs$), and
defined (i.e., not $\undefined$), respectively.

\begin{definition}\label{Def:truth-tables}    
  The value of a formula dominated by a connective is obtained from the
  value(s) of the subformula(e) in a truth-functional way. Therefore it
  suffices to define the truth tables for the connectives:

\[\begin{array}{c@{\;}|@{\;\;}c@{\;\;}c@{\;\;}c@{\;\;}c@{\;\;}c}
\land & \twts & \twfs & \twti & \twfi & \twu\\ \hline\\[-1.9ex]
\twts & \twts & \twfs & \twti & \twfi & \twu\\
\twfs & \twfs & \twfs & \twfi & \twfi & \twfi\\
\twti & \twti & \twfi & \twti & \twfi & \twu\\
\twfi & \twfi & \twfi & \twfi & \twfi & \twfi\\
\twu  & \twu  & \twfi & \twu  & \twfi & \twu
\end{array}\qquad
\begin{array}{c@{\;}|@{\;\;}c@{\;\;}c@{\;\;}c@{\;\;}c@{\;\;}c}
\lor  & \twts & \twfs & \twti & \twfi & \twu\\ \hline\\[-1.9ex]
\twts & \twts & \twts & \twti & \twti & \twti\\
\twfs & \twts & \twfs & \twti & \twfi & \twu\\
\twti & \twti & \twti & \twti & \twti & \twti\\
\twfi & \twti & \twfi & \twti & \twfi & \twu\\
\twu  & \twti & \twu  & \twti & \twu  & \twu
\end{array}\qquad
\begin{array}{c@{\;}|@{\;}c}
\neg  & \\ \hline\\[-1.9ex]
\twts & \twfs   \\
\twfs & \twts  \\
\twti & \twfi \\
\twfi & \twti \\
\twu  & \twu
\end{array}\]\vspace{0ex}

\[\begin{array}{c@{\;}|@{\;\;}c@{\;\;}c@{\;\;}c@{\;\;}c@{\;\;}c}
\implies & \twts & \twfs & \twti & \twfi & \twu\\ \hline\\[-1.9ex]
\twts    & \twts & \twfs & \twti & \twfi & \twu\\
\twfs    & \twts & \twts & \twti & \twti & \twti\\
\twti    & \twti & \twfi & \twti & \twfi & \twu\\
\twfi    & \twti & \twti & \twti & \twti & \twti\\
\twu     & \twti & \twu  & \twti & \twu  & \twu
\end{array}\qquad
\begin{array}{c@{\;}|@{\;}c}
\holds & \\ \hline\\[-1.9ex]
\twts & \twts   \\
\twfs & \twfs  \\
\twti & \twts \\
\twfi & \twfs \\
\twu  & \twfs \\
\end{array}\qquad
\begin{array}{c@{\;}|@{\;}c}
\deft & \\ \hline\\[-1.9ex]
\twts  & \twts \\
\twfs  & \twts \\
\twti  & \twfs \\
\twfi  & \twfs \\
\twu   & \twfs 
\end{array}\qquad
\begin{array}{c@{\;}|@{\;}c}
\tf    & \\ \hline\\[-1.9ex]
\twts  & \twts \\
\twfs  & \twts \\
\twti  & \twts \\
\twfi  & \twts \\
\twu   & \twfs 
\end{array}\]\\[1ex]
\end{definition}

The semantics of the universal quantifier is defined such that an
expression is true only if it evaluates to true under all
instantiations, true and secure only if in addition it evaluates to
secure values under secure instantiations and these are non-empty.  It
evaluates to false if there is one false instance, secure in addition
if all secure instantiations result in secure truth values. In the
other cases, that is, involving undefined instantiations and possibly
true instantiations, the truth value is the fifth truth value,
undefined. For conciseness we present only the universal
quantifier. As usual $\exists x_S\sdot A$ can be considered as an
abbreviation for $\non\forall x_S\sdot\non A$.
\begin{definition}\label{Def:quantifiers}    
For each formula $S$ and each variable $x$ (we call the pair $(x,S)$ a {\bf
  restriction}) let
\begin{eqnarray*}
\cA^+_\phi(S,x)   & = & \{a\in\cA\setdivider\cI_{\phi[a/x]}S=\twts\}\\
\cA^-_\phi(S,x)   & = & \{a\in\cA\setdivider\cI_{\phi[a/x]}S=\twti\}{\rm\ and}\\
\cA^\pm_\phi(S,x) & = & \cA^+_\phi(S,x)\cup\cA^-_\phi(S,x)
\end{eqnarray*}

We call a restriction $(x,S)$ {\bf empty} under $\phi$, if
$\cA^\pm_\phi(S,x)$ is the empty set.  With this we can define the
semantics of the universal quantifier and give a complete case
analysis we distinguish four axioms, \textit{(t)}, \textit{(f)}, \textit{(+)}, and \textit{(--)}.

\begin{tabular}{cccll}
$\cI_\phi(\forall x_S\sdot A)$ & \raisebox{-2ex}[0ex][-4ex]{iff} & $\cI_{\phi[a/x]}$\\
$\in$                        &     & $\in$\\\hline
$\set{\twts,\twti}$ &  &$\set{\twts,\twti}$ & for all
   $a\in\cA^\pm_\phi(S,x)$ & (t)\\

$\set{\twfs,\twfi}$ & & $\set{\twfs,\twfi}$ & for
  some $a\in\cA^\pm_\phi(S,x)$ & (f)\\

$\set{\twts,\twfs}$ &  & $\set{\twts,\twfs}$ & for all
   $a\in\cA^+_\phi(S,x)\not=\emptyset$ & (+)\\
 $\set{\twti,\twfi}$ &  & $\set{\twti,\twfi}$ & for some
   $a\in\cA^+_\phi(S,x)$ or $\cA^+_\phi(S,x)=\emptyset$ & (--)
 \end{tabular}

A universally quantified formula is interpreted as $\twts$, $\twti$,
$\twfs$, and $\twfi$ if and only
if the two corresponding cases hold as
specified in the following table:\\
\begin{tabular}{l|ll}
$\cI_\phi(A)$ & (+) & (--) \\\hline
(t)         & $\twts$ & $\twti$\\
(f)         & $\twfs$ & $\twfi$
\end{tabular}

In all other cases $\cI_\phi(A) = \twu$.
\end{definition}

\begin{example}\label{ex:literature-quantifiers}
  Let us take up example~\ref{ex:literature} to get an intuition on the formal
  definition of quantifiers.
  \begin{description}
  \item[\phantom{--} $\twts$] ``All detectives are mortal'' is true and secure, since the
    assertion holds for all detectives (all existing ones and also Sherlock
    Holmes) and there is at least one existing detective (look him up in your
    local telephone directory).
  \item[\phantom{--} $\twfs$] ``All detectives are rich'' is false and secure, since there
    are poor ones, 
  \item[\phantom{--} $\twti$] ``All detectives have a guardian angel'' is
    true and insecure if we assume guardian angels to be fictional and
    everybody to have a guardian angel. ``All female catholic
    priests are mortal'' is true and insecure, since there are none.
  \item[\phantom{--} $\twfi$] ``All detectives go to heaven'', is
    false and insecure assumed that at least one bad detective goes to
    hell and that the existence of heaven is insecure.
  \item[\phantom{--} $\undefined$] ``All numbers are larger than $1/0$.'' or ``All
    islands are west of the equator'' are undefined since ``$1/0$'' and ``west of the
    equator'' are nonsensical.
  \end{description}
\end{example}

\begin{example}\label{ex-John}
  Let us assume the following information. There is a company {\COMP}
  which wants to fire people, but for social reasons they do not
  fire any person with children. We are worried whether John will be
  fired, but then we hear that his children are sleeping. Implicitly
  we can conclude from this information that John has children,
  because if all his children are sleeping then John must have
  children and hence will not be fired. The semantic definition of
  universal quantification means that a restriction constraint ${\sf
    child}(x,{\sf John})$ will not be empty. This is mirrored in the
  calculus given in the next section, and discussed in more detail in
  Section~\ref{sec:example}.
\end{example}

With the specification of the behaviors of the connectives and quantifiers we have
completed the definition of the semantics of formulae. Let us now come back to the
presuppositions of a formula $A$, which we can now determine from its semantics.
\begin{remark}[Presuppositions and Preimplications]
  Following Bergmann~\cite{Bergmann:patdl81}, we will determine presuppositions in terms
  of semantic anomaly (insecurity) rather than truth; we will use the logical
  concept of {\bf preimplication} as a formalization of the intuitive concept of
  presuppositions. Let us say that formula $A$ {\bf preimplies} formula $B$ (under
  an assignment $\phi$) if and only if $\cI_\phi(A)\in\{\twts,\twfs\}$ entails that
  $\cI_\phi(B)\in\{\twts,\twti\}$, in other words, if the security of $A$ entails
  the truth of $B$. With this we can see that the sentence ``The present king of
  France is wise'' (which is insecure) preimplies the existence of a king of
  France irrespective of the truth value, i.e., it is irrelevant of whether we generally
  think that kings of France are wise or not. Furthermore, we can see that
  preimplication passes the classical linguistic test for presuppositions: The set
  of presuppositions must be invariant under negation. This condition is trivially
  met by preimplication, since negation does not change the security value of a
  formula.
\end{remark}

Bergmann uses a different negation connective $\sim$ that is identical to ours, except that
$\cI(\sim)(\twfi)=\twfi$.  This is motivated by a slightly different understanding
of presuppositions, cf.~\cite{Bergmann:patdl81}. Our choice of negation, which is more
pleasing to a logician, and indeed more suited to hypothetical reasoning is not a
restriction in principle, since we could have both connectives in our system.  Moreover, the
two connectives are interdefinable; therefore, we only need to treat one of them
here: It is an easy exercise to see that $\neg A=\sim A\lor((\sim\holds
A)\land \tf A)$ and $\sim A=\neg A\land(\deft A\lor \neg\tf A)$.

In the next definition we introduce labeled formulae. Although
redundant for expressiveness we shall need them in the generation of the
clause normal form from labeled formulae as well as in the resolution
calculus in form of labeled clauses.

\begin{definition}[$\Sigma$-Model]\label{Def:model}
  Let $A$ be a formula, then we call $A^\alpha$ (the formula $A$ indexed with
  the intended truth value $\alpha\in\Tval$), a {\bf labeled formula}. 
  
  Let $A$ be a formula, then we call a $\Sigma$-algebra $\cM\deq(\cA,\cI)$ a
  {\bf $\Sigma$-model for $A$} (written $\cM\models A$) iff
  $\cI_\phi(A)=\twts$ for all $\Sigma$-assignments $\phi$. Analogously,
  $\cM\models A^{\alpha}$ if and only if $\cI_{\phi}(A)=\alpha$ for all assignments
  $\phi$. With this notion we can define the notions of {\bf validity}, {\bf
    (un)-satisfiability}, and {\bf entailment} in the usual way.
\end{definition}


We are normally only interested in classes of models that satisfy some common
background knowledge about definedness of concepts (corresponding to
questions whether Pegasus exists, whether he is a horse, or about the nature
and existence of his left front hoof). For this we restrict our class of
models and variable assignments to those that satisfy a set of so-called term
declarations that can be supplied by the user to specify just this background
knowledge.

\begin{definition}[Term Declarations]\label{Def:termdec}
  In the following, we will assume a set $\TermDecs$ of labeled
  formulae that we will call {\bf term declarations} and restrict our
  attention to $\TermDecs$-models, i.e., models that satisfy all labeled
  formulae in $\TermDecs$.
\end{definition}

{}From a purely theoretical point of view, term declarations do not yield more
expressiveness, since they can be axiomatized (any intended truth value can be
characterized by combinations of the connectives $\non$, $\deft$,
$\holds$, and $\tf$).  However, from a practical point of view, the term
declarations provide a convenient means of specifying the belief about
existence and sortality in the world. Furthermore, the term declarations can
be used for optimizations of the calculus by sorted unification as
in~\cite{KeKo:atcfpf96}.

  
\begin{remark}
  Now we can further study the relationship of restricted quantification
  to sorted logics. Those usually define the {\bf carrier}
  $\cA_P\subseteq\cA$ for any sort (unary predicate $P\in\Preds$) as
  $\cA_P\deq\{a\in\cA\setdivider\cI(P)(a)$ $=\true\}$ and use that to
  define sorted quantification as $\cI_\phi(\forall x_P\sdot A)$ to be
  true if and only if $\cI_{\phi[a/x]}(A)$ is true for all $a\in\cA_P$. Note
  that sorted logics usually assume that the $\cA_S$ are
  non-empty\footnote{The logics of Cohn and
    Weidenbach~\cite{Cohn87,Weidenbach91} did away with this
    restriction that has always been considered as a technical anomaly. Originally
    non-emptiness was used only to alleviate the need for special treatments in the
    transformation to clause normal form and for instantiations in the
    resolution calculus. A unifier that contains variables of sorts
    that are empty does not -- in general --  lead to a correct refutation.} and
  therefore lead to the same presuppositions as $\presuppl$ on the
  sorted fragment.

  We exploit this similarity in this paper by generalizing sort techniques for
  the mechanization of $\presuppl$. 
\end{remark}

\begin{remark}
  Bergmann's generalization of the restrictions to formulae adds practical
  expressiveness to the logical system. For instance, the domain of
  quantification in the formula $A=\forall x_{{\sf son}(x,y)}.B$  will be sensitive
  to the assignment to the variable $y$. In particular, $\cI_{[a/y]}(A)$ will
  make an assertion about the sons of $a$, while $\cI_{[b/y]}(A)$ one about the
  sons of $b$. The negative side of this generalization is that the property
  of well-sortedness (and thus presupposition failure) becomes undecidable;
  this further illustrates the need for inference calculi for the entailment
  relation of presupposition logics.
\end{remark}

\begin{remark}
  The ``tertium non datur'' principle of FOL is no longer valid, since formulae
  can indeed be evaluated to five different truth values.  We do however have a
  ``sextum non datur'' principle, that is, formulae cannot get a truth value
  different from the five in $\cB$.  This allows us to derive the validity of a
  formula (i.e., that it is true and secure in all models) by refuting that it is
  $\twti$, $\twfs$, $\twfi$, or $\undefined$.  (In the same line, if we are not
  interested in security we can show the truth of a formula by refuting that it
  is $\twfs$, $\twfi$, or $\undefined$.)  We will use this observation in our
  resolution calculus, introduced in the next section.
\end{remark}


%\input{middle}
\section{Resolution Calculus ($\presuppres$)}\label{sec:presupplres} 

In this section we present a resolution calculus $\presuppres$ that is a generalization of
the resolution calculi for partial functions~\cite{KeKo94}, and that for Bergmann's
logic~\cite{KeKo:arcfp96}, which are in turn joint generalizations of Weidenbach's logics
with dynamic sorts~\cite{Weidenbach91} with ideas from~\cite{BaFe92,Haehnle:adimvl94}. At
first, we present the transformation to clause normal form as in a traditional resolution
calculus. To this end we give for each possible combination of a label and a top level
connective/quantifier one transformation rule. This translation is directly derived from
the semantics. For instance, the first rule is a combination of the label $\twts$ and the
connective $\land$. A set of labeled formulae $\cF$ containing a labeled formula
$C,(A\land B)^\twts$ is satisfiable if and only if the labeled formula set generated from
$\cF$ by replacing the labeled formula $C,(A\land B)^\twts$ by the two labeled formulae
$C,A^\twts$ and $C,B^\twts$ is satisfiable.

\begin{definition}[Clause Normal Form]\label{Def:cnf}
  We will call a labeled atom $L^\alpha$ a {\bf literal} and a set of
  literals $\{L_1^{\alpha_1},\ldots,L_n^{\alpha_n}\}$ a {\bf clause}. In
  order to enhance legibility, we employ the ``,'' as the operator for the
  disjoint union of sets so that $C,L^\alpha$ means $C\cup\{L^\alpha\}$ and
  $L^\alpha$ is not a member of $C$. Furthermore we adopt
  H\"ahnle's~\cite{Haehnle:adimvl94} notion of multi-labels in the form
  $C,A^{\alpha\beta}$ to mean $C,A^\alpha,A^\beta$.
  The clause normal form of a labeled formula set is generated by the application 
  of the following rules.\vspace{1ex}

\noindent\underline{$\land$-rules}
\[\rulespacing%\small
\begin{array}{c}
\ianc{C,(A\land B)^\twts}
     {C,A^\twts\quad C,B^\twts}
     {}  \hspace{2em}
\ianc{C,(A\land B)^\twfs}
     {C,\holds A^\twfs,\holds B^\twfs\quad C,\deft A^\twts\quad C,\deft B^\twts}
     {} 
\end{array}\]
\[\rulespacing
\begin{array}{c}
\ianc{C,(A\land B)^\twti}
     {C,\holds A^\twts\quad C,\holds B^\twts\quad C,\deft A^\twfs,\deft B^\twfs}
     {} \hspace{2em}
\ianc{C,(A\land B)^\twu}
     {C,A^{\twu\twts\twti}\quad C,B^{\twu\twts\twti}\quad C,A^\undefined,B^\undefined}
     {}  \\
\ianc{C,(A\land B)^\twfi}
     {C,\holds A^\twfs,\holds B^\twfs\quad C,\deft A^\twfs,\deft B^\twfs
      \quad C,\tf A^\twts,\tf B^\twts}
     {} 
\end{array}\]
\noindent\underline{unary connective-rules}
\[\rulespacing
\begin{array}{c}
\ianc{C,(\neg A)^\twts}
     {C,A^\twfs}
     {}  \hspace{1.6em}
\ianc{C,(\neg A)^\twfs}
     {C,A^\twts}
     {}  \hspace{1.6em}
\ianc{C,(\neg A)^\twti}
     {C,A^\twfi}
     {}  \hspace{1.6em}
\ianc{C,(\neg A)^\twfi}
     {C,A^{\twti}}
     {}  \hspace{1.6em}
\ianc{C,(\neg A)^\twu}
     {C,A^{\twu}}
     {}  \end{array}\]
\[\rulespacing
\begin{array}{c}
\ianc{C,(\deft A)^\twts}
     {C,A^{\twts\twfs}}
     {}   \hspace{2em}
\ianc{C,(\deft A)^\twfs}
     {C,A^{\twti\twfi\twu}}
     {}   \hspace{2em}
\ianc{C,(\deft A)^\twti}
     {C}
     {}   \hspace{2em}
\ianc{C,(\deft A)^\twfi}
     {C}
     {}   \hspace{2em}
\ianc{C,(\deft A)^\twu}
     {C}
     {}   \end{array}\]
\[\rulespacing
\begin{array}{c}
\ianc{C,(\holds A)^\twts}
     {C,A^{\twts\twti}}
     {} \hspace{1em}
\ianc{C,(\holds A)^\twfs}
     {C,A^{\twfs\twfi\twu}}
     {} \hspace{1em}
\ianc{C,(\holds A)^\twti}
     {C}
     {}   \hspace{1em}
\ianc{C,(\holds A)^\twfi}
     {C}
     {}  \hspace{1em}
\ianc{C,(\holds A)^\twu}
     {C}
     {} 
\end{array}\]
\[\rulespacing
\begin{array}{c}
\ianc{C,(\tf A)^\twts}
     {C,A^{\twts\twti\twfs\twfi}}
     {} \hspace{1em}
\ianc{C,(\tf A)^\twfs}
     {C,A^\twu}
     {} \hspace{1em}
\ianc{C,(\tf A)^\twti}
     {C}
     {}   \hspace{1em}
\ianc{C,(\tf A)^\twfi}
     {C}
     {}   \hspace{1em}
\ianc{C,(\tf A)^\twu}
     {C}
     {} 
\end{array}\]

\noindent\underline{$\forall$-rules}
In order to model these rules we decompose formulae in their
true/false and their secure/insecure components (unless they are
undefined) and state rules as (note that in our clauseset `,' stands
implicitly for `or' and `;' for `and'.)

\def\Q{\textbf{Q}}
\[\rulespacing
 \begin{array}{c}
 \ianc{C,(\forall x_S\sdot A[x_S])^{\alpha^\beta}}
      {\Q^\alpha(C,A,S) \quad \Q^\beta(C,A,S)}
      {}
\end{array}\]

for $\alpha\in\set{\true,\false}$ and $\beta\in\set{\texttt{+},\texttt{-}}$ where the normalization function \Q\ is defined as:

\(
\begin{array}{ll}
\Q^+(C,A,S) = & C, A[x_S^+]^{\twts\twfs};\\
             &  C, [g(z^1,\ldots,z^m)/x]S^\twts\\
\Q^-(C,A,S) = & C, [f(y^1,\ldots,y^n)/x]S^\twts, [z^+/x]S^\twfs; \\
              & C, [f(y^1,\ldots,y^n)/x]A^{\twti\twfi}, [z^+/x]S^\twfs\\
\Q^\true(C,A,S) = & C, A[x_S^\pm]^{\twts\twti}\\
\Q^\false(C,A,S) = & C, [f(y^1,\ldots,y^n)/x]A^{\twfs\twfi}; \\
                  & C, [f(y^1,\ldots,y^n)/x]S^{\twts\twti}
\end{array}
\)

\noindent where $\{x_S,y^1,\ldots,y^n\}=\free(A)$ (the set of free
variables in $A$), $\{x_S,z^1,\ldots,z^m\}=\free(S)$ and $f$ is a new
function symbol of arity $n$, and $g$ one of arity $m$.

In order to define the rule for undefinedness we define four rules for
the complements of the above four rules: 

\(
\begin{array}{ll}
\Q^{\non +}(C,A,S) = & C, [f(y^1,\ldots,y^n)/x]A^{\twti\twfi\twu}, [z^+/x]S^\twfs;\\
                    &C, [f(y^1,\ldots,y^n)/x]S^{\twts}, [z^+/x]S^\twfs \\

\Q^{\non -}(C,A,S) = & C, A[x_S^+]^{\twts\twfs\twu};\\
                   & C, [g(z^1,\ldots,z^m)/x]S^\twts\\

\Q^{\non\true}(C,A,S) = & C, [f(y^1,\ldots,y^n)/x]A^{\twfs,\twfi,\twu};\\
                      &  C, [f(y^1,\ldots,y^n)/x]S^{\twts,\twti}\\
\Q^{\non\false}(C,A,S) = & C, A[x_S^\pm]^{\twts,\twti,\twu}\\

\end{array}
\)

% \[\rulespacing
% \begin{array}{c}
% \ianc{C,(\forall x_S\sdot A[x_S])^\twts}
%      {C,A[x_S^+]^\twts\quad C,[f(y^1,\ldots,y^n)/x]S^\twts\quad
%        C,A[x_S^-]^\twti\quad C,\Secure(f(y^1,\ldots,y^n))^\twts}
%      {}\\
% \ianc{C,(\forall x_S\sdot A[x_S])^\twfs}
%      {{C,A[f(y^1,\ldots,y^n)]^\twfs\quad 
%       C,([f(y^1,\ldots,y^n)/x]S)^{\twts}}\atop {C,A[x_S^+]^{\twts\twfs}
%       \quad C,\Secure(f(y^1,\ldots,y^n))^\twts}}
%      {} \\[2ex]
% \ianc{C,(\forall x_S\sdot A[x_S])^\twti}
%      {{C,A[x_S^\pm]^{\twts\twti}\quad 
%       C,[f(y^1,\ldots,y^n)/x]S^{\twts},[z^+/x]S^\twfs \atop
%       {C,\Secure(f(y^1,\ldots,y^n))^\twts,[z^+/x]S^\twfs}\quad C,A[f(y^1,\ldots,y^n)]^\twti,[z^+/x]S^\twfs}}
%      {}\\[2ex]
% \ianc{C,(\forall x_S\sdot A[x_S])^\twfi}
%      {C,A[f(y^1,\ldots,y^n)]^\twfi\quad
%       C,([f(y^1,\ldots,y^n)/x]S)^{\twts\twti}}
%      {}  \\
% \ianc{C,(\forall x_S\sdot A[x_S])^\twu}
%      {C,A[f(y^1,\ldots,y^n)]^\twu\quad
%        C,A[x_S^\pm]^{\twu\twts\twti}\quad 
%        C,([f(y^1,\ldots,y^n)/x]S)^{\twts\twti}}
%      {}  \\
% \end{array}\]

The rule for undefinedness follows from the semantics of undefinedness
of a universal quantification that it can be neither true nor false
and neither secure nor insecure, that is,

\[
 \begin{array}{c}
 C,(\forall x_S\sdot A[x_S])^\twu\\[1ex]\hline
     \rule{0ex}{3ex}\Q^{\non\true}(C,A,S)\qquad\Q^{\non\false}(C,A,S)\qquad
       \Q^{\non+}(C,A,S)\qquad\Q^{\non-}(C,A,S)
  \end{array}\]
\end{definition}

Note that the transformations for the universal quantifier introduce new free
variables $x^*_S$, consisting of a variable $x$, a mode specifier $\in\{+,-,\pm\}$
(see Definition~\ref{Def:restconst}) and a restriction $S$.

Note furthermore that this set of transformations is confluent,
therefore any total reduction of a set $\Phi$ of labeled sentences
results in a unique set of clauses. We will denote this set with
$\Cnf(\Phi)$.

The quantifier rules read that the true/false part and the
secure/insecure parts of the rules are separated. For instance in
order to have the truth value $\twts$, universally quantified formulae
must be secure for all secure instantiations and the restriction must
be non-empty on the one hand, and must be true (secure or insecure)
for all instantiations in the restriction. Likewise for $\twfi$ there
must be an instance in the restriction (secure or not) which make the
formula false (secure or not) and there must be a secure instance in
the restriction which makes the formula insecure.

% The second quantifier rule, ($\forall$, $\twfs$), reads, for instance,
% a universal quantification is false and secure for a restriction $S$ iff
% there is a secure witness in $S$ that demonstrates the falsity and all
% secure instances are secure. 
% The third quantifier rule reads, a universal quantifier is true and
% insecure iff any defined instantiation makes the body of the
% quantification true and furthermore the restriction $S$ does either
% not contain any secure elements or it does contain a secure element
% which makes the body of the quantification insecure.
% The last rule reads, a universal
% quantification gets the value undefined iff there is witness for
% undefinedness in the restriction of the quantification and all
% instantiations result in either undefined or true (i.e., $\undefined$,
% $\twts$, or $\twti$).

The clause normal form transformations as presented above are not complete, that
is, they do not transform every given labeled formula into clause form, since the
rules for quantified formulae insist that the bound variable occurs in the scope.
In fact the handling of degenerate quantifications poses some problems in the
presence of possibly empty restrictions, as quantification over empty sets should
semantically be true. In this situation we have the possibilities, either to forbid
degenerate quantifications, or empty restrictions, or treat degenerate
quantifications in the clause normal form transformations. For this paper we chose
the first, since degenerate quantifications do not make much sense and normally do not
appear in everyday language. This choice has the additional advantage that it can be very easily checked, each quantification variable must occur in the body of the quantification.   See~\cite{KeKo:atcfpf96} for the other possibilities.
Thus we will assume that in all formulae in this paper the bound variables of
quantifications occur in the scopes. Note that this assumption is not about the
restrictions. In particular, $\forall x_{S}.A$ is allowed, even if
$x\notin\free(S)$. Intuitively, this means that $x$ is unrestricted in a context
that makes $S$ true and the whole formula carries the presupposition that $S$ is
indeed true there.

\begin{remark}\label{Rem:combine}
  As in many regular multi-valued logics like Kleene logic it is possible to define
  combined transformation rules for truth value sets.  For instance, the
  transformation for $\forall$ with the set $\{\twts,\twti\}$ has the following form:
\[\rulespacing
\ianc{C,(\forall x_S\sdot A[x_S])^{\twts\twti}}
     {C,A[x_S^\pm]^{\twts\twti}}
     {}
\]
Using such transformations can result in a much shorter clause normal
form\footnote{A computation using only the rules from Definition~\ref{Def:cnf}
  would have yielded 4 clauses in the succedent of the rule shown above.}, which
in turn is a much better starting point for any refutation. Moreover, in some
regular multi-valued logics, cf.~\cite{Haehnle:adimvl94}, a careful analysis using sets
of signs can result in a calculus that can be realized as a restriction strategy
to existing theorem provers such as Spass~\cite{WeiSchHil:Spass07}. We have
executed this for three-valued Kleene-logic in~\cite{KeKo:mpwri97}, obtaining a
simple path to a highly efficient implementation. Since $\presuppl$ is very
similar to Kleene logic, it seems plausible that a similar reduction is possible
for $\presuppres$ as well, but we leave this to further research.
\end{remark}

\begin{definition}
  Let $\cM=(\cA,\cI)$ be a $\Sigma$-model, then we will call a variable assignment
  $\phi$ {\bf well-sorted} if and only if $\phi(x^*_S)\in\cA^*_\phi(S,x)$ for all
  $x\in\Dom(\phi)$ and say that $\phi$ {\bf satisfies} a clause $C$ in $\cM$,
  if and only if it satisfies one of its literals $L^\alpha\in C$, that is,
  $\cI_\phi(L^\alpha)=\alpha$. $\phi$ satisfies a set of clauses in
  $\cM$ if and only if it satisfies
  each clause.
\end{definition}

\begin{theorem}\label{Thm:cnf-theorem}
  Let $\Phi$ be a set of labeled sentences, then the clause normal
  form $\Cnf(\Phi)$ is satisfiable if and only if $\Phi$ is (under the
  assumption that all quantifications are not degenerated).
\end{theorem}
\begin{sketch}
  The assertion critically depends on the fact that the notion of
  satisfiability employed here takes the restrictions into account: A clause
  is valid in a $\Signat$-model $\cM$ if and only if for one literal $L^\alpha$ we have
  $\cI_\phi(L)=\alpha$ for all well-sorted assignments $\phi$ into $\cM$. With
  this notion, the assertion can be reduced to the standard argumentation
  about Skolemization and a tedious calculation with the truth tables
  from Definition~\ref{Def:truth-tables}.
\end{sketch}

Now we proceed to give a simple resolution calculus, which utilizes standard
(unsorted) unification.  In~\cite{KeKo:atcfpf96}, we have further improved a similar
calculus by using a sorted unification algorithm, which delegates parts of the
search into the unification algorithm.  For unsorted substitutions a naive
resolution rule is unsound.  Therefore we have to add a residual (the
restriction constraint) that ensures the soundness (with respect to the
restrictions on the variables) of the unifier.

\begin{definition}[Restriction Constraints]\label{Def:restconst}$\phantom{a}$\\
  The {\bf restriction constraint} of a substitution $[t/x^*_S]$, depends on the
  mode $*$ and is defined as follows:
  \[\begin{array}{l}
  \restconst([t/x^+_S])\deq\{([t/x]S)^{\twti\twfs\twfi\undefined},\Secure(t)^{\twfs}\},\\
  \restconst([t/x^-_S])\deq\{([t/x]S)^{\twts\twfs\twfi\undefined},\Secure(t)^{\twts}\},
  \\
  \restconst([t/x^\pm_S])\deq\{([t/x]S)^{\twfs\twfi\undefined}\}, {\rm\ and}\\

  \restconst([t/x^+])\deq\{\Secure(t)^{\twfs}\}.\end{array}\] 
  For a substitution $\sigma=\sigma',[t/x^*_S]$ we have
  $\restconst(\sigma)\deq\restconst(\sigma')\cup\restconst{[t/x^*_S]}$.
\end{definition}

These clauses are residuated in the $\presuppres$ rules and have to be refuted in
order to guarantee that the restriction $\cA^*_\phi(S,t)$ holds
(cf.~Definition~\ref{Def:truth-tables}) for every instance $t$ instantiated for a
free variable $x^*_S$.

\begin{definition}[Resolution Inference Rules
  ($\presuppres$)]\label{def:res}$\phantom{a}$
The inference rules of the resolution calculus are
\begin{center}
\(\rulespacing\begin{array}{c}
\ianc{L^\alpha,C\qquad M^\beta,D}
     {\sigma(C),\sigma(D),\restconst(\sigma)}
     {Res}\hspace{3em}
\ianc{L^\alpha,M^\alpha,C}
     {\sigma(L^\alpha),\sigma(C),\restconst(\sigma)}
     {Fac} \\
\ianc{\Defined(t)^\twfs,C\qquad L^{\gamma},D}
     {\rho(C),\rho(D),\restconst(\rho)}
     {\mbox{$\it Strict^\Defined$}} \hspace{3em}
\ianc{\Secure(t)^\twfs,C\qquad L^{\delta},D}
     {\rho(C),\rho(D),\restconst(\rho)}
     {\mbox{$\it Strict^\Secure$}}
\end{array}\)
\end{center}
\noindent where $\alpha\neq\beta$ and $\gamma\ne\undefined$, and
$\delta\in\{\twts,\twfs\}$. For ${\it Res}$ and ${\it Fac}$, the substitution
$\sigma$ is the most general unifier of $L$ and $M$; and for $\it
Strict^\Defined$ and $\it Strict^\Secure$ there
exists a subterm $s$ of $L$, such that $\rho$ is a most general unifier of $t$
and $s$.
\end{definition}

Here we have assumed $\alpha,\beta,\gamma,\delta$ to be single truth values,
naturally the rules can be easily extended to sets of truth values. For instance,
the resolution of $L^\alpha,C$ and $M^\beta,D$ would lead to the clause
$\sigma(L)^{\alpha\cap\beta},\sigma(C),\sigma(D),\restconst(\sigma)$.

\begin{remark}
Note that clauses containing $A^{\twts\twfs\twti\twfi\undefined}$ are
tautological and can therefore be deleted in the generation of the
clause normal form as well as in the deduction process. The calculus
can be extended by the usual subsumption rule, allowing the deletion
clauses that are subsumed (super-sets).
\end{remark}

\begin{definition}
  Let $A$ be a sentence and $\Phi$ be the clause normal form of the set
  $\{\{A^{\twfs\twti\twfi\undefined}\}\}$ then we say that $A$ can be
  {\bf proved in} $\presuppres$ ($\vdash A$) if and only if there is a derivation of
  the empty clause $\Box$ from $\Phi$ with the inference rules
  above.
\end{definition}

\begin{theorem}[Soundness]\label{Thm:sound}
$\presuppres$ is sound.
\end{theorem}
\begin{sketch}
  The soundness of the resolution and factoring rules is established in the usual
  way taking into account that the restriction constraints make the substitutions
  ``well-sorted'' and thus compatible with the semantics: The restriction
  constraints add literals $([t/x^*]S)^{\alpha}$ and $\Secure(t)^\beta$ (with
  truth value sets $\alpha$ and $\beta=\beta(*)$) per component of the
  substitution, which only can be refuted if indeed $([t/x]S)^\gamma$ with
  $\alpha\cap\gamma=\emptyset$ and $t$ has the right security characteristic $*$.
  
  The soundness of the rules {\mbox{$\it Strict^\Secure$}} and {\mbox{$\it
      Strict^\Defined$}} hinges on the fact that an atom must be insecure
  (undefined) if and only if  it has an insecure (undefined) subterm, since we have assumed
  strictness of functions and predicates.
\end{sketch}

\begin{definition}
 Let $C\deq\{L_1^{\alpha_1},\ldots,L_n^{\alpha_n}\}$ be a clause, then
  the {\bf conditional instantiation $\scapply\sigma{C}$ of $\sigma$
  to $C$} is defined by
\[\scapply\sigma{C}\deq\{\sigma(L_1^{\alpha_1}),\ldots,\sigma(L_n^{\alpha_n})\}\cup\restconst({\restrict\sigma{\free(C)}})\]
\end{definition}

The following result from~\cite{Weidenbach91} is independent of the number of
truth values.

\begin{lemma}
  Conditional instantiation is sound: for any clause $C$, substitution
    $\sigma$ and $\Sigma$-model $\cM$ we have that
    $\cM\models\scapply\sigma{C}$, whenever $\cM\models C$.
\end{lemma}

\begin{definition}[Herbrand Model]
  Let $A$ be a sentence and $\Cnf(A)$ be the clause normal form of $A$, then we
  define the {\bf Herbrand set of clauses $\Cnfh(A)$ for $A$} as
  \(\{\scapply\sigma{C}\setdivider C\in\Cnf(A), \sigma\ \mbox{\rm ground},
  \Dom(\sigma)=\free(C)\}\)
  
  Let $\Phi$ be a set of clauses, then the {\bf Herbrand base
    $\Herbrandbase(\Phi)$ of $\Phi$} is defined to be the set of all ground
  atoms containing only function symbols that appear in the clauses of $\Phi$. If
  there is no individual constant in $\Phi$, we add a new constant $c$. A {\bf
    valuation} $\nu$ is a function $\Herbrandbase(\Phi)\to\Tval$.  The {\bf
    $\Sigma$-Herbrand model $\cH$ for $\Phi$ and $\nu$} is the set
  $\cH\deq\{L^\alpha\setdivider\alpha=\nu(L), L\in\Herbrandbase(\Phi)\}$. Note
  that $\cH$ cannot contain {\bf complementary} literals (i.e., two literals
  $L^\alpha$ and $L^\beta$, where $\alpha\ne\beta$) since $\nu$ is a function.
  
  We say that a $\Sigma$-Herbrand model $\cH$ {\bf satisfies a clause set $\Phi$}
  if and only if for all ground substitutions $\sigma$ and clauses $C\in\Phi$ we have
  $\scapply{\sigma}{C}\cap \cH\ne\emptyset$. A clause set is called {\bf
    $\Sigma$-Herbrand-unsatisfiable} if and only if there is no $\Sigma$-Herbrand-model for
  $\Phi$.
\end{definition}
The following theorem which links Herbrand satisfiability to satisfiability in
$\Sigma$-models is a key ingredient to the completeness proof in $\presuppl$,
therefore we will give the proof in detail.

\def\leqc{[\kern-0.18em[} 
\def\reqc{]\kern-0.18em]} 
\def\eqc#1{{\leqc{#1}\reqc}}


\begin{theorem}[Herbrand\hspace{-0.16pt} Theorem]\label{Thm:Herbrand-theorem}
Let $A$ be a formula, then the clause normal form $\Cnf(A)$ has 
a $\Sigma$-model if and only if $\Cnfh(A)$ has a $\Sigma$-Herbrand-model.
\end{theorem}
\begin{proof}
  Let $\cM=(\cA,\cI)$ be a $\Sigma$-model for $\Phi\deq\Cnf(A)$ and
  $\Psi\deq\Cnfh(A)$. We will show that the set $\cH\deq\{L^\alpha\setdivider
  L\in\Herbrandbase(\Phi),\alpha=\cI_\emptyset(L)\}$ is a $\Sigma$-Herbrand model
  for $\Psi$. For this we assume the opposite, that is, there is a clause
  $C\in\Psi$, such that $\cH\cap C=\emptyset$. This is sufficient, since $C$ is
  ground and therefore $\scapply\sigma{C}=C$. Since $C\in\Psi$ there is a ground
  substitution $\sigma=[t^i/x^{*_i}_{S_i}]$ and a clause $D\in\Phi$, such that
  $C=\scapply\sigma{D}=\sigma(D)\cup\restconst(\sigma)$ by definition.
  
  Without loss of generality we can assume that the assignment
  $\psi\!\deq[\cI_\emptyset(t^i)/x^i]$ is well-sorted, i.e.
  $\psi(x^*_S)\in\cA^*_\phi(S,x)=\{a\in\cA\setdivider\cI_{\psi,[a/x]}(S)=*\}$ or
  equivalently $\cI_\psi(S)=*$, since otherwise $\cI_\psi(S)\in\Tval\setminus\{*\}$,
  and therefore $([t^i/x_i]S^i)^\gamma\in\cH$ for some $\gamma\in\Tval\setminus\{*\}$,
  which contradicts the assumption that  $\cH\cap C=\emptyset$, since
  $([t^i/x_i]S^i)^\gamma\in\restconst(\sigma)$ by construction.
  
  Note that since $\cM$ is a model of $\Phi$, we have that $\cM\models D$ and
  therefore there is a literal $L^\alpha\in D$, such that
  $\alpha=\cI_\psi(L)=\cI_\phi(\sigma(L))$, hence $\sigma(L)\in\cH$, which
  contradicts the assumption.

  \vspace{2ex}
  
  For the converse direction let $\cH$ be a $\Sigma$-Herbrand model for $\Psi$ and
  $\cH\cB$ the Herbrand base for $\cH$ together with $\bot$. In order to construct
  the carrier $\cA$ of the intended model, we have to identify all elements in
  $\cH\cB$ that are undefined, $(\Defined(t))^\twfs\in\cH$, and identify them
  with $\bot$. We continue the traditional construction of the model by defining the meanings
  $\cI(f^n)$ and $\cI(P^n)$ of functions and predicates, such that
  \begin{eqnarray*}
      \cI(f^n)(t^1,\ldots,t^n)\deq f^n(t^1,\ldots,t^n) &\qquad {\rm iff}\qquad &
         f^n(t^1,\ldots,t^n)\in\cA \\
      \cI(P^n)(t^1,\ldots,t^n)\deq \alpha &\qquad {\rm iff}\qquad &
         (P^n(t^1,\ldots,t^n))^\alpha\in\cH
  \end{eqnarray*}
  
  However, this definition does not make $\cI(f)$ strict for $\Defined$, since
  $\cI(f)(\bot)=f(\bot)\ne\bot$. To repair this defect we take the carrier $\cA$
  to be the quotient of $\cT_\bot$ with respect to the equality theory $=_\bot$
  induced by the set
  \[E_\bot=\{t=_\bot\bot\setdivider(\Defined(t))^\twfs\in\cH\}\cup
           \{f^k(x_1,\ldots,\bot,\ldots,x_k)=_\bot\bot\setdivider f^k\in\Sigma^k\}\]
  of equations. Thus $\cA$ is the set of equivalence classes 
  $\eqc{t}_\bot=\{s\setdivider E_\bot\models s=_\bot t\}$. The
  function $f_\bot\colon(\eqc{t_1}_\bot,\ldots,\eqc{t_1}_\bot)\mapsto\eqc{f(t_1,\ldots,t_n)}_\bot$ is a
  well-defined function, since $=_\bot$ is a congruence relation. We define
  $\cI(f):=f_\bot$ and note that the special construction of $E_\bot$ 
  entails the strictness of $f_\bot$ for $\Defined$. Strictness for $\Secure$ is
  not a problem in this construction, since terms are not identified
  to a single element as in the case of undefinedness.
    
  For $P\in\Preds^n$ let
  $P_\cH(\eqc{t^1}_\bot,\ldots,\eqc{t^n}_\bot)=\alpha$ iff
  $P(t^1,\ldots,t^n)^\alpha\in\cH$. Clearly $P_\cH$ is well-defined, since the
  definition only depends on $=_\bot$-equivalence classes, furthermore it is
  strict for $\Defined$ and $\Secure$ by construction.

  Note that any assignment into $\cM\deq(\cA,\cI)$ has the form $[\eqc{t_i}_\bot/x_i]$, where the
  $t_i$ are ground terms, so the set of substitutions and the set of assignments are
  isomorphic (we will neglect to make this isomorphism explicit in the
  following). Moreover, a simple structural induction shows that
  $\cI_\phi(t)=\eqc{\phi(t)}_\bot$. In fact, the argument can be extended to show that
  the isomorphism respects well-sortedness. 
  
  Finally we convince ourselves that $\cM\models\Phi$. Let $C\in\Phi$ and
  $\phi\!\deq[t^i/x^{*_i}_{S_i}]$ be an arbitrary well-sorted $\Sigma$-assignment.
  So we have $\phi(S_i)^\gamma\in\cH$ for some $\gamma\in\{*_i\}$. $\cH$ is a
  $\Sigma$-Herbrand model for $\Psi$ and thus
  $\scapply\phi{C}\cap\cH=(\phi(C)\cup\restconst(\phi))\cap\cH\ne\emptyset$.
  Because $\cH$ cannot contain complementary literals we must already have a
  literal $\phi(L^\alpha)\in\phi(C)\cap\cH$. Now let $\nu$ be the valuation
  associated with $\cH$. Since $\phi(L^\alpha)\in\cH$ we have
  $\alpha=\nu(\phi(L))=\cI_\phi(L)$, which implies $\cM\models_\phi L^\alpha$. We
  have taken $C$ and $\phi$ arbitrary, so we get the assertion.
\end{proof}

\begin{theorem}[Refutation Completeness]\label{Lemma:unsat}
  $\presuppres$ is refutation complete, i.e., if $\Phi$ is an unsatisfiable set of
  ground clauses then there exists a $\presuppres$ derivation of the empty clause
  from $\Phi$.
\end{theorem}
\begin{sketch}
  The proof of completeness for unification-based refutation calculi is usually in
  two parts. First completeness for ground clauses is established, and then the
  result is generalized with a lifting argument. It turns out that the lifting
  property can be established by methods from~\cite{Weidenbach91}, since they are
  independent of the number of truth values. 
  
  Thus we only have to establish ground completeness. We do this using the
  so-called $k$-parameter induction method. For a clause set $\Phi$ let $k(\Phi)$
  be the number of literals minus the number of clauses in $\Phi$. If $k=0$ then
  all clauses in $\Phi$ are units. Since $\Phi$ is unsatisfiable there
  must be two complementary literals (otherwise a model can be
  constructed from $\Phi$). The empty clause can be deduced with a
  resolution step in the calculus. If $k>0$ then there is a non-unit clause
  $C\in\Phi$, which we can divide into two subclauses $D$ and $E$. By inductive
  hypothesis there is an $\presuppres$-refutation $\cD_D$ of $\Phi_D=\Phi\setminus
  C\cup\{D\}$. The refutation of $\Phi_E$ can be used to construct an
  $\presuppres$-derivation $\cD_D^E$ of the clause $E$ out of $\Phi$ (the
  rest $E$ of $D$ in $C$ is conserved during the proof). Again by inductive
  hypothesis, there is an $\presuppres$ derivation $\cD_E$ of $\Phi_E$. Thus
  the combined derivation $\cD_D^E,\cD_E$ is a refutation of $\Phi$ and we have
  completed the proof for the ground case.
\end{sketch}

\begin{corollary}[Completeness]
Any valid $\presuppl$ formula can be proved by $\presuppres$.
\end{corollary}
\begin{sketch}
  Let $A$ be a valid formula, then the labeled formula
  $A^{\twfs\twti\twfi\undefined}$ must be unsatisfiable.  By
  Theorem~\ref{Thm:cnf-theorem}, $\Phi\deq\Cnf(\{\{A^{\twfs\twti\twfi\undefined}\}\})$
  must be unsatisfiable as well, so by the refutation completeness result above,
  there is a refutation of $\Phi$, which is a proof of $A$ by definition.
\end{sketch}

Note that alternative soundness and completeness proofs can be built
using techniques from~\cite{GS00}, in which it has been shown that the
signed CNF translation is a translation to classical logic and that
superposition specializes to a refined version of the many-valued
resolution rules introduced by Baaz and Ferm\"uller. The use of this
argument would allow to directly apply a classical supperposition
principle with sorts~\cite{Weidenbach96}.

In the next two sections we will give two examples to examplify the two different main
phenoma we set out to model, firstly qantificational presuppositions, and secondly
existential presuppositions. These are two separate phenomena and modeled by two
independent extensions of standard first-order logic. The first is modeled by a restriction
system which assumes that restrictions are non-empty, the second is modeled by Bergmann's truth
values which distinguish between secure and insecure formulae.
 
\section{Example: Quantificational Presuppositions }
\label{sec:example}

In this section we want to see how the calculus deals with
Example~\ref{ex-John}. Remember, the company {\COMP} wants to fire
people, but they do not fire persons with children. We are worried
whether John will be fired, but then we hear that his children are
sleeping. Implicitly we can conclude from this information that John
has children, because if all his children are sleeping then John must
have children and hence will not be fired. The presupposition will be
derived from $\all{x}{{\sf child}(x,{\sf John})}{{\sf sleeps}(x)}$,
which in clause C2 guarantees the existence of a child of
John. Technically the existence follows from the restriction
constraint ${\sf child}(x,{\sf John})$, which must be non-empty. The
presupposition is not modeled by the many-valued aspect of the
logic~\presuppl.



\def\somespace{0.9ex}
This can be encoded in $\presuppl$ by the following statements:\\[\somespace]
\hspace*{2ex}\begin{minipage}[t]{0.7\textwidth}
\bd
A & $\all{x}{\Defined(x)}{{\sf fires}(\COMP,x)\impl \non {\sf parent}(x)}$\\
B & $\all{x}{\Defined(x)}{\ex{y}{{\Defined(y)}} {\sf child}(y,x)\impl{\sf parent}(x)}$\\
C & $\all{x}{{\sf child}(x,{\sf John})}{{\sf sleeps}(x)}$\\[1ex]

T & $\non {\sf fires}{(\COMP,{\sf John})}$
\ed
\end{minipage}\\[\somespace]
with the term declarations $(\Defined(\COMP))^\twts\!\!$,
$(\Defined({\sf John}))^\twts\!\!$, $(\Secure(\COMP))^\twts\!\!$, and
$(\Secure({\sf John}))^\twts$.
In order to prove the theorem T, the following generalized clause set
has to be refuted:\\[\somespace]
\hspace*{2ex}\begin{minipage}[t]{0.7\textwidth}
\bd
A & $(\all{x}{\Defined(x)}{{\sf fires}(\COMP,x)\impl \non {\sf parent}(x)})^\twts$\\
B & $(\all{x}{\Defined(x)}{\ex{y}{{\Defined(y)}} {\sf child}(y,x)}\impl{\sf parent}(x))^\twts$\\
C & $(\all{x}{{\sf child}(x,{\sf John})}{{\sf sleeps}(x)})^\twts$\\[1ex]

T & $(\non {\sf fires}{(\COMP,{\sf John})})^{\twti\twfs\twfi\undefined}$
\ed
\end{minipage}\\[\somespace]
By the rules for forming a clause normal form we get the clauses:\\[\somespace]
\hspace*{2ex}
\begin{tabular}{l@{$\;\;$}ll@{$\;\;$}l}
A1 & $({\sf fires}(\COMP,x_{\Defined}^+))^\twfs,({\sf parent}(x^+_{\Defined}))^\twfs$\\
A2 & ${\Defined}(c_1)^\twts$ &
A3 & ${\Secure}(c_1)^\twts$\\

B1 & \multicolumn{3}{l}{$({\sf child}(y^+_{\Defined},x^+_{\Defined}))^\twfs,({\sf parent}(x^+_{\Defined}))^\twts$}\\
B2 & ${\Defined}(c_2)^\twts$ &
B3 & ${\Secure}(c_2)^\twts$\\
B4 & ${\Defined}(c_3)^\twts$ &
B5 & ${\Secure}(c_3)^\twts$\\

C1 & $({\sf sleeps}(x^+_{{\sf child}(x,{\sf John})}))^\twts$\\
C2 & $({\sf child}(c_4,{\sf John}))^\twts$\\
C3 & ${\Defined}(c_4)^\twts$ &
C4 & ${\Secure}(c_4)^\twts$\\

T  & \multicolumn{3}{l}{$({\sf fires}(\COMP,{\sf John}))^\twfi,({\sf fires}(\COMP,{\sf John}))^{\twts},$}\\
   & \multicolumn{3}{l}{$({\sf fires}(\COMP,{\sf John}))^{\twti}$,$({\sf fires}(\COMP,{\sf John}))^{\undefined}$}\\
\ed\\[\somespace]
By resolution we get from Res(B1,C2):\\[\somespace]
\hspace*{2ex}\begin{minipage}[t]{0.7\textwidth}\bd
R1 & $({\sf parent}({\sf John}))^\twts, (\Defined({\sf
  John}))^{\twti\twfs\twfi\undefined},(\Secure({\sf John}))^\twfs,$\\
   & $\phantom{({\sf parent}({\sf John}))^\twts, }(\Defined(c_4))^{\twti\twfs\twfi\undefined},(\Secure(c_4))^\twfs$\\
\ed\end{minipage}\\[\somespace]
Two-times resolving with clauses from the term declarations and resolving with C3 and C4 results in:\\[\somespace]
\hspace*{2ex}\begin{minipage}[t]{0.7\textwidth}\bd
R2 & $({\sf parent}({\sf John}))^\twts$\\
\ed\end{minipage}\\[\somespace]
which in turn can be resolved with A1:\\[\somespace]
\hspace*{2ex}\begin{minipage}[t]{0.9\textwidth}\bd
R3 & $({\sf fires}(\COMP,{\sf John}))^\twfs,(\Defined({\sf John}))^{\twti\twfs\twfi\undefined},(\Secure({\sf John}))^\twfs$
\ed
\end{minipage}\vspace{\somespace}

The last two literals can again be resolved away using the term
declarations. T can be resolved four times with the resulting unit
$({\sf fires}(\COMP,{\sf John}))^\twfs$, whereby finally the empty
clause is derived.

Please note that in a direct translation of the above text into FOL, the essential information in C2 that John
has a child cannot be derived and hence no proof can be found.


\section{Example: Existential Presupposition}
\label{sec:expres}


In this example we show how we can {\em avoid\/} the presupposition in
classical logic that all constants exist just because of mentioning
them. For instance, classical logic is not
a good tool for a dispute of a theist and an atheist about the properties of God,
supposed he existed, since if the atheist only mentioned God he would admit the
existence of God. In $\presuppl$, however, the status of statements about
constants can be insecure and in particular no existence is assumed unless
otherwise specified by term declarations.

Let us look at a simplified version of Leibniz's argument that God cannot be
almighty, since if he were then there could not be any evil in the world.  The
argument is as follows: {\sl Assume God existed (1) and God were good (2).
  Furthermore we assume that everything that is created by an almighty and good
  being must be good (3). Since God has created the world (4) and the world is not
  good at all (5), God cannot be almighty (T).}

In our logic this can be formalized in a way that can be accepted by a theist and an
atheist alike:\\
\newcommand{\God}{{\sf God}}
\newcommand{\good}{{\sf good}}
\newcommand{\World}{{\sf World}}
\newcommand{\almighty}{{\sf almty}}
\newcommand{\created}{{\sf crtd}}
Term declarations:\\[\somespace]
\hspace*{2ex}\begin{minipage}[t]{0.9\textwidth}
\bd
(1)     & $\Secure(\God)^{\twts\twfs}\qquad
           \Defined(\God)^{\twts}\qquad
           \Secure(\World)^{\twts}\qquad
           \Defined(\World)^{\twts}$
\ed
\end{minipage}\\[\somespace]
Propositions:\\[\somespace]
\hspace*{2ex}\begin{minipage}[t]{0.9\textwidth}
\bd
(2)     & $\holds(\good(\God))$\\
(3)     & $\holds(\all{x}{\Defined(x)}{\all{y}{\Defined(y)}{
            \almighty(x)\land\good(x)\land\created(x,y) \implies
            \good(y)}})$\\
(4)     & $\holds(\created(\God,\World))$\\
(5)     & $\neg(\good(\World))$

\ed
\end{minipage}\\[\somespace]
Theorem:\\[\somespace]
\hspace*{2ex}\begin{minipage}[t]{0.9\textwidth}
\bd
(T)     & $\holds\neg\almighty(\God)$
\ed
\end{minipage}\vspace{\somespace}

Note that most of the statements are prefixed by the $\holds$
connective, which essentially says that the security value is of no
concern. The clause normal form of the assumptions and the negated
theorem looks like (using combined transformations as discussed
in Remark~\ref{Rem:combine}):\\[\somespace]
\hspace*{2ex}
\begin{minipage}[t]{0.965\textwidth} 
\bd 
     (2) &  $(\good(\God))^{\twts\twti}$\\ 
     (3) &  $(\almighty(x^\pm_\Defined))^{\twfs\twfi}\!,(\good(x^\pm_\Defined))^{\twfs\twfi}\!,
             (\created(x^\pm_\Defined,y^\pm_\Defined))^{\twfs\twfi}\!,(\good(y^\pm_\Defined))^{\twts\twti}$\\ 
     (4) & $(\created(\God,\World))^{\twts\twti}$\\
     (5) & $(\good(\World))^\twfs$\\ 
($\neg$T)& $(\almighty(\God))^{\twts\twti}$ 
\ed
\end{minipage}

It is now possible to resolve clause
(3) with the unit clauses (2), (4), (5), and ($\neg$T).  Because of the term
declarations $\Defined(\God)^{\twts}$ and $\Defined(\World)^{\twts}$, the
generated restriction constraints for the variables $x^\pm_\Defined$
and $y^\pm_\Defined$ can be resolved
away too so that we can derive the empty clause.

\section{Conclusion}
\label{sec:concl}

We have developed a five-valued logic for the formalization of
everyday reasoning with presuppositions and undefinedness. Our logic
is a common generalization of the system proposed by Bergmann
in~\cite{Bergmann:patdl81} for dealing with presuppositions and the one
proposed by Kleene~\cite{Kleene52} for dealing with undefinedness. We were able to show that they
together deal with two different  phenomena  of presuppositions, firstly qantificational presuppositions, and secondly
existential presuppositions. These are two separate phenomena and modeled by two
independent extensions of standard first-order logic. The first is modeled by a restriction system (which is comparable to a sort system)
which assumes that restrictions are non-empty, the second is modeled by Bergmann's truth
values which distinguish between secure and insecure formulae.

Furthermore we have presented a sound and complete resolution calculus
for our system, which uses the restriction mechanism to capture bounded
quantifications. Our calculus can be seen as an extension of classical logic that combines
methods from many-valued logics, cf.~\cite{BaFe92,Haehnle:adimvl94}, for a correct
treatment of the secure and insecure or undefinedness information on the one
hand and
order-sorted logics, see~\cite{Weidenbach91}, for an adequate treatment of
restricted domains, on the other hand.  In contrast to the mere partial function calculi
in~\cite{KeKo94,KeKo:atcfpf96}, from which we have drawn most of the technical
ideas, in this paper we strictly distinguish between undefined terms which
are all identified (with a special ``undefined object'' in each model), and insecure terms, which can be different from
each other.  However, just like in the mere Kleene case, most definedness
preconditions can be taken care of in the unification, making inferences
quite efficient. 

Although certainly the phenomenon of presuppositions cannot be fully captured
by a static analysis and in fact research on presuppositions in linguistics
has nowadays turned mostly to dynamic and more pragmatically driven analyses, and
away from the multi-valued treatment, this is not a counter-argument to our
system. In contrast to FOL our system $\presuppl$ makes it possible to
specify (and reason with) presuppositions and undefinedness.
First steps for developping dynamic versions of logics have been taken
in~\cite{Saurer:andsfd93,ReyGab:ddcodrs94,MonRij:arcfds98}. Work by the
second author that links dynamic features and inference in DRT to the dynamic
introduction of witnesses in tableau-based model generation~\cite{Kohlhase:mgfdrt00}
provides a promising setting  for developping a
dynamic version of $\presuppl$. This would alleviate one central
methodological weakness of the linguistic analyses, which use (intuitive) reasoning on
the meta-level to argue about presupposition and definedness in the presence of world
knowledge without being able to make this formal and thus part of the analysis.

\newif\ifgerman\germanfalse\newif\iffrench\frenchfalse
\newcommand{\CADE}{{CADE}}
\newcommand{\IJCAI}{{IJCAI}}
\newcommand{\LPAR}{{LPAR}}
\newcommand{\JACM}{Journal of the ACM}
\newcommand{\JAR}{Journal of Automated Reasoning}
\newcommand{\JSL}{The Journal of Symbolic Logic}
\newcommand{\LNAI}{LNAI}
\newcommand{\PROC}{{Proceedings}}
\def\UdS{Universit\"at des Saarlandes}
\def\AddressUdS{Im Stadtwald, \Saarbruecken\Germany}
\def\UdSAddress{\AddressUdS}
\def\Saarbruecken{\ifgerman{Saar\-br\"ucken}\else{\iffrench{Sarrebruck}\else{Saar\-br\"ucken}\fi}\fi}
\def\FRANCE{\ifgerman{Frankreich}\else{France}\fi}
\def\GERMANY{\ifgerman{}\else{\iffrench{, Allemagne}\else{, Germany}\fi}\fi}\def\Germany{\GERMANY}
\def\NETHERLANDS{\ifgerman{Niederlande}\else{\iffrench{Pays Bas}\else{The Netherlands}\fi}\fi}
\def\RUSSIA{\ifgerman{Ru{\ss}land}\else{\iffrench{Russie}\else{Russia}\fi}\fi}



\bibliography{kwarc}

\end{document}

%%% Local Variables: 
%%% mode: latex
%%% TeX-master: t
%%% End: 
