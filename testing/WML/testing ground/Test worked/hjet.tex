In diesem Kapitel soll ein Feld von Minimalfl�chen um eine gegebene 
kompakte stabile Minimalfl�che $\MM$ konstruiert werden. 
Dazu wird eine Schar von Parallelfl�chen $M_s$ um $M$ zu einer Schar von 
Minimalfl�chen verbogen. Dann wird gezeigt, da� die einzelnen Bl�tter
der Schar sich nicht schneiden und eine Tubenumgebung von $M$ 
�berdecken. \par
Dazu mu� man zuerst das Vektorfeld der mittleren Kr�mmung an die
induzierten Mannigfaltigkeiten betrachten.
\Para{Die Operatoren $H_s$, $L_s$ und $E_s$}{}
Wir betrachten den Operator $H$, der einem Schnitt im Normalenb�ndel
von $M$ das Vektorfeld der mittleren Kr�mmung an die induzierte 
Mannigfaltigkeit zuordnet. Ein Schnitt $u$ mit $H(u)=0$ induziert also 
eine Minimalfl�che. 
Nach {\HLAPLACE} ist $H$ gerade der Laplace-Beltrami-Operator.\par
Nun ist der mittlere Kr�mmungsvektor ein Normalenvektor an die induzierte
Mannigfaltigkeit und deswegen im allgemeinen nicht normal an $M$. 
Wir betrachten also die Projektion $H^\bot$ von $H$ auf das Normalenb�ndel 
von $M$. 
Ist $\pi_{\N M}\colon T\widetilde M\longrightarrow \N M$ die Projektion auf das 
Normalenb�ndel von $M$, so ist der Operator 
$$H^\bot=\pi_{\N M}\circ H\colon 
      \C^{2,\alpha}(M,\N M)\longrightarrow \C^{0,\alpha}(M,\N M)$$
linearer Differentialoperator.
\Lemma {}{\NORMPRO}
{\it Es gibt ein $\epsilon_0>0$, so da� f�r alle $u\in \C^1(M,\N M)$ 
mit $|u|_{\C^1(M,\N M)}<\epsilon_0$, und f�r alle Vektorfelder
$V\in L^1(M_u,T\widetilde M)$ gilt:
$$\pi_{\N M}(V) = 0 \quad \Longleftrightarrow \quad V = 0 $$}
\Beweis
Andernfalls w�re in einem Punkt $p\in M$ $V_u(p)\perp \nu_p$ und 
damit $|u|_{C^1(M,\N M)}=\infty$. \kasten
\medskip
Es reicht also zu zeigen, da� $H^\bot(u)=0$.\par
In lokalen Koordinaten gilt dann wegen {\LAPLACELOC},
$$H(u)={1\over {\sqrt{g(u)}}}{\partial \over \partial x_i}
       \bigl (\sqrt {g(u)} g^{ij}(u)
              {\partial \over \partial x_j}(\psi +u) 
       \bigr )$$
mit $g_{ij}(u)=(\psi +u)_{x_i} (\psi +u)_{x_j}$, 
$\bigl ( g^{ij} (u) \bigr )= \bigl ( g_{ij}(u)\bigr )$,
$g(u)=\det\bigl ( g_{ij}(u) \bigr )$. \par
Wir wollen nun kleine St�rungen der Schar der Parallelfl�chen von $M$ 
betrachten.
Die Parallelfl�chen $M_s$ von $M$ werden von den Schnitten 
$s\nu\in \C^\infty(M,\N M)$ induziert und haben das gleiche Normalenb�ndel wie
$M$. $H_s(u)\colon =H^\bot(s\nu+u)$ ordnet einem Schnitt $u$  in $\N M$ das mittlere
Kr�mmungsvektorfeld der durch $u$ �ber $M_s$ induzierten Fl�che zu.\par
Es gilt nach dem Satz von Taylor
$$\eqalign{H_s(u)=&H^\bot(s\nu + u)                                 \cr 
        =& H_s(0)+{{dH_s \over dt}(tu) \bigr |}_{t=0}
           + \int _0^1 {{(1-\tau){d^2 H_s \over dt^2}(tu) \bigr |}
           _{t=\tau} d\tau}                                       \cr
       = & H_s+L_s(u)+ E_s(u)                                     \cr
            }$$
$L_s$ ist der Jacobi-Operator von der Parallelfl�che $M_s$ und als solcher
selbstadjungierter, stetiger linearer Differentialoperator. \par
Im folgenden sei $\lambda _s$ der kleinste Eigenwert von $L_s$. Er h�ngt
nach {\LAMBDASTET} stetig von der Immersion und damit stetig von $s$ ab. 
Wir k�nnen daher ohne Einschr�nkung annehmen, da� $\lambda_s>\lambda_0/2$
f�r alle $s<\rho$. Wegen {\SPEKSATZ} ist $L_s$ f�r alle $s\in[-\rho,\rho]$ 
invertierbar.\par
Jetzt k�nnen wir das Problem
$$H_s(u)=H_s+L_su+E_su=0$$
umformulieren in das Fixpunktproblem
$$T_su\colon =-L_s^{-1}(H_s)-L_s^{-1}E_su=u\;.$$
\Bem{}{}
$$T_s\colon \C\longrightarrow \C^{2,\alpha}(M,\N M)$$
ist f�r alle $s\in [-\rho,\rho]$ ein stetiger linerarer Operator.\par
Dieses Fixpunktproblem wird im folgenden mit dem Banachschen Fixpunktsatz
gel�st.\par
\Para{Formeln f�r diese Operatoren}{}
$$\eqalign{ 
   H^\bot(s\nu+tu)=& {1\over \sqrt{g(s\nu + tu)}} 
           \bigl ( \sqrt {g(s\nu + tu)} \bigr )_{x_j}
                  g^{ij}(s\nu + tu) (\psi + s\nu + tu)_{x_j}          \cr
         & + \bigl (g^{ij} (s\nu + tu)\bigr )_{x_i} 
                  (\psi + s\nu + tu)_{x_j}                            \cr
         & + g^{ij}(s\nu + tu)(\psi +tu)_{x_ix_j}                     \cr
       = & {1\over 2g(s\nu + tu)} g_{x_i}(s\nu + tu)
               g^{ij}(s\nu + tu)( \psi +s\nu )_{x_j}                  \cr
         & + {t\over 2g(s\nu + tu)} g_{x_i}(s\nu + tu)
              g^{ij}(s\nu + tu)u_{x_j}                                \cr
         & + g^{ij}(s\nu + tu)(\psi + s\nu) _{x_j} 
           + tg^{ij}_{x_i}(s\nu + tu)u_{x_j}                          \cr
         & + g^{ij}(s\nu + tu)(\psi + s\nu)_{x_ix_j} 
           + tg^{ij}(s\nu + tu)u_{x_ix_j}                             \cr
           }$$
Weil aber $(\psi +s\nu)_{x_i} $ f�r alle $i$ tangential zu $M$ ist, gilt
$$\eqalignno{ 
   H_s(tu)= 
                &{t\over 2g(s\nu + tu)} g_{x_i}(s\nu + tu)
                        g^{ij}(s\nu + tu)u_{x_j}^\bot             &(H_1) \cr
                &+ tg^{ij}_{x_i}(s\nu + tu)u_{x_j}^\bot           &(H_2) \cr
                &+ g^{ij}(s\nu + tu)(\psi + s\nu)_{x_ix_j}^\bot   &(H_3) \cr
                &+ tg^{ij}(s\nu + tu)u_{x_ix_j}^\bot\;.           &(H_4) \cr
           }$$

\Lemma {}{}
{\it $H_s$ ist in $s$ stetig differenzierbar.}
\Beweis $g_{ij}(s\nu)=\psi_{x_i}\psi_{x_j}
                   +s(\psi_{x_i}\nu_{x_j}+\nu_{x_i}\psi_{x_j})
                   +s^2\nu_{x_i}\nu_{x_j}$ 
ist ein Polynom in $s$, $g^{kl}$ ist eine rationale Funktion in $g^{ij}$,
damit ein rationale Funktion in $s$ und als solche von der Klasse $\C^1$.
$H_s=H^\bot(s\nu)=g^{ij}(s\nu)\bigl ( \psi + s\nu \bigr )_{x_ix_j}$
ist also von der Klasse $\C^1$ in $s$.\kasten
\Satz{}{\HCNULL}
{\it F�r die Mittlere Kr�mmung $H_s$ der Parallelfl�chen gilt
$$|H_s|_{C^0} \leq Ks.$$}
\Beweis $H_s(x)=H_0(x) + s {{dH(x)\over ds}\Bigl |}_{s=\xi(x)}$ f�r ein 
$0\leq\xi(x) \leq s$. 
\par 
F�r $K\colon =\sup_{x\in \MM}\bigl ({{dH(x)\over ds}\Bigr |}_{s=\xi(x)}\bigr )$
ergibt sich die Behauptung.\kasten
\Lemma {$L^p$ Absch�tzungen} {\LPABSCH}
{\it Sei $\lambda_s >0$, dann gilt: 
$$\lpu \leq {1/\lambda _s} \lpnorm {L_su}$$} 
\Beweis
\def\p2{{p/2}}
$$\eqalign{\lpu ^p &= \int {|u|^\p2 |u|^\p2}                            \cr
                   &\leq \lambda_s ^{-\p2}\int{|u|^\p2 |Lu|^\p2}        \cr
                   &\leq \lambda_s^{-\p2} 
                         \bigl | |u|^\p2 \bigr |_{L^2}
                         \bigl | |L_su|^\p2 \bigr |_{L^2}               \cr
                   &=\lambda_s^{-\p2} \lpu ^\p2 \lpnorm {L_su}^\p2\;.   \cr
          }$$
\par Aus $\lpu ^\p2 \leq \lambda_s^{-\p2} \lpnorm {L_su}^\p2$
folgt dann die Behauptung.\kasten

