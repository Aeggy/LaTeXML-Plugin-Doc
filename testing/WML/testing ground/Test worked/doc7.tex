%febrero 2014

\documentclass[12pt]{amsart}
\usepackage[all]{xy}
\usepackage{graphicx}
\usepackage{amssymb}
\setlength{\textheight}{8.0in}
\setlength{\textwidth}{6.0in}


\newenvironment{lemma}{\smallskip\begin{trivlist}
 \item[\hspace{\labelsep}{\noindent\bf Lemma.}]\it
 }{\end{trivlist}\smallskip}
\newenvironment{propo}{\smallskip\begin{trivlist}
 \item[\hspace{\labelsep}{\noindent\bf Proposition.}]\it
 }{\end{trivlist}\smallskip}
\newenvironment{defin}{\smallskip\begin{trivlist}
 \item[\hspace{\labelsep}{\noindent\em Definition:}]
 }{\end{trivlist}\smallskip}
\newenvironment{theor}{\smallskip\begin{trivlist}
 \item[\hspace{\labelsep}{\noindent\bf Theorem.}]\it
 }{\end{trivlist}\smallskip}
\newenvironment{theors}{\smallskip\begin{trivlist}
 \item[\hspace{\labelsep}{\noindent\bf Theorem}]\it
 }{\end{trivlist}\smallskip}
\newenvironment{coro}{\smallskip\begin{trivlist}
 \item[\hspace{\labelsep}{\noindent\bf Corollary.}]\it
 }{\end{trivlist}\smallskip}
\newcommand{\cuadro}{\hfill{$\qed$}}
\newenvironment{notation}{\begin{trivlist}
\item[\hspace{\labelsep}{\noindent\em Notation:}]
}{\end{trivlist}}
\newenvironment{proofs}{\begin{trivlist}  %Proof sin cuadro
\item[\hspace{\labelsep}{\noindent\em Proof:}]
}{\end{trivlist}}
\newenvironment{claim}{\smallskip\begin{trivlist}
 \item[\hspace{\labelsep}{\noindent\em Claim:}]
 }{\end{trivlist}\smallskip}
\newenvironment{example}{\smallskip\begin{trivlist}
 \item[\hspace{\labelsep}{\noindent\bf Example:}]
 }{\end{trivlist}\smallskip}
\newenvironment{examples}{\smallskip\begin{trivlist}
 \item[\hspace{\labelsep}{\noindent\bf Examples:}]
 }{\end{trivlist}\smallskip}
\newenvironment{remark}{\smallskip\begin{trivlist}
 \item[\hspace{\labelsep}{\noindent\em Remark:}]
 }{\end{trivlist}\smallskip}

\newcommand{\aA}{\lower1pt\hbox{$\!_A$}}
\newcommand{\phia}{\varphi_{\aA}}
\newcommand{\aAm}{\lower1pt\hbox{$\!_{A_m}$}}
\newcommand{\phiam}{\varphi_{\aAm}}
\newcommand{\A}{\mathbb{A}}
\newcommand{\C}{\mathbb{C}}
\newcommand{\D}{\mathbb{D}}
\newcommand{\N}{\mathbb{N}}
\newcommand{\R}{\mathbb{R}}
\newcommand{\T}{\mathbb{T}}
\newcommand{\X}{\mathbb{X}}
\newcommand{\Z}{\mathbb{Z}}
\newcommand{\mA}{\mathcal{A}}
\newcommand{\mF}{\mathcal{F}}
\newcommand{\mG}{\mathcal{G}}
\newcommand{\mI}{\mathcal{I}}
\newcommand{\mP}{\mathcal{P}}
\newcommand{\mR}{\mathcal{R}}
\newcommand{\mS}{\mathcal{S}}
\newcommand{\mX}{\mathcal{X}}
\newcommand{\mY}{\mathcal{Y}}
\newcommand{\corank}{\hbox{\rm corank}}
\renewcommand{\det}{\hbox{\rm det}}
\renewcommand{\dim}{\hbox{\rm dim}}
\newcommand{\udim}{\hbox{\bf dim}}
\newcommand{\Ext}{\hbox{\rm Ext}} 
\newcommand{\End}{\hbox{\rm End}}
\newcommand{\Hom}{\hbox{\rm Hom}} 
\newcommand{\oHom}{\overline{\hbox{\rm Hom}}}
\newcommand{\uHom}{\underline{\hbox{\rm Hom}}\,}
\newcommand{\id}{\hbox{\rm id}}
\newcommand{\iIm}{\hbox{\rm Im}}
\newcommand{\ind}{\hbox{\rm ind}} 
\renewcommand{\ker}{\hbox{\rm ker}}
\renewcommand{\max}{\hbox{\rm max}}
\renewcommand{\mod}{\hbox{\rm mod}}
\newcommand{\omod}{\overline{\hbox{\rm mod}}}
\newcommand{\umod}{\underline{\hbox{\rm mod}}\,}
\newcommand{\rad}{\hbox{\rm rad}} 
\newcommand{\Stab}{\hbox{\rm Stab}} 
\newcommand{\supp}{\hbox{\rm supp}\,}
\newcommand{\Tor}{\hbox{\rm Tor}} 
\newcommand{\tr}{\hbox{\rm tr}}
\def\raya{\raise1.5pt\hbox to 25pt{\vrule height1.5pt depth-1pt
           width25pt}}
\newcommand{\bulito}{\ {\scriptstyle \bullet}\, \ }
\def\subsetnoteq{\mathbin{\hbox{$\subseteq \joinrel \hskip-8pt \lower3pt
                 \hbox{$\scriptscriptstyle /$}\ $}}}
\def\aderecha#1{\smash{\mathop{\longrightarrow}\limits^{#1}}}
\def\toto{\aderecha{} \hskip-16pt \aderecha{}}
\def\totol#1{\smash{\mathop{\aderecha{} \hskip-16pt
\aderecha{}}\limits^{#1}}}
\def\mapdown#1{\Bigg\downarrow \rlap{$\vcenter{\hbox{$\scriptstyle#1$}}$}}
\def\mapdowndown#1{\Bigg\downarrow \hskip-7pt\lower10pt\hbox{$\downarrow$}
        \rlap{$\vcenter{\hbox{$\scriptstyle#1$}}$}}
\def\vmapdown#1{\raise5pt\hbox{$\sssty\vert$} 
         \hskip-2.25pt \raise17pt\hbox{$\sssty\vert$} 
         \hskip-4.25pt \lower13pt\hbox{$\downarrow$}
         \rlap{$\vcenter{\hbox{$\scriptstyle#1$}}$}}
\def\sssty{\scriptscriptstyle}

\begin{document}
\baselineskip=18pt plus 1 pt

\title{Maximum Walk Entropy Implies Walk Regularity}
\maketitle
\authors{Ernesto Estrada$^{1,2}$ and Jos\'e  A. de la Pe\~na$^{2,3}$}

\noindent{{\footnotesize $^1$ Department of Mathematics and Statistics, University of Strathclyde, Glasgow G1 1XH, U.K.,
$\qquad ^2$ CIMAT, Guanajuato, 36240 M\'exico, $^3$ Instituto de Matem\'aticas, UNAM, M\'exico, 04510, M\'exico}}
\begin{abstract}
The notion of walk entropy $S^V(G,\beta)$  for a graph $G$   at the inverse temperature $\beta$  was put forward recently by Estrada et al. (2014) \cite{6}. It was further proved by Benzi \cite{1} that a graph is walk-regular if and only if its walk entropy is maximum for all temperatures $\beta \in I$, where $I$ is a set of real numbers containing at least an accumulation point. Benzi \cite{1} conjectured that walk regularity can be characterized by the walk entropy if and only if there is a $\beta>0$ , such that $S^V(G,\beta)$  is maximum. Here we prove that a graph is walk regular if and only if the $S^V(G,\beta=1)=\ln n$. We also prove that if the graph is regular but not walk-regular $S^V(G,\beta)<\ln n$  for every   $\beta >0$ and $\lim_{\beta \to 0} S^V(G,\beta)=\ln n=\lim_{\beta \to \infty} S^V(G,\beta)$. If the graph is not regular then 
$S^V(G,\beta) \leq \ln n-\epsilon$ for every $\beta>0$,  for some  $\epsilon>0$.

\qquad MSC: 05C50; 15A16; 82C20.

\qquad Keywords: Walk-regularity; Graph entropies; Graph walks.
\end{abstract}

	
\section{Introduction}
The concept of walk entropy was recently proposed as a way of characterizing graphs using statistical mechanics concepts \cite{6}. For a simple, undirected graph $G=(V,E)$  with   nodes $1 \leq i \leq n$  and adjacency matrix $A$  the walk entropy is defined as
$$S^V(G,\beta)=-\sum\limits_{i=1}^n p_i(\beta) \ln p_i(\beta),$$
where $p_i(\beta)=(e^{\beta A})_{ii}/Z$  and $\beta=1/k_B T >0$ (where, $k_B$ is the Boltzmann constant  and $T$ is  the absolute temperature). Here $Z=\tr(e^{\beta A})$  represents the partition function of the graph, frequently referred to in the literature as the Estrada index of the graph \cite{3, 4, 9}. The term $(e^{\beta A})_{ii}$  represents the weighted contribution of every subgraph to the centrality of the corresponding node, known as the subgraph centrality $SC(i)$  of the node \cite{7, 5, 8}. The walk entropy called immediately the attention in the literature \cite{1} due to its many interesting mathematical properties as well as its potential for characterizing graphs and networks. In \cite{6} the authors stated a conjecture which was subsequently proved by Benzi \cite{1} as the following 
\vskip.5cm
{\bf Theorem 1.1.} \cite{1} {\em A graph $G$ is walk-regular if and only if $S^V(G,\beta)=\ln n$  for all $\beta \geq 0$ .}
\vskip.5cm

Benzi [1] also reformulated another conjecture stated by Estrada et al. [6] in the following stronger form
\vskip.5cm
{\bf Conjecture 1.2. }[1] A graph is walk-regular if and only if there exists a $\beta>0$  such that  
$S^V(G,\beta)=\ln n$.
\vskip.5cm
A third conjecture to be considered here generalizes the graphic examples given by Estrada et al. \cite{6} and can be stated as 
\vskip.5cm
{\bf Conjecture 1.3.} Let $G$ be a non-regular graph, then $S^V(G,\beta)<\ln n$  for every $\beta>0$.

In this note we prove these two conjectures, which immediately imply that the walk-entropy is a strong characterization of the walk-regularity in graphs and also gives strong mathematical support to the strength of this graph invariant for studying the structure of graphs and networks. 

\section{Main results}

We start here by stating the two main results of this work.
\vskip.5cm
{\bf Theorem 2.1.} {\em  Let $A$  be the adjacency matrix of a connected graph  $G$. Then the following conditions are equivalent:

{\rm (a)}	$G$  is walk-regular;

{\rm (b)}	$A^k$  has a constant diagonal for natural numbers $k$;

{\rm (c)}	$e^A$  has constant diagonal;

{\rm (d)}	$e^{\beta A}$  has constant diagonal for  $\beta\geq 0$;

{\rm (e)}	The walk entropy  $S^V(G,1)=\ln n$.}
\vskip.5cm
{\bf Theorem 2.2.} {\em Let $A$  be the adjacency matrix of a graph $G$. Then one and only one of the following conditions holds:

{\rm (a)}	$G$  is walk-regular. Then $S^V(G,\beta)=\ln n$  for every  $\beta>0$;

{\rm (b)}	$G$  is a regular but not walk-regular graph. Then $S^V(G,\beta)<\ln n$  for every $\beta>0$. Moreover, $\lim_{\beta \to 0} S^V(G,\beta)=\ln n=\lim_{\beta \to \infty} S^V(G,\beta)$;

{\rm (c)}	There is some $\epsilon>0$  such that $S^V(G,\beta)\leq \ln n-\epsilon$ for every  $\beta>0$.}
\vskip.5cm
To avoid cross-reference in the proofs of the above Theorems we present first the proof of Theorem 2.2.

\section{Auxiliary definitions and results}

Before stating the proof of the Theorem 2.2 we need to introduce some definitions and auxiliary results, which are given below. We remind the reader that given a set $X=\{x_1,\ldots,x_s\}$  of real numbers, the {\em variance} is defined as
$$\sigma^2(X)=E(X^2)-(E(X))^2=\frac{1}{s} \sum\limits_{i=1}^s x_i^2-\left(\frac{1}{s} \sum\limits_{i=1}^s x_i \right)^2.$$

{\bf Definition 3.1}: Given a matrix $M$  with diagonal entries $M_{11},\ldots,M_{nn}$ , not all zero, we introduce the {\em diagonal variance} as
$$\sigma_d^2(M)=\frac{1}{\sum\limits_{i=1}^n |M_{ii}|} \sigma^2(M_{11},\ldots,M_{nn}).$$
\noindent
Let us now state and proof the following auxiliary result. We notice in passing that the diagonal variance of $e^A$ was studied by Ejov et al. \cite{4a} in a different context for regular graphs.
\vskip.5cm
{\bf Proposition 3.2}: {\em Let $A$  be the adjacency matrix of a connected graph  $G$. Then one of the following conditions holds:

{\rm (a)}	$e^A$  has constant diagonal;

{\rm (b)}	$e^A$  has no constant diagonal entries and $G$  is a regular graph. Then $\sigma_d^2(e^{\beta A})>0$  for $\beta>0$ and 
$\lim_{\beta \to \infty}\sigma_d^2(e^{\beta A})=0$;

{\rm (c)}	There is some $\epsilon >0$  such that $\sigma_d^2(e^{\beta A})>\epsilon$  for every 
$\beta>0$.}

{\bf Proof}: We distinguish the following mutually excluding cases:

(1)	$G$  is walk-regular which implies that $e^A$ has constant diagonal. 

(2)	$e^{\beta A}$ does not have constant diagonal entries, for any $\beta>0$.  Then $\sigma_d^2(e^{\beta A})>0$ for $\beta>0$.

Observe that for $\beta>0$  we have 
$ (e^{\beta A})_{ii}\sim \phi_1^2 (i) e^{\beta \lambda_1}$ and 
$Z(\beta A) \sim e^{\beta \lambda_1}$ , where  $\phi_1$ is the (Perron) eigenvector of  $A$ corresponding to the maximal eigenvalue $\lambda_1$ . Here the symbol $\sim$ means that the quantities are asymptotically equal.

In that situation 
$$\lim_{\beta \to \infty} \sigma_d^2(e^{\beta A})=\frac{1}{Z(\beta)} \sigma_d^2
\left((e^{\beta A})_{ii}: 1 \leq i \leq n \right)=\sigma_d^2(\phi_1^2 (i): 1 \leq i \leq n) .$$
\noindent
Therefore $\lim_{\beta \to \infty} (e^{\beta A})=0$  is equivalent to $\phi_1$  being constant, or $G$ being regular. 
\noindent
If $G$  is not regular then the analytic function $\sigma_d^2(e^{\beta A})>0$, for $\beta>0$,  and $\lim_{\beta \to \infty} \sigma_d^2(e^{\beta A})>0$ . Clearly, there is some $\epsilon>0$  such that $\sigma_d^2(e^{\beta A})\geq \epsilon$, for every $\beta >0$ .   \qed
\vskip.5cm
We continue now with some other auxiliary results needed to prove the Theorem 2.2. 
Let $\lambda_1,\cdots,\lambda_n$  be the eigenvalues of $A$, such that 
$\sum\limits_{j=1}^n  \lambda_j = 0$ (since $G$ is a simple graph without loops). For the vector of diagonal entries 
$y=(y_1,\cdots,y_n)$ of $e^{\beta A}$  we define a vector $z=\ln y=(\ln y_1,\cdots,\ln y_n)$  of real numbers. We have
 $$\sum\limits_{i=1}^n z_i e^{z_i}=\sum\limits_{i=1}^n y_i \ln y_i$$
with $\sum\limits_{i=1}^n z_i =\ln \prod_{i=1}^n y_i \geq \ln \det(e^{\beta A})=
\sum\limits_{j=1}^n  \lambda_j = 0$, where the inequality is a direct application of Hadamard's theorem for the positive definite matrix  $e^{\beta A}$, 
see for instance \cite{11}. The remarkable result of Borwein and Girgensohn \cite{2} states the following.

{\bf Theorem 3.4}. {\em Let $c_n=2 (n=2,3,4)$  and $c_n=e(1-1/n)(n \geq 5)$. Let $z_i$  be defined as before. Then \cite{2} yields
$$ \frac{c_n}{n} \sum\limits_{i=1}^n z_i^2 \leq \sum\limits_{i=1}^n z_i e^{z_i}.$$}
\vskip.5cm
 
\section{ Proof of the Theorem 2.2}

We know that $S^V(G,\beta) \leq \ln n$  for every $\beta>0$. Observe that for 
$Z(\beta)=\tr(e^{\beta A})$ the walk vertex entropy is
$$S^V(G,\beta) =\ln Z-\frac{1}{Z} \sum\limits_{i=1}^n z_i e^{z_i}|_{\beta}$$
\noindent
The Borwein-Girgersohn inequality yields
$$S^V(G,\beta) \leq \ln Z-\frac{1}{Z} \frac{c_n}{n}\sum\limits_{i=1}^n z_i^2|_{\beta}$$
\noindent 
We distinguish two situations at $\beta>0$ :

$(1)$ $\sum\limits_{i=1}^n z_i^2|_{\beta}	=0$ , that is $y_i(\beta)=1$  for $i=1,\ldots,n$ . Then, $Z(\beta)=n$ which is only possible if $A=0$. Therefore $S^V(G,\gamma)=\ln n$ for any 
$\gamma>0$.
 
$(2)$ $\sum\limits_{i=1}^n z_i^2>0$	. Then there is a differentiable function $c_n \leq d_n(\beta)$  such that
$$S^V(G,\beta) = \ln Z-\frac{1}{Z} \frac{d_n}{n}\sum\limits_{i=1}^n z_i^2|_{\beta}< \ln n.$$
\noindent
Since $Z \geq n$  there is a differentiable function $e_n$  satisfying 
$0<e_n(\beta) \leq d_n(\beta)$  such that
$$S^V(G,\beta) = \ln n- \frac{e_n}{n^2}\sum\limits_{i=1}^n z_i^2|_{\beta}.$$
\noindent
For every $M>0$, using the compactness of the interval $[0,M]$, there exists an 
$\epsilon(M)>0$  such that $ \frac{e_n}{n^2}\sum\limits_{i=1}^n z_i^2|_{\beta}\geq \epsilon(M)$  for $\beta \in (0,M]$. 
Choose $\epsilon(M)$ such that
$$\inf \{\epsilon(M):0<M \}=\lim\limits_{\beta \to \infty} \frac{e_n}{n^2}\sum\limits_{i=1}^n z_i^2|_{\beta}.$$

Moreover, recall from \cite{6} that
$$S^V(G,\beta \to \infty)=-\sum\limits_{i=1}^n \phi_1^2(i) \ln \phi_1^2(i).$$
\noindent
This limit is $< \ln n$  except when there is a common value $\phi_1(i)=c_1$, for $i=1,\ldots,n$. The latter property implies that $G$  is a regular graph. We consider these cases separately.

$(3)$	Assume that $G$  is not a regular graph. Then $S^V(G,\beta \to \infty)< \ln n$.  Therefore there exists an $\epsilon>0$  such that for $M>0$  we have $\epsilon(M) \geq \epsilon$.
That is, $S^V(G,\beta) \leq \ln n-\epsilon$, for $\beta>0$.

$(4)$	Assume that $G$  is a regular but not a walk-regular graph. Then, according 
with the analysis in 3.2, the maximal value $S^V(G,\beta)=\ln n$  is not attained for any $\beta>0$. Moreover,  
$$\lim_{\beta \to 0} S^V(G,\beta)=\ln n=\lim_{\beta \to \infty} S^V(G,\beta)$$
 \qed


\section{ Proof of the Theorem 2.1} The following are obvious implications:

(a) implies (b), (a) implies (d), (d) implies (c), (c) implies (e), which leaves open only two implications.

For (b) implies (a), let
$$p(T)=T^n+p_{n-1}T^{n-1}+\cdots+p_0$$  
\noindent
be the characteristic polynomial of the graph  $G$. The Cayley-Hamilton theorem yields  $p(A)=0$. If $A^k$  has a constant diagonal for natural numbers $0 \leq k \leq m$  and  
$n-1 \leq m$, then
$$ A^{m+1}=-(p_{n-1}A^m+\cdots+p_0 A^{m-n+1})$$
has a constant diagonal.

(e) implies (a):  follows from Theorem 2.2 \qed
\vskip.5cm
In closing, the maximum of the walk entropy at $\beta=1$ , i.e., $S^V(G,1)=\ln n$ , is attained only for the walk-regular graphs. This means that $S^V(G,1)$  can be used as an invariant to characterize walk-regularity in graphs.

\vskip.5cm
{\bf Acknowledgement}: We thank the referees for suggestions on the presentation of the paper.
\vskip.5cm

\begin{thebibliography}{10000}

\bibitem{1}	M. Benzi, {\em A note on walk entropies in graphs,} Linear Algebra Appl. 445 (2014) 395-399.

\bibitem{2}	J. Borwein, R. Girgensohn, {\em A class of exponential inequalities}, Math. Inequal. Appl., 6(3), 2003,  397–411.

\bibitem{3}	J.A. de la Pe\~na, I. Gutman, J. Rada, {\em  Estimating the Estrada index}, Linear Algebra Appl. 427 (2007) 70-76.

\bibitem{4}	H. Deng, S. Radenkovic, I. Gutman, {\em The Estrada index. Applications of Graph Spectra}, Math. Inst., Belgrade, (2009) 123-140.

\bibitem{4a} V. Ejov, J.A. Filar, S.K. Lucas and P. Zograf, {\em Clustering of spectra and fractals of regular graphs}, J.  Math. Anal. Appl. 333 (2007) 236-246.

\bibitem{5}	E. Estrada, {\em The Structure of Complex Networks. Theory and Applications}, Oxford University Press, UK, 2011.

\bibitem{6}	E. Estrada, J.A. de la Pe\~na, N. Hatano,{\em  Walk entropies in graphs}, Linear Algebra Appl. 443 (2014) 235-244.

\bibitem{7}	E. Estrada, J.A. Rodr\'iguez-Vel\'azquez, {\em Subgraph centrality in complex networks}, Phys. Rev. E 71 (2005) 671-696.

\bibitem{8}	E. Estrada, , N. Hatano, M. Benzi, {\em The physics of communicability in complex networks}, Phys. Rep. 514 (2012) 89-119.  

\bibitem{9}	I. Gutman, H. Deng, S. Radenkovic, {\em The Estrada index: an updated survey. Selected Topics on Applications of Graph Spectra}, Math. Inst., Beograd, (2011) 155-174.

\bibitem{10}	B. Kostant, P. W. Michor,{\em  The generalized Cayley map from an algebraic group to its Lie algebra}, In The orbit method in geometry and physics, pp. 259-296. Birkhauser Boston, 2003.

\bibitem{11}   F. Zhang, {\em Matrix Theory: Basic results and Techniques}, Springer (1999).
\end{thebibliography}
\end{document}
