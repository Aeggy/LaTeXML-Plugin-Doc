\documentclass{llncs}
\usepackage{local,wrapfig}
\pagestyle{plain}%comment this out for the final version
\usepackage{url}
\usepackage{lstom}
%\usepackage[eso-foot,today]{svninfo}
\newtheorem{Rule}{Rule}
%\fulltrue 
\fullfalse

\iffull
\usepackage[show]{ed}
\usepackage[final]{svninfo}
\svnInfo $Id: MKM2009v2.tex 532 2009-04-13 23:09:26Z jhd $
\svnKeyword $HeadURL: https://svn.kwarc.info/repos/kohlhase/tex/papers/mkm/mmlstrict/MKM2009v2.tex $
\else
\newlabel{table4}{{1}{19}}
\newlabel{4218}{{10.1}{19}}
\newlabel{Ctable}{{2}{55}}
\fi
% \usepackage[a4paper=true, linkcolor=blue,
%             citecolor=blue,urlcolor=blue,colorlinks=true,pagecolor=black,
%             breaklinks=true, bookmarksopen=true]{hyperref}
%\bibliographystyle{alpha}
\bibliographystyle{plain}

\title{Unifying Math Ontologies: A tale of two standards\iffull{}
(full version)\fi}
\author{James H.~Davenport\inst{1} \and Michael Kohlhase\inst{2}}
\institute{Department of Computer Science\\
  University of Bath, Bath BA2 7AY, United Kingdom\\
  {\tt J.H.Davenport@bath.ac.uk}
  \and
  School of Engineering \& Science, Jacobs University Bremen\\
  Campus Ring 12,
  D-28759 Bremen, Germany\\
  {\tt m.kohlhase@jacobs-university.de}} 

\begin{document}
\maketitle
\begin{abstract}\noindent
 One of the fundamental and seemingly simple aims of mathematical knowledge management
  (MKM) is to develop and standardize formats that allow to ``represent the meaning of the
  objects of mathematics''.
The open formats OpenMath and MathML address this\iffull{} from a
content markup perspective\fi, but differ subtly in syntax, rigor, and structural
  viewpoints (notably over calculus). To avoid fragmentation and smooth out
  interoperability obstacles, effort is under way to align them into a joint format
  OpenMath/MathML 3. We illustrate the \iffull conceptual and practical \fi issues that come up in such
  an alignment by looking at three main areas: bound variables and conditions, calculus
  (which relates to the previous) and ``lifted'' $n$-ary operators.
\end{abstract}
%\vskip -0.5in
\begin{quotation}\noindent\em
  Whenever anyone says ``you know what I mean'', you can be pretty sure that {\emph{he}}
  does not know what he means, for if he did, he would tell you.  \hfill{\rm{--- H.~Davenport (1907--1969)}\break}
\end{quotation}
%\setcounter{tocdepth}{1}\tableofcontents\newpage
\section{Introduction}

One of the fundamental and seemingly simple aims of mathematical knowledge management
(MKM) is to develop and standardize representation formats that allow one to specify the
meaning of the objects and documents of mathematics. The open formats OpenMath and MathML
address the key sub-problem of representing mathematical objects from a content markup
perspective: mathematical objects are represented as expression trees. As the formats were
%developed by different communities with different goals, they differ subtly in syntax,
% JHD removed different goals, since I want to say that the underlying goal is the same.
developed by different communities, they differ subtly in syntax,
rigor, and structural viewpoints (notably over calculus).
\iffull
This has caused double
developments, interoperability problems, and confusion in developers, system vendors, and
users of mathematical software systems and has considerably weakened the uptake of MKM
methods.\else\fi{}
 The efforts to mitigate the interoperability problem by establishing translations
between the formats have done more to unearth subtle problems than to completely
solve them in the past.

Both efforts shared the goal of representing mathematics ``as it is'', rather
than ``as it ought to be''. A relevant example of the difference is given by
\cite{KamareddineNederpelt2004}, where the original text is 
\begin{equation}\label{FK:orig}
\hbox{The function }\sqrt{|x|} \hbox{ is not differentiable at 0}
\end{equation}
while its formalised equivalent is
\begin{equation}\label{FK:formal}
\lnot(\lambda_{x:\R}(\sqrt{|x|}) \hbox{ is differentiable at 0}).
\end{equation}
The key features are the typing of $x$ as being in $\R$, and the
conversion of $\sqrt{|x|}$ from an expression to a function. Both OpenMath and
MathML, the latter explicitly as one of its design goals
\begin{quote}
``Encode mathematical material suitable for teaching and scientific
communication at all levels'' \cite[1.2.4]{WorldWideWebConsortium2003b},
\end{quote}
wish to encode both styles, or levels of formality, of mathematics. This is a
particular problem for calculus.
MathML and OpenMath have rather different views of calculus, which goes back to a
fundamental duality in mathematics. These views can, simplistically, be regarded as:
\begin{itemize}
\item what one learned in calculus/analysis about {\it functions\/}, which
{\emph{we}} will write as $\Ded$: the
  ``differentiation of $\epsilon$--$\delta$ analysis'' (similarly $\fracDed{}{x}$, and its
  inverse $\Ied$);
\item what is taught in differential algebra about ({\it expressions\/} in)
differential fields, which {\emph{we}} will write as $\DDA$: the
  ``differentiation of differential algebra'' (similarly $\fracDDA{}{x}$, and
its inverse $\IDA$).
\end{itemize}
(\ref{FK:formal}) is unashamedly the former, while (\ref{FK:orig}) talks about
a function, but actually gives an expression. This duality shows up whenever
one talks about variables: while
\begin{equation}\label{JHD:orig}
2x \ne 2y,
\end{equation}
\begin{equation}\label{JHD:formal}
(\lambda x. 2x) \hbox{ =, or at least $\equiv_\alpha$, } (\lambda y. 2y).
\end{equation}
So does 
\begin{equation}\label{JHD:issue}
\frac{{\rm d} x^2}{{\rm d}x} = \frac{{\rm d}y^2}{{\rm d}y} ?
\end{equation}
The variables are clearly free in (\ref{JHD:orig}) and bound in
(\ref{JHD:formal}). Any system which attempted to {\it force\/} either
interpretation on (\ref{JHD:issue}) would not meet the goal stated above.

In this paper we report on an ongoing effort of the W3C MathML Working group and members
of the OpenMath Society to merge the ontologies\footnote{Here we use the word ``ontology''
  in its general, philosophical meaning as the study of the existence of objects, their
  categories and relations amongst each other, and not in the Semantic Web usage, where it
  is restricted to formal systems with tractable inference properties (description
  logics). Note furthermore that we are speaking as much about a ``meta-ontology'' of
  mathematical representation concepts as about ``domain ontologies'' that describe the
% mathematical concepts themselves. Now that we have made this distinction we will
  mathematical concepts themselves. Having made this distinction, we will
  conveniently gloss over it in the rest of the paper.} on which the OpenMath and MathML formats
are based and thus align the formats, so that they only differ in their concrete XML
encodings. This task proves to be harder than might initially be expected.  We explain
why, motivated by a study of four areas (which in fact turn out to be inter-related):
\begin{enumerate}
\item constructions with bound variables;
\item the {\element{<condition>}} element of MathML;
\item the different handling of calculus-related operations in the two;
\item the ``lifting'' of $n$-ary operators, such as $+$ to $\sum$.
\end{enumerate}

\iffull A shorter version of this paper has been submitted to Mathematical Knowledge Management 2009.
\else This paper is a short version of~\cite{DavenportKohlhase2009b}, which contains the details
of the constructions.\fi{} OpenMath-specific details of the proposals are in
\cite{DavenportKohlhase2009c,DavenportKohlhase2009d}.

\section{OpenMath and MathML}\label{sec:formatsnhistory}
We will now recap the two formats focusing on their provenance and representational
assumptions and then sketch the measures taken for aligning the languages. Sections 3, 4,
6, and 7 will detail the problem areas identified above. The first two leading to an
extension proposal for OpenMath Objects and strict content MathML in Section 5, which is
evaluated in the latter two. Section 7 concludes the paper.


\subsection{MathML}\label{sec:mathML}

MathML is an XML-based language for capturing mathematical the presentation, structure and
content of mathematical formulae, so that they can be served, received, and processed on
the World Wide Web. Thus the goal of MathML is to provide a similar functionality that
HTML has for text. The present recommended version of MathML format is MathML 2 (second
edition) of October 2003~\cite{WorldWideWebConsortium2003b}. MathML 1 had been recommended
in April 1998 and revised as MathML 1.01 in July 1999.

MathML, starting from version 1.0, had a split into {\textbf{presentation MathML}},
describing what mathematics ``looked like''\footnote{Which could include ``sounded like''
  (for aural rendering) or ``felt like'' (e.g. for Braille), and MathML included a range
  of symbols such as {\texttt{\&InvisibleTimes;}} to help with this task.}, and
{\textbf{content MathML}}, describing what it ``meant''. In this paper we will concentrate
on content MathML, since the role of presentation MathML as a high-level presentation
format for Math on the Web is (largely) uncontested. MathML's Content markup has ambitious
goals:
\begin{quotation}\noindent\em
  The intent of the content markup in the Mathematical Markup Language is to provide an
  explicit encoding of the {\emph{underlying mathematical structure}} of an expression,
  rather than any particular rendering for the expression.  \hfill{\hbox{\rm\cite[section
      4.1.1]{WorldWideWebConsortium2003b}}}
\end{quotation}
This mandate is met in MathML 1/2 by representing mathematical formulae as XML expression
trees that follow the applicative structure of operators and their arguments: function
application is represented by the {\element{apply}} elements where the first child is
interpreted as the operator and the remaining children as their arguments. MathML 2
supplies about 90 built-in elements for mathematical operators, and the
{\element{csymbol}} extension mechanism described later. The language has a fairly limited
vision of what might be in ``content'':
\begin{quotation}\noindent\em
  % It would be an enormous job to systematically codify most of mathematics --- a task
  % which can never be complete. Instead, MathML makes explicit a relatively small number
  % of commonplace mathematical constructs, chosen carefully to be sufficient in a large
  % number of applications. In addition, it provides a mechanism for associating semantics
  % with new notational constructs. In this way, mathematical concepts that are not in the
  % base collection of tags can still be encoded (see section 4.2.6).
  The base set of content elements are chosen to be adequate for simple coding of most of
  the formulas used from kindergarten to the end of high school in the United States, and
  probably beyond through the first two years of college, that is up to A-Level or
  Baccalaureate level in Europe.\hfill{\hbox{\cite[4.1.2]{WorldWideWebConsortium2003b}}}
\end{quotation}
This is often referred to as the {\textbf{K-14 fragment}} of mathematics, by
analogy with some countries use of ``K--12'' for the range of school mathematics. Since Version 2,
MathML does have an extension mechanism via the {\element{csymbol}} elements and their
{\texttt{definitionURL}} attributes, but this was rarely used except to achieve some form of OpenMath interoperability, or for proprietary extensions (e.g. Maple).

MathML tries to cater to the prevalent representational practices of mathematicians, and
provides a good dozen structural XML elements for special constructions, e.g. set,
interval and matrix constructors, and allows to ``lift'' various associative operators to
``big operators'' acting on sets and sequences simply by associating them by bound
variables and possibly qualifier elements to specify the domain of application.

The MathML approach to specifying the ``meaning'' of expression trees largely follows a
``you know what I mean'' approach that alludes to a perceived consensus among mathematical
practitioners on the K-14 fragment. The meaning of a construction is alluded to via
examples rather than defined rigorously, intending to be ``formal enough'' to cover
``{\emph{a large number of applications}}''~\cite[4.1.2]{WorldWideWebConsortium2003b},
while remaining flexible enough not to preclude too many.

\subsection{OpenMath}\label{sec:openmath}

OpenMath \cite{OpenMath2004a} is a standard for the representation and communication of mathematical objects.
It has similar goals to content MathML and focuses on encoding the meaning of objects
rather than visual representations to allow the free exchange of mathematical objects
between software systems and human beings. OpenMath has been developed in a long series of
workshops and (mostly European) research projects that began in 1993 and continues through
today. The OpenMath 1.0 and 2.0 Standards were released by the OpenMath Society in
February 2000 and June 2004. OpenMath 1 fixed the basic language architecture, while
OpenMath2 brought better XML integration, structure sharing and separated the notion of OpenMath Content Dictionaries from their encoding.

Like content MathML, OpenMath represents mathematical formulae as expression trees, but
concentrates on an extensible framework built on a minimal structural core language with a
well-defined extension mechanism. Where MathML supplies more than a dozen elements for
special constructions, OpenMath only supplies concepts for function application
({\element{OMA}}), binding constructions ({\element{OMBIND}}; MathML 2 lacks an analogous
element and simply uses {\element{apply}} with bound variables, hence the
(inferred) Rule \ref{rule:MML} below). Where MathML provides
close to 100 elements for the K-14 fragment, OpenMath gets by with only an {\element{OMS}}
element that identifies symbols by pointing to declarations in an open-ended set of
Content Dictionaries (see below). 

An OpenMath Content Dictionary (CD) is a document that declares names (OpenMath
``symbols'') for basic mathematical concepts and objects. CDs act as the unique points of
reference for OpenMath symbols (and their encodings the {\element{OMS}} elements) and
thus supply a notion of context that situates and disambiguates OpenMath expression
trees. To maximize modularity and reuse, a CD typically contains a relatively small
collection of definitions for closely related concepts. The OpenMath Society maintains a
large set of public CDs, including CDs for all pre-defined symbols in MathML 2. There is a
process for contributing privately developed CDs to the OpenMath Society repository to
facilitate discovery and reuse. OpenMath does not require CDs be publicly available,
though in most situations the goals of semantic markup will be best served by referencing
public CDs available to all user agents.
 
The fundamental difference to MathML is in terms of establishing meaning for mathematical
objects. Rather than appealing to mathematical intuition, OpenMath defines a free algebra
$\mathcal{O}$ of ``OpenMath Objects'' which acts as (initial) model for encodings of
mathematical formulae. OpenMath Objects are essentially labeled trees, with
$\alpha$-conversion for binding structures and Currying for nested semantic
annotations. Note that since $\mathcal{O}$ is initial it is essentially unique and
identifies (in the sense of ``declares to be the same'') fewer objects than any other
model. As a consequence two mathematical objects must be identical, if their OpenMath
representations are, but may coincide, even if their representations are different. The
OpenMath standard therefore considers OpenMath objects as primary citizens and views the
``OpenMath XML encoding'' as just an incidental design choice for an XML-based markup
language. In fact OpenMath specifies another encoding: the ``binary encoding'' designed to
be more space efficient at the cost of being less human-readable.
``OpenMath XML encoding'' as just an incidental design choice for an 
XML-based\footnote{{\openmath} also has a more space-efficient binary encoding.} markup language.
%In fact OpenMath specifies another encoding: the ``binary encoding'' designed to
%be more space efficient at the cost of being less human-readable.

The initial algebra semantics of {\openmath} objects is intentionally weak to make the
% JHD@MK: You had 'formal', but I think you mean 'format'
{\openmath} format ontologically unconstrained and thus universally applicable. It
basically represents the accepted design choice of representing objects as formulae. Any
further (meaning-giving) properties of an object $o$ are relegated to the content
dictionaries referenced in $o$, where they can be specified formally (``Formal
Mathematical Properties'' as {\texttt{FMP}} elements which are themselves {\openmath} objects) or
informally (``Commented Mathematical Properties'' as {\texttt{CMP}} elements containing
text). Thus the precision of {\openmath} as a representation language can be adapted by
allowing CDs to range from fully formal (by providing CDs based on some logical system)
to fully informal (where CDs are essentially empty). While this can be seen as a failure
of {\openmath} to supply semantics (``{\openmath} is only syntax''), we see it as being as
flexible as mathematical vernacular that gives the same freedom.

% New paragraph by JHD
The question ``does this OpenMath object $o$ have formal semantics?'' does not have an unambiguous answer. Rather, $o$ has a meaning {\em for the system $S$\/} if each OpenMath symbol in $o$ {\bf either}:
\begin{enumerate}
\item is built into the OpenMath $\leftrightarrow S$ phrasebook {\bf or}
\item has enough semantics deducible in $S$ from the {\texttt{FMP}}s (which
may be a recursive process).
\end{enumerate}
Here $S$ might be either a software system, or a logical system such as ZF.
\subsection{The OpenMath/MathML 3 Alignment Process}\label{sec:alignment}

Most of these differences between MathML and OpenMath can be traced to the different
communities who developed these representation formats. MathML came out of the ``HTML Math
Module'', an attempt to develop {\LaTeX}-quality presentation of mathematical on the Web,
something sorely missing from the otherwise very successful HTML. The guiding goal for OpenMath on
the other hand was to develop an open interchange format among computer algebra systems,
which resulted in a much stronger emphasis on the meaning of objects to make the exchange
of sub-problems safe.

Even though interoperability between OpenMath and and MathML was always a strong
desideratum for both communities, the two representation formats evolved independently and
in line with the fundamental assumptions outlined in the two previous
sections. Interoperability was attempted from the MathML side by integrating the
{\element{csymbol}} element in MathML 2 and specifying parallel markup, i.e. allowing
OpenMath representations to be embedded into MathML with fine-grained
cross-referencing. The OpenMath Society developed CDs with analogues for ``all predefined
operators'' and specified the correspondence between expression trees
in~\cite{CarDAvDew:cbmao01}. Although 30 pages long, the fact that this document is still
incomplete may serve as an indication that the problem is not trivial. As we will see
below, mapping the MathML operators is not enough in the presence of different structural
elements in the formats.

In June 2006 the W3C rechartered the MathML Working Group to produce a MathML 3
Recommendation, and the group identified the lack of regularity and specified meaning as a
problem to be remedied in the charter period. The group decided to establish meaning for
content MathML expressions based on OpenMath objects without losing backwards
compatibility to content MathML 2. In the end, content MathML was extended to incorporate
concepts like binding structures and full semantic annotations from OpenMath and a
structurally regular subset of the extended content MathML was identified that is
isomorphic to OpenMath objects. This subset is called {\textbf{strict content MathML}} to
contrast it to full content MathML that was seen to strike a more pragmatic balance
between regularity and human readability. Full content MathML borrows the semantics from
strict MathML by a mapping specified in the MathML 3 specification that defines the meaning
of non-strict ({\textbf{pragmatic}}) MathML expressions in terms of strict MathML
equivalents. The division into two sub-languages serves a very important goal in
standardization: to clarify and codify best (engineering) practices without breaking
legitimate uses in legacy documents. In the current third version of MathML, the latter is
a primary concern. 

In June 2007, the OpenMath society chartered a group of members which includes the authors
of this paper to work on version 3 of the OpenMath standard which would recognize content
MathML 3 as a legitimate OpenMath encoding, to help define the pragmatic to strict mapping
MathML, and to provide the necessary CDs, which would be endorsed by the W3C Math Group
and the OpenMath Society. The discussions and the resulting CDs are online in the SWiM
Wiki~\cite{LangeGonzalez:SWiM-Sentido08}~\cite{OpenMathWiki}

Subsequent sections describe the problem areas that came up during the work and
needed to be circumnavigated.


\section{Set Constructors in MathML}\label{sec:setconstructors}

With the K-14 scope discussed above, MathML found that it needed more sophisticated
concepts, such as bound variables, to express the concepts that are manipulated
{\emph{informally}} at that level. One conspicuous example from K-14 is that of
sets constructed by rules \cite[4.2.1.8]{WorldWideWebConsortium2003b}.
%\strut\hfill{\hbox{\rm\cite[4.2.1.8]{WorldWideWebConsortium2003b}}}
\begin{quotation}\noindent\em 
  A typical use of a qualifier is to identify a bound variable through use of the bvar
  element [\ldots] The {\element{condition}} element is used to place conditions on bound
  variables in other expressions. This allows MathML to define sets by rule, rather than
  enumeration, for example. The following markup, for instance, encodes the
set $\{x \mid x <
  1\}$:
\begin{lstlisting}[language=MathML2]
<set>
  <bvar><ci>x</ci></bvar>
  <condition>
    <apply><lt/><ci>x</ci><cn>1</cn></apply>
  </condition>
</set>
\end{lstlisting}
\end{quotation}
\par\noindent
Here (with the benefit of a great deal of hindsight, it should be pointed out) we can see
the start of the problem. What would we have meant if we had changed the
second\footnote{Changing both of them would have been an $\alpha$-conversion\iffull{} \cite[Definition 2.1.11]{Barendregt1984}, though MathML does not analyse the concept\fi.}
{\element{x}} to {\element{y}}? We would, of course, have written the MathML equivalent of
$\{x \mid y < 1\}$, and the MathML would be as eccentric as that set of symbols.
%In other words, the variable in {\element{bvar}} is meant to bind the variable in the condition as well as the variables in the rest of the construction.
We therefore deduce the following (undocumented) rule, which corresponds to
OpenMath's formal rules for {\element{OMBIND}}.
\begin{Rule}[MathML]\label{rule:MML}
  Variables in {\element{bvar}} constructions `bind' the corresponding variable occurrences in
  the scope of the parent of the {\element{bvar}}. However, the variable may
(e.g. $\forall$) or may not (e.g. $\frac{\rm d}{{\rm d}x}$) be bound in the
%$\lambda$-calculus sense.
sense of $\alpha$-convertibility.
\end{Rule}
\par
Here the first problem of interpreting pragmatic MathML elements raises its ugly head. In
OpenMath, we can represent the set\footnote{Note that the OpenMath CDs
% JHD: moved the footnote to avoid "footnote marker on formula"
  require a larger set to be specified (to avoid Russell's paradox). It would not be a
  problem to provide a CD for what is often called ``na\"\i{}ve set theory'' that leaves out
  this safety device. However, such a system would have the same difficulties that the MathML
  above has: do we mean $(-\infty,1)$ or $[0,1)$, and is this a subset of $\Z$ or $\R$?}
$\{x\in\R | x < 1\}$ by the representation
\begin{lstlisting}
<OMOBJ version="2.0">
  <OMA>
    <OMS cd="set1" name="suchthat"/>
    <OMS cd="setname1" name="R"/>
    <OMBIND>
      <OMS cd="fns1" name="lambda"/>
      <OMBVAR><OMV name="x"/></OMBVAR>
      <OMA>
        <OMS cd="relation1" name="lt"/>
        <OMV name="x"/>
        <OMI> 1 </OMI>
      </OMA>
    </OMBIND>
  </OMA>
</OMOBJ>
\end{lstlisting}
This makes use of a binding construction (\element{OMBIND}) with a $\lambda$ operator that
constructs functions\footnote{Here we also make use of the duality between sets and
  Boolean-valued functions that are their characteristic functions} from an expression
with a bound variable. This kind of construction is standard in logical systems and
$\lambda$-calculus, for which is is motivated as follows in a standard introductory
textbook (our emphasis):
\begin{quotation}\noindent%\em JHD: took out emphasis since I want to
% emphasise part
  To motivate the $\lambda$-notation, consider the everyday mathematical expression
  `$x-y$'. This can be thought of as defining either a function $f$ of $x$ or $g$ of $y$
  \dots{} And there is need for a notation that gives $f$ and $g$ different names in some
  systematic way. {\em In practice\/} mathematicians usually avoid this need by various `ad hoc'
  special notations, but these can get very clumsy when higher-order functions are
  involved.\hfill\hbox{\rm\cite[p. 1]{HindleySeldin2008}}
\end{quotation}
To achieve interoperability with OpenMath objects, MathML 3 introduces the {\element{bind}}
element in analogy to the OpenMath {\element{OMBIND}}. It could be argued that the
``K--14'' brief of MathML rules out higher-order functions, but in the example above we
can see here the need, in a purely first-order case, to resort to ``well, you know what I
mean'' without it. Extending MathML 3 with a {\element{bind}} element that encodes an
{\emph{OpenMath binding object}} takes the guessing of Rule 1 out of MathML and makes the
meaning unambiguous. The MathML 3 specification does however need to specify the strict
content MathML equivalent for the MathML 2 example above in order to give it an OpenMath
Object semantics.\iffull\ednote{show this here!}\fi


\section{Calculus Issues}
MathML and OpenMath have rather different views of calculus, which goes back
to the fundamental duality in mathematics mentioned earlier. 
%These can, simplistically, be regarded as:
%\begin{itemize}
%\item what one learned in calculus, which {\emph{we}} will write as $\Ded$: the
  %``differentiation of $\epsilon$--$\delta$ analysis''. Also $\fracDed{}{x}$, and its
  %inverse $\Ied$;
%\item what is taught in differential algebra, which {\emph{we}} will write as $\DDA$: the
  %``differentiation of differential algebra''. Also $\fracDDA{}{x}$, and its inverse
  %$\IDA$.
%\end{itemize}

\begin{figure}\lstset{frame=none,numbers=none}\centering
\lstset{aboveskip=-.7em,belowskip=-1.2em}
\begin{tabular}{|p{4.5cm}|p{7cm}|}\hline
\begin{lstlisting}[language=MathML2]
<apply>
  <diff/>
         
         
  <bvar><ci>x</ci></bvar>
  <apply>                
    <power/>             
    <ci>x</ci>           
    <cn>2</cn>           
  </apply>               
                         
</apply>                 
\end{lstlisting}&
\begin{lstlisting}
<OMA>
  <OMS cd="calculus1" "name="diff"/>
  <OMBIND>
    <OMS cd="fns1" name="lambda"/>
    <OMBVAR><OMV name="x"/></OMBVAR>
    <OMA>
      <OMS cd="arith1" name="power"/>
      <OMV name="x"/>
      <OMI>2</OMI>
    </OMA>
  </OMBIND>
</OMA>
\end{lstlisting}\\\hline
\end{tabular}
\caption{MathML 2 and OpenMath2 differentiation compared}\label{lst:deriv}
\end{figure}

\iffull\subsection{Differentiation}\fi

{\emph{Roughly speaking}}, the MathML encoding corresponds more closely to $\Ded$ and the
OpenMath one to $\DDA$. If we were to look at the derivative of $x^2$ as in
Figure~\ref{lst:deriv}, we might be tempted to see only trivial syntactic differences: in
the MathML encoding we see a differential operator that {\emph{constructs a function from
    an expression with a bound variable}}\footnote{With the insights from the last
  section, MathML 3 would probably use a {\element{bind}} element, emphasizing the role of
  the differentiation operator as a function constructor.} declared by a {\element{bvar}}
element.  The {\openmath} encoding sees the differential operator as a functional that
transforms one function (the square function) into another (its derivative).  It is
possible to do this without any variables, as in $\sin'=\cos$.  Given the history of the
two standards, this difference of encoding is not surprising, since $\DDA$ is what
computer algebra systems do (and what humans do, most of the time, even while interpreting
the symbols as $\Ded$), whereas human beings generally {\em think\/} they are doing $\Ded$ and
communicate mathematics that way.

For partial differentiation we see the same general picture, but the concrete
representations drift further apart: For $\frac{d^{m+n}}{dx^m dy^n}f(x,y)$, MathML would
use
\begin{lstlisting}[language=MathML2]
<apply>
  <partialdiff/>
  <bvar><ci>x</ci><degree><ci>m</ci></degree></bvar>
  <bvar><ci>y</ci><degree><ci>n</ci></degree></bvar>
  <degree><apply><plus/><ci>m</ci><ci>n</ci></apply></degree>
  <apply><ci type="function">f</ci><ci>x</ci><ci>y</ci></apply>
</apply>
\end{lstlisting}
using {\element{degree}} qualifiers inside the {\element{bvar}} elements for the orders of
partial differentiations and a {\element{degree}} qualifier outside for the total
degree. The following representation is proposed in~\cite{CarDAvDew:cbmao01}:
\begin{lstlisting}[language=OpenMath]
<OMA>
  <OMS cd="calculus1" name="partialdiff"/>
  <OMA>
    <OMS cd="list1" name="list">
    <OMV name="m"/>
    <OMV name="n"/>
  </OMA>
  <OMBIND>
    <OMS cd="fns1" name="lambda"/>
    <OMBVAR><OMV name="x"/><OMV name="y"/></OMBVAR>
    <OMA><OMV name="f"><OMV name="x"/><OMV name="y"/></OMA>
  </OMBIND>
</OMA>
\end{lstlisting}
For the problems caused by wishing to represent $\frac{d^k}{dx^m dy^n}f(x,y)$, see
\cite{Kohlhase2008} and the proposed solution in \cite{DavenportKohlhase2009d}.

\iffull\subsection{Integration}\fi

Integration is even more problematic than differentiation. MathML interprets integration
as an operator on expressions in one bound variable and presents as paradigmatic examples
the three expressions below, which differ in which ways the bound variables are handled.
\begin{center}
\lstset{frame=none,numbers=none,lineskip=-.7ex,aboveskip=-.5em,belowskip=-1em,language=MathML2}
% JHD first was 5.9, others 4.4
\begin{tabular}{|p{3.3cm}|p{3.5cm}|p{3.8cm}|}\hline
  a: $\int_0^af(x) dx$ & b: $\int_{x\in D}f(x) dx$ & c: $\int_Df(x)dx$\\\hline
\begin{lstlisting}
<apply>
  <int/>
  <bvar>
    <ci>x</ci>
  </bvar>
  <lowlimit>
    <cn>0</cn>
  </lowlimit>
  <uplimit>
    <ci>a</ci>
  </uplimit>
  <apply><ci>f</ci>
    <ci>x</ci>
  </apply>
</apply>
\end{lstlisting}
&
\begin{lstlisting}[language=MathML2]
<apply>
  <int/>
  <bvar>
    <ci>x</ci>
  </bvar>
  <condition>
    <apply><in/>
      <ci>x</ci>
      <ci>D</ci>
    </apply>
  </condition>
  <apply><ci>f</ci>
    <ci>x</ci>
  </apply>
</apply>
\end{lstlisting}
&
\begin{lstlisting}[language=MathML2]
<apply>
  <int/>
  <bvar>
    <ci>x</ci>
  </bvar>
  <domainofapplication>
    <ci>D</ci>
  </domainofapplication>
  <apply><ci>f</ci>
    <ci>x</ci>
  </apply>
</apply>
\end{lstlisting}
\\\hline
\end{tabular}
\end{center}
OpenMath can model usages (a) and (c) easily enough, via its {\element{defint}} operator:
in fact usage (a) is modeled on the lines of (c), as $\int_{[0,a]}f(x) dx$, which means
that we need to give an eccentric\footnote{Along the lines of ``the set $[b,a]$ is the
  same as $[a,b]$ except that, where it appears as a range of integration, we should
  negate the value of the integral''! \cite{Kohlhase2008}. It is possible to regard
  `backwards integration' as an ``idiom'' \iffull{in the sense of
\cite{LuoCallaghan1999}}\fi and
  (\ref{eq:backint}) as the explanation of that idiom, but this seems circular.} meaning
to `backwards' intervals in order to encode the traditional mathematical statement
\begin{equation}\label{eq:backint}
\int_a^bf(x) dx=-\int_b^af(x) dx.
\end{equation}
A more logical view is to regard the two notations as different, and define
$\Ied_{[a,b]}$ (via limits of Riemann sums, or whatever other definition is
appropriate), and then
\begin{equation}\label{eq:backinted}
\Ied_a^bf=%\begin{cases}
\left\{
\begin{array}{lr}
\Ied_{[a,b]}f & a\le b\\
-\Ied_{[b,a]}f & a>b\\
\end{array}\right.,
%\end{cases} 
\end{equation}
whereas 
\begin{equation}\label{eq:backintda} 
\IDA_a^bf = \left(\IDA f\right)(b)-\left(\IDA f\right)(a)
\end{equation}
by definition.

Usage (b) might not worry us too much at first, since it is apparently only a variant of
(c). The challenge comes when we move to multidimensional integration (in the $\Ied$
sense).  \cite[p. 189]{BorweinErdelyi1995} has a real integral over a curve in the complex
plane,
\begin{equation}\label{eq:bounds2} 
\frac1{2\pi}\int_{|t|=R}\left|\frac{f(t)}{t^{n+1}}\right| |dt|
\end{equation}
whereas \cite[p. 413, exercise 4, slightly recast]{Apostol1967} has an integral
where we clearly want to connect the variables in the integrand to the variables defining
the set:
\begin{equation}\label{eq:bounds1} 
\mathop{\int\int\int}_{\left\{\frac{x^2}{a^2}+\frac{y^2}{b^2}+\frac{z^2}{c^2}\le1\right\}}
\left(\frac{x^2}{a^2}+\frac{y^2}{b^2}+\frac{z^2}{c^2}\right)dxdydz
\end{equation}

\section{A Radical Proposal: Enhanced Binding Operators}\label{sec:proposal}

%\begin{newpart}{MK@JHD, this is new, in response to David's mail; see if you can support this.}
  The multiple points of view in the $\epsilon\delta$ vs. DA discussion 
can be seen in other situations, 
  as witnessed by the difference between the OpenMath and MathML representations of
  the set $\{x | x < 1\}$ above. There seem to be two styles of thinking about
  mathematical objects. The first one --- we will call it the {\textbf{first-order style}}
  --- manifests itself as the $\epsilon\delta$-style in calculus. This style avoids
  passing around functions and sets as arguments to operators and uses expressions with
  bound variables instead. The second style --- which we will call the
  {\textbf{higher-order style}} --- allows functions and sets as arguments and relies
  heavily on this feature for conceptual clarity. It can be argued that the higher-order
  style is more modern\footnote{It has gained traction in the second half of the 20$^{th}$
    century with the advent of category theory in Math and type theories in Logic}, but
  arguably the first-order style still permeates much of mathematical practice. And if we
  take the use of mathematics in the Sciences and Engineering into account probably
  accounts for the vast majority of mathematical communication. Therefore we argue that
  both representational styles must be supported by MathML and OpenMath (and strict
  content MathML)
%\end{newpart}


Examples like (\ref{eq:bounds2}) and (\ref{eq:bounds1}) show that the binding objects in
OpenMath are too weak representationally to accomodate the first-order style of
representation faithfully, and so force the reader into a higher-order style: we want the triple
integration operator in (\ref{eq:bounds1}) to range over a restricted domain of
integration, and we want to give this domain as an {\emph{expression over the integration
%  variables}}\footnote{Note that the objection that the original formulation in
%  \cite{Apostol1967}, which was ``$\mathop{\int\int\int}_S\ldots$ where $S=\{\cdots\}$'',
% transcends the scope of both MathML and OpenMath, which restrict themselves to
variables}}\footnote{The original formulation in \cite{Apostol1967}, which was 
``$\mathop{\int\int\int}_S\ldots$ where $S=\{\cdots\}$'', 
  transcends the scope of both MathML and OpenMath, which restrict themselves to
  mathematical formulae. In fact MathML 2 had limited support for inter-formula effects
  with the {\element{declare}} element, but deprecates this element in MathML 3 since it
  cannot be defined on an intra-formula level. Thus the (important) issue of connecting
  bindings between different formula must be relegated to representation formats that
  transcend individual formulae, such as the OMDoc format~\cite{Kohlhase:omdoc1.2}.}, at
least in $\epsilon\delta$ variant of integration. Moreover, given the discussion in
Section~\ref{sec:setconstructors} we need these variables to participate in
$\alpha$-conversion.
% This is impossible to do in OpenMath and strict content MathML at the moment without
% betraying the intuition of working with expressions in bound variables. Let us look at
% the two variants we could attack the representation problem:
How might we encode this in OpenMath? Figure~\ref{fig:alternatives} shows 4
alternatives\footnote{We use boxed formulae as placeholders for their
  (straightforward but lengthy) OpenMath2 encodings.}:
\begin{enumerate}
\item{\textbf{In the binder}}\label{OM:bind} We can interpret
  $\scriptstyle\mathop{\int\int\int}_{\scriptscriptstyle\left\{\frac{x^2}{a^2}+\frac{y^2}{b^2}+\frac{z^2}{c^2}\le1\right\}}$
  as a complex binding operator, as in {\element{forallin}} and try to use that in
  a binding object. But this runs foul of the OpenMath2 dictum that the binding operator
  is not subject to $\alpha$-conversion by its own variables; so this avenue is closed.
\item{\textbf{In the body}}\label{OM:multi} On the other hand we can interpret the domain
  restriction as part of the binding object, and represent (\ref{eq:bounds1}) as (2) in
  Figure~\ref{fig:alternatives}. But this is impossible in OpenMath2, since only
  one OpenMath object after the {\element{OMBVAR}} element is allowed.
\item{\textbf{In the body (2)}}\label{OM:wrapped} We can solve this problem by inventing a
  mathematically meaningless ``gluing'' operator
\item{\textbf{separately}}\label{OM:sep} It is possible to represent an integration
  formula in OpenMath2 that is supposedly equivalent mathematically to (\ref{eq:bounds1})
  using the Differential Algebra approach: but this is, from the $\epsilon\delta$ point of
  view, totally unnatural, since it is $\alpha$-equivalent to the expression
in Figure \ref{fig:alpha}
  which is unreadable for a human, and also destroys commonality of formulae.
\end{enumerate}

\begin{figure}\centering\vspace*{-.8cm}
\lstset{mathescape,numbers=none,frame=none,aboveskip=-.7em,belowskip=-1.2em}
  \begin{tabular}{|l|p{5.2cm}||l|p{5.2cm}|}\hline
    1.&
\begin{lstlisting}
<OMBIND>
  <OMA>
    <OMS cd="calculus_new" 
         name="tripleintcond"/>
    $\psom{\frac{x^2}{a^2}+\frac{y^2}{b^2}+\frac{z^2}{c^2}\le1}$
  </OMA>
  <OMBVAR>$\psom{x,y,z}$</OMBVAR>
  $\psom{\frac{x^2}{a^2}+\frac{y^2}{b^2}+\frac{z^2}{c^2}}$
</OMBIND>
\end{lstlisting}
& 2. & 
\begin{lstlisting}
<OMBIND>
 <OMS cd="calculus_new" 
      name="tripleintcond"/>
 <OMBVAR>$\psom{x,y,z}$</OMBVAR>
  $\psom{\frac{x^2}{a^2}+\frac{y^2}{b^2}+\frac{z^2}{c^2}\le1}$
  $\psom{\frac{x^2}{a^2}+\frac{y^2}{b^2}+\frac{z^2}{c^2}}$
</OMBIND>
\end{lstlisting}
\\\hline
3. & 
\begin{lstlisting}
<OMBIND>
 <OMS cd="calculus_new" 
      name="tripleintcond"/>
 <OMBVAR>$\psom{x,y,z}$</OMBVAR>
 <OMA>
   <OMS cd="calculus_new" 
        name="tripleint_inner"/>
   $\psom{\frac{x^2}{a^2}+\frac{y^2}{b^2}+\frac{z^2}{c^2}\le1}$
   $\psom{\frac{x^2}{a^2}+\frac{y^2}{b^2}+\frac{z^2}{c^2}}$
 </OMA>
</OMBIND>
\end{lstlisting}
& 4. & 
\begin{lstlisting}
<OMA>
  <OMS cd="calculus_new" 
       name="tripleintcond"/>
  $\psom{\lambda{x,y,z}.\frac{x^2}{a^2}+\frac{y^2}{b^2}+\frac{z^2}{c^2}\le1}$
  $\psom{\lambda{r,s,t}.\frac{r^2}{a^2}+\frac{s^2}{b^2}+\frac{t^2}{c^2}}$
</OMA>
\end{lstlisting}
\\\hline
\end{tabular}
\caption{The Alternatives}\label{fig:alternatives}\vspace*{-2em}
\end{figure}

\begin{wrapfigure}r{5.8cm}%\vspace*{-2em}
%\begin{figure}%\vspace*{-2em}
%\lstset{mathescape,numbers=none,frame=none,aboveskip=-.7em,belowskip=-1.3em}
\lstset{mathescape,numbers=none,frame=none}
%\begin{tabular}{|l|l|}\hline
%5.& 
\begin{lstlisting}[mathescape]
<OMA>
  <OMS cd="calculus_new"
       name="tripleintcond"/>
  $\psom{\lambda{x,y,z}.\frac{x^2}{a^2}+\frac{y^2}{b^2}+\frac{z^2}{c^2}\le1}$
  $\psom{\lambda{z,y,x}.\frac{z^2}{a^2}+\frac{y^2}{b^2}+\frac{x^2}{c^2}}$
</OMA>
\end{lstlisting}%\\\hline
%\end{tabular}%\vspace*{-2.2em}
\caption{$\alpha$-equivalent of 4 above}\label{fig:alpha}
%\end{figure}
\end{wrapfigure}
Solution \ref{OM:bind} makes bound variables have an unusual, to say the least, scope, and
solution \ref{OM:sep} is higher-order style, so we are left with the other two. They
have quite a lot in common, since they both achieve the fundamental goal of making both
the region and the integrand subject to the {\it same\/} binding operation. We can
summarise the points as follows.
\begin{description}
\item[\ref{OM:multi}: \emph{pro}:]Mathematically elegant; fits into both the
XML and binary encodings of OpenMath.
\item[\ref{OM:multi}: \emph{con}:]Requires a change to the abstract
description of the OpenMath standard.
\item[\ref{OM:wrapped}: \emph{pro}:]No change to the OpenMath standard.
\item[\ref{OM:wrapped}: \emph{con}:]Needs a new, mathematically meaningless,
%symbol such as {\element{tripleint\_inner}} for every
%JHD: \url to let the hyphenation algorithm work.
symbol such as \url{tripleint_inner} for each symbol such as
{\element{tripleintcond}}.
\end{description}
Option \ref{OM:multi} is our preferred route, and the rest of this paper assumes that, but
the changes to adopt option \ref{OM:wrapped} should be obvious. The changes to the
OpenMath standard to adopt option \ref{OM:multi} are in the Appendix\iffull\else{} of the
full paper \cite{DavenportKohlhase2009b}\fi.

\section{Conditions in MathML}

Our proposal above still leaves us with the problem to figure out the meaning of the
{\element{condition}} from the examples and to specify their meaning in terms of OpenMath3
objects.  MathML 2 introduces 23 examples of its usage, described in Table \tref{table4},
and a further 31 in Appendix C, described in Table \tref{Ctable}. These can be roughly
categorised as follows (where $a+b$ means ``$a$ in Chapter 4 and $b$ in Appendix C'').
\begin{description}
\item[5+14]are used to encode $\exists n\in \N$ or $\forall n\in \N$ (or
equivalents). Strictly speaking, these usages are not necessary, because of
the equivalences below.
\begin{eqnarray}
\exists v\in S\quad p(v)&\Leftrightarrow \exists v\quad (v\in S)\land p(v)\\
\forall v\in S\quad p(v)&\Leftrightarrow \forall v\quad (v\in S)\Rightarrow p(v)
\end{eqnarray}
However, in practice, it would be better to have a convenient shorthand for these, hence
the proposal in \cite{DavenportKohlhase2009c} for OpenMath symbols
{\element{existsin}} and {\element{forallin}}, which are constructors for
complex binding operators that include restricting the domain of quantification.
\item[6+4]can be replaced by the OpenMath {\element{suchthat}} 
construct\iffull : See section
  \ref{4218}\else{} \cite[\ref{4218}]{DavenportKohlhase2009b}\fi.
\item[2+2]are solved by the use of {\element{map}} in OpenMath.
\end{description}

\begin{wrapfigure}r{8cm}\centering\vspace*{-2em}
\lstset{language=MathML2,mathescape,frame=none,numbers=none}
\lstset{aboveskip=-.7em,belowskip=-1.2em}
\begin{tabular}{|p{4.3cm}|p{3cm}|}\hline
  Pragmatic MathML & Strict MathML\\\hline\hline
\begin{lstlisting}
<apply> $W$
  <bvar>$X$</bvar>
  <condition>$Y$</condition>
  $Z$
<apply>
\end{lstlisting}
& 
\begin{lstlisting}[language=MathML2]
<bind> $W'$
  <bvar>$X$</bvar>
  $Z$
  $Y$
<bind>
\end{lstlisting}\\\hline
\end{tabular}
\caption{Translating MathML with {\element{condition}}}\label{fig:conditiontrans}\vspace*{-2em}
\end{wrapfigure}
So we see that for all concrete operators, we have a natural strict content
MathML/OpenMath equivalent. In the other cases we use the translation in
Figure~\ref{fig:conditiontrans} afforded by OpenMath/strict MathML extended according to
our proposal. Here $W$ is a binding operator and $X$ stands for any number of variables in
the {\element{bvar}} construct and $Y$, $Z$ are arbitrary MathML expressions. Since we
have treated all concrete operators, $W$ must be either a {\element{ci}}, {\element{cn}},
a complex MathML expression, or a {\element{csymbol}} element. We believe the
first two cases have not been used, since there is no plausible way to give them meaning;
we propose to deprecate such usages in MathML 3. In contrast to that, the
{\element{csymbol}} case is an eminently legitimate use, and therefore have to provide a
$W'$ in the rule above. But in MathML 2, a {\element{csymbol}} element only has a
discernible meaning, if it carries a {\texttt{definitionURL}} attribute that points to a
description $D$ of the symbols' meaning, which will specify the meaning of the expression
in terms of $X$, $Y$ and $Z$. This description can be counted as (or turned into) a CD
$D'$ that declares a binary binding operator that can be referenced by a
{\element{csymbol}} element $W'$ which points to this declaration. Note that if $D$
described a usage of the operator $W$ without a {\element{condition}} qualifier, then $D'$
must also declare the unary binding operator $W$; this must be different from $W'$, since
OpenMath operators have fixed arities. Finally, note that the case where $W$
is a complex expression is analogous to the previous cases depending on the head symbol of
$W$.\iffull\ednote{we may want to make an example of this in the long version}\fi
%\end{newpart}

\section{Lifting Associative Operators}
Binary associative operators have notational peculiarities of their own. While we tend to
write then as binary, as ``$a+b+c$'', we recognise that this is ``really'' one addition of
three numbers, and both MathML-Content and OpenMath would represent this as a
{\element{plus}} with three arguments. Mathematica distinguishes such operators as
{\element{Flat}} and OpenMath's Simple Type System \cite{Davenport2000c} as
{\element{nassoc}}. It therefore makes sense to think of applying them to collections of
arguments, and mathematical notation does this all the time (see table \ref{Tab:big}).

\begin{figure}[h]\vspace*{-1.5em}
\begin{tabular}{lcccccc}\hline
``small''&$a_1+a_2+a_3$&$a_1a_2a_3$&$a_1\cap a_2\cap a_3$
&$a_1\cup a_2\cup a_3$
&$a_1\otimes a_2\otimes a_3$
&$a_1\lor a_2\lor a_3$
\\
small Unicode&&&225C&225B&220A&225F\\
``big''&$\sum_{i=1}^3 a_i$&$\prod_{i=1}^3 a_i$&$\bigcap_{i=1}^3 a_i$
&$\bigcup_{i=1}^3 a_i$
&$\bigotimes_{i=1}^3 a_i$
&$\bigvee_{i=1}^3 a_i$
\\
big Unicode&1350&1351&1354&1353&134E&1357\\\hline
\end{tabular}\vspace*{-.5em}
\caption{``Big'' operators}\label{Tab:big}\vspace*{-2em}
\end{figure}

With the exception of $\sum$ and $\prod$, which \cite{Davenportminor2008}
regarded as being among the ``irregular verbs'' of mathematical notation, we can see a
familiar pattern: the operator that applies to a collection of argument is
``bigger'' than its infix binary equivalent. The designers of
Unicode{\iffull{\footnote{One might object that the designer of \TeX{} did not do as
well, since the last column of Figure \ref{Tab:big} is coded using
{\element{lor}} for the first row, but {\element{bigvee}} for the third.
However,  {\element{lor}} is in fact merely an alias for {\element{vee}}.}\fi}
have done as well as might be hoped for in mapping these symbols to `related'
code points in Unicode space for the corresponding glyphs.

How are these ``big'' operators going to be represented?  For those it ``knows'' about
\cite[4.2.3.2]{WorldWideWebConsortium2003b} (the list is, with our decorations. given in
Figure \ref{fig:MMLnary}: the ones marked ${}^\P$ are no longer $n$-ary in {\emph{strict}}
MathML 3), 
\begin{figure}\vspace*{-1.5em}\centering
  \begin{tabular}{p{11cm}}\hline
{\tt plus}, {\tt times}, {\tt max}${}^*$, {\tt min}${}^*$, {\tt gcd}${}^*$,
{\tt lcm}${}^*$, {\tt mean}${}\ddag$, {\tt sdev}${}\ddag$, {\tt
variance}${}\ddag$, {\tt median}${}\ddag$, {\tt mode}${}\ddag$, {\tt
and}${}^*$, {\tt or}${}^*$, {\tt xor}${}\dag$, {\tt union}${}^*$, {\tt
intersect}${}^*$, {\tt cartesianproduct}${}\dag$, {\tt compose}${}\dag$, {\tt
eq}${}^\P$, {\tt leq}${}^\P$,
{\tt lt}${}^\P$, {\tt geq}${}^\P$, {\tt gt}${}^\P$\\\hline
\end{tabular}
\vspace*{-.5em}
\caption{MathML 2's $n$-ary operators}\label{fig:MMLnary}\vspace*{-1.5em}
\end{figure}
%$\sum$ and $\prod$ will be considered later.
MathML can use bound variables and conditions, so the last item from Figure~\ref{Tab:big}
would be shown on the left in Figure~\ref{fig:bigvee}. It is not clear from
\cite{WorldWideWebConsortium2003b} whether the same construct can be applied to a
user-defined operator, but it would be reasonable. OpenMath, on the other hand,
has an explicit lifting operator {\element{apply\_to\_list}},
see Figure~\ref{fig:bigvee} right.

\begin{figure}[ht]\centering%\vspace*{-1.5cm}
\lstset{language=MathML2,mathescape,frame=none,numbers=none}
\lstset{mathescape,aboveskip=-.7em,belowskip=-1.2em}
\begin{tabular}{|p{5.7cm}|p{5.8cm}|}\hline
\begin{lstlisting}
<apply>
  <or/>
 <bvar><ci>i</ci></bvar>
  <lowlimit><cn>1</cn></lowlimit>
  <uplimit><cn>3</cn></uplimit>
  $\psom{a_i}$
</apply>
\end{lstlisting}
& \lstset{language=OpenMath}
\begin{lstlisting}
<OMA>
  <OMS name="apply_to_list" cd="fns2"/>
  <OMS name="or" cd="logic1"/>
  <OMA>
    <OMS cd="list1" name="make_list"/>
    $\psom{1}$ $\psom{3}$ $\psom{\lambda{i}.a_i}$
 </OMA>
</OMA>
\end{lstlisting}\\\hline
%    <OMI>1</OMI>
%    <OMI>3</OMI>
\end{tabular}\vspace*{-.5em}
\caption{$\bigvee$ in OpenMath and MathML}\label{fig:bigvee}\vspace*{-1.5em}
\end{figure}

Many of the operators $\oplus$ listed in Figure \ref{fig:MMLnary}, those we
have marked ${}^*$, have two additional properties:
\begin{description}
\item[idempotence]$\forall f$ $f\oplus f=f$;
\item[monotonicity]There is some discrete order $\succ$ such that $\forall
f,g$ $f\oplus g\succ g$.
\end{description}
The first means that it make sense to apply $\oplus$ to a {\it set\/}, i.e.
$\bigoplus S$. The second means that it makes sense to talk about
$\bigoplus_{i=1}^\infty s_i$, as being the point where the construct
stabilises under $\succ$, or some kind of infinite object otherwise. 
OpenMath's construction has no problem with, say, $\bigvee F$, but MathML has to
write this as $\bigvee_{p\in F}p$ and use {\element{condition}} to represent
the $p\in F$.
\par
The statistical operators (marked ${}\ddag$), when applied to discrete sets,
and those marked ${}\dag$, only make sense over finite collections, but $\sum$
and $\prod$, as well as being lexically irregular in not being the infix
operators writ large, are different in that they {\it can\/} have a calculus
connotation. Here neither OpenMath nor MathML 3 make any clear distinctions,
nor, in their defence, do the vast majority of mathematics texts. Is that sum
meant to be absolutely convergent or only conditionally convergent? Only a
careful analysis of the surrounding text will show, if then.
\par
To help those authors who wish to make such distinctions, OpenMath probably
{\it ought\/} to have a CD of symbols with finer distinctions, just as it should for
the various kinds of integrals such as Cauchy Principal Value.
%the various kinds of integrals such as Cauchy Principal Value, but this is not
%an OpenMath/MathML issue.
\section{Conclusion}
We have listed four areas where MathML (1--2) and OpenMath have taken different routes to
the expressivity of mathematical meaning.  
In the case of MathML's {\element{condition}}, we have seen one very general concept that
does not have a single formalisation, and this led to the pragmatic/strict distinction in
MathML 3. We have seen the utility of ``restricted'' quantifiers, even though they are not
logically necessary, and \cite{DavenportKohlhase2009c} proposes their addition to
OpenMath.

In the case of the calculus operations, this reflected a genuine split in the approaches
to the calculus operations, whether one viewed them as algebraic or analytic
operations. Since neither is `wrong', but the two {\it are\/} different (for example the
``Fundamental Theorem of Calculus'' is a theorem from the analytic point of view, but a
definition in the algebraic view), a converged view at MathML/OpenMath 3 should
incorporate both.

\section*{Acknowledgements}
The unification effort described here has benefited from the input of many people, notably
Olga Caprotti, David Carlisle, Sam Dooley, Christoph Lange, Paul Libbrecht, Bruce Miller,
Robert Miner, Florian Rabe, Chris Rowley. The authors are indebted to David Carlisle for
comments on an earlier version of the paper.

%\vfill
%\pagebreak
\iffull
%\centerline{--- Old version from here ---}
\begin{lstlisting}[label=ex:exists,language=MathML2,
  caption={MathML 1 for ``there exists $x$ such that $x^5 < 3$''}]
    <apply><exists/>
      <bvar><ci> x </ci></bvar>
      <condition>
        <reln><lt/>
          <apply><power/>
            <ci>x</ci>
            <cn>5</cn>
          </apply>
          <cn>3</cn>
        </reln>
      </condition>
    </apply>
\end{lstlisting}

\begin{lstlisting}[label=ex:forall,language=MathML2, 
  caption={MathML-1 for ``for all $x$,$y$ such that $x^y < 1$ and $y^x < x + y$,$x < Q(y)$''}]
    <apply><forall/>
      <bvar><ci>x</ci></bvar>
      <bvar><ci>y</ci></bvar>
      <condition>
        <apply><and/>
          <reln>
            <lt/>
            <apply><power/>
              <ci>x</ci>
              <ci>y</ci>
            </apply>
            <cn>1</cn>
          </reln>
          <reln>
            <lt/>
            <apply><power/>
              <ci>y</ci>
              <ci>x</ci>
            </apply>
            <apply><plus/>
              <ci>y</ci>
              <ci>x</ci>
            </apply>
          </reln>
        </apply>
      </condition>
      <reln><lt/>
         <ci> x </ci>
         <apply>
           <fn><ci> x </ci></fn>
           <ci> y </ci>
         </apply>
      </reln>
    </apply>
\end{lstlisting}

\begin{lstlisting}[label=ex:exists2,language=MathML2,
  caption={MathML 1 for ``there exists $x < 3$ such that $x^2=4$''}]
    <apply>
      <exists/>
      <bvar><ci> x </ci></bvar>
      <condition>
        <reln><lt/><ci>x</ci><cn>3</cn></reln>
      </condition>
      <reln>
        <eq/>
        <apply>
          <power/><ci>x</ci><cn>2</cn>
        </apply>
        <cn>4</cn>
      </reln>
    </apply>
\end{lstlisting}
\fi
\iffull

\begin{lstlisting}[label=ex:connotin,language=MathML2,
  caption={MathML 1 for ``there exists $x$ such that $x^5 < 3$''}]
    <apply><min/>
      <bvar><ci>x</ci></bvar>
      <condition>
        <reln><notin/><ci> x </ci><ci type="set"> B </ci></reln>
      </condition>
      <apply>
        <power/>
        <ci> x </ci>
        <cn> 2 </cn>
      </apply>
    </apply>
\end{lstlisting}

\begin{table}[h]
\caption{{\tt condition} in Chapter  4\label{table4}}
\begin{tabular}{llll}
MathML 2&heading&\iffull This\else\cite{DavenportKohlhase2009b}\fi&resolution\\
Chapter 4&&\iffull document\fi&\\
4.2.1.8&qualifiers&\ref{42181}&{\tt suchthat}\\
&&\ref{42182}&{\tt map}\\
4.2.3.2&operators with&\ref{42321}&A calculus issue\\
       &qualifiers&\ref{42322}&{\tt suchthat}\\
       &          &\ref{42323}&no solution\\
       &          &\ref{42324}&{\tt forallin}${}^1$\\
4.2.5  &Conditions&\ref{4251}&none needed\\
       &          &\ref{4252}&{\tt forall/existsin}\\
       &          &\ref{4253}&{\tt existsin}${}^1$\\
4.4.2.7.2&Conditions&\ref{442723}&{\tt suchthat}\\
4.4.3.4&Maximum&\ref{44341}&{\tt map}?${}^2$\\
&&\ref{44342}&{\tt suchthat}\\
4.4.3.17&{\tt forall}&\ref{443171}&{\tt forallin}${}^3$\\
        &            &\ref{443172}&{\tt forallin}${}^3$\\
4.4.5.1&{\tt int}&\ref{4451}&Needs work\\
4.4.5.6.1&{\tt bvar}&\ref{44561}&{\tt suchthat}\\
4.4.5.6.2&{\tt bvar}&\ref{44562}&Needs work\\
4.4.6.1.2&{\tt set}&\ref{44612}&{\tt suchthat}\\
4.4.6.2.2&{\tt list}&\ref{44622}&{\tt integer\_interval}\\
4.4.6.7&Subset&\ref{4467}&loose\\
4.4.7.1&Sum&\ref{4471}&None needed\\
4.4.7.2&Product&\ref{4472}&None needed\\
4.4.7.3&Limit&\ref{4473}&Use {\tt limit1} CD\\
\end{tabular}
\begin{center}
Notes
\end{center}
\begin{enumerate}\itemsep=0pt
\setlength\itemsep{0pt}
\item or nothing special.
\item the query is caused by the fact that this fragment is
close to meaningless.
\item but {\tt forallin} doesn't do a perfect job.
\end{enumerate}
\end{table}
\begin{table}[h]
\caption{{\tt condition} in Appendix C\label{Ctable}}
\begin{tabular}{llll}
MathML 2&heading&\iffull This\else\cite{DavenportKohlhase2009b}\fi&resolution\\
Appendix C&&\iffull document\fi&\\
C.2.2.4&{\tt interval}&section \ref{C224}&{\tt suchthat}; ??\\
C.2.2.5&{\tt inverse}&section \ref{C225}&{\tt forallin}\\
C.2.2.7&{\tt condition}&section \ref{C2271}&{\tt suchthat}\\
C.2.2.7&{\tt condition}&section \ref{C2272}&Needs work\\
C.2.2.14&{\tt image}&section \ref{C2214}&{\tt forallin}${}^1$\\
C.2.2.15&{\tt domainofapplication}&section \ref{C2215}&none needed\\
C.2.3.1&{\tt quotient}&section \ref{C231}&{\tt forallin}${}^1$\\
C.2.3.2&{\tt factorial}&section \ref{C232}&{\tt forallin}${}^1$\\
C.2.3.3&{\tt divide}&section \ref{C233}&{\tt forallin}${}^1$\\
C.2.3.4&{\tt max}&section \ref{C234}&{\tt map}\\
C.2.3.5&{\tt min}&section \ref{C235}&{\tt map}\\
C.2.3.7&{\tt plus}&section \ref{C237}&{\tt forallin}${}^1$\\
C.2.3.8&{\tt power}&section \ref{C238}&{\tt forallin}${}^1$\\
C.2.3.9&{\tt quotient}&section \ref{C239}&As section \ref{C231}\\
C.2.3.10&{\tt times}&section \ref{C2310}&No formal MathML;\\
&&&As section \ref{C237}\\
C.2.3.18&{\tt times}&section \ref{C2318}&Apparently meaningless\\
C.2.3.23&{\tt real}&section \ref{C2323}&{\tt forallin}${}^1$\\
C.2.5.1&{\tt int}&section \ref{C251}&As section \ref{C2272}\\
C.2.5.6&{\tt bvar}&section \ref{C256}&{\tt forallin}${}^1$\\
C.2.5.8&{\tt divergence}&section \ref{C258}&Use {\tt limit1} CD\\
C.2.6.1&{\tt set}&section \ref{C261}&{\tt suchthat}\\
C.2.6.2&{\tt list}&section \ref{C262}&{\tt suchthat/list1}\\
C.2.6.7&{\tt subset}&section \ref{C267}&Apparently meaningless\\
C.2.7.1&{\tt sum}&section \ref{C271}&{\tt sum}\\
C.2.7.2&{\tt product}&section \ref{C272}&{\tt product}\\
C.2.7.3&{\tt limit}&section \ref{C273}&{\tt limit}\\
C.2.10.2&{\tt matrix}&section \ref{C2102}&No equivalent\\
C.2.11.3&{\tt rational}&section \ref{C2113}&{\tt forallin}\\
C.2.11.6&{\tt primes}&section \ref{C2116}&{\tt forallin}${}^1$\\
C.2.11.15&{\tt infinity}&section \ref{C211151}&{\tt forallin}${}^1$\\
C.2.11.15&{\tt infinity}&section \ref{C211152}&Use {\tt limit1} CD\\
\end{tabular}
\begin{center}
${}^1$ or nothing special
\end{center}
\end{table}
%\begin{abstract}\noindent
%JHD's attempt to look at the uses of {\element{condition}} in the MathML 2
%specification. Chapter 4 and Appendix C are now complete, in the sense that
%every use of {\element{condition}} in them has been analysed. The conclusion is
%that almost all of the can be represented in terms of current OpenMath, or
%(better but not strictly necessary) current OpenMath with two additional
%symbols: {\element{forallin}} and {\element{existsin}}. 
%\par
%This version updated following the flashmeeting
%(\url{http://fm.ea-tel.eu/fm/flashmeeting.php?pwd=356a23-15076}) on 6 November
%2008.
%\end{abstract}
%\section{Introduction}
%This note follows from the OpenMath tele-conference at 15:00
%GMT\footnote{16:00 BST, 17:00 CET.} on 10/10\discretionary{/}{/}{/}2008. See
%Michael Kohlhase (\url{m.kohlhase@jacobs-university.de})'s mail
%\url{48F05077.7010004@jacobs-university.de}.
%%Date: Sat, 11 Oct 2008 09:06:31 +0200
%\par
%The summary reads as follows.
%\begin{quotation}\noindent
%James was tasked to make sense of the integration/differentiation
%examples from the MathML 2 spec and make concrete suggestions for the
%expression-based calculus CD.
%\end{quotation}
%The detailed record shows that it should be a wider look at {\it all\/} uses
%of condition, and this version is an attempt to look at all occurrences of
%{\element{condition}} in the MathML 2 specification.
\section{The current state of MathML 2}
From table \ref{table4} (page \pageref{table4}) we see that the first dubious
case in the body of the standard is described in our section \ref{42321},
where the alternative in MK's note \cite{Kohlhase2008} seems not to conform to OpenMath
in practice. This example also appears in sections \ref{4451} and \ref{44562}.
%\par
From table \ref{Ctable} (page \pageref{Ctable}) we see that the only dubious
case in Appendix C is in our section \ref{C2272}, which is essentially the same
example.
\par
Our section \ref{42323} illustrates a problem with multivariate definite
integrals that OpenMath cannot represent directly, and that MathML requires a
non-intuitive correspondence of order of variables to express.
\par
We note that our section \ref{4252} is an excellent tribute to the power of the
proposed new symbols {\element{forallin}} and {\element{existsin}}. JHD
has been trying to find the origin of these symbols, but it seems not to be in
his archive. Memory is that MK suggested {\element{forallin}}. The point is
that the head of an {\element{OMBIND}} need for be a symbol, but can be a compound
expression, so using 
\begin{lstlisting}
<OMA>
  <OMS name="forallin" cd="quant2"/>
  <OMV name="S"/>
</OMA>
\end{lstlisting}
as the head is the same as using {\element{<OMS name="forall" cd="quant1"/>}}
except that the variable(s) bound are restricted to range over $S$. Note that
the bound variables do {\it not\/} explicitly appear in the restriction, so
this does not fall foul of OpenMath's requirement highlighted on page
\pageref{OMbound}.
We should note what {\element{forallin}} does, and does not, encode.
\begin{description}
\item[does]$\forall n\in\N$, $\forall m,n\in\N$, $\forall x\in[0,1]$, $\forall
x\in(0,\infty)$ (but not the equivalent $\forall x>0$).
\item[does not]$\forall n\in\N,x \in\R$ (this needs two nested
{\element{forallin}}s), $\forall n>2$, $\forall m<n\in\N$.
\end{description}
It is a legitimate argument that this {\element{forallin}} symbol is
privileging the use of $\in$ in what MathML called {\element{condition}}s.
\fi
\iffull
\section{Chapter 4}
This chapter is the substantive specification of MathML (Content).
\subsection{4.2.1.8 The use of qualifier elements}\label{4218}
This section introduces the {\tt condition} element, in what might be called a
``proof by example'' style.
\subsubsection{4.2.1.8(1) The use of qualifier elements}\label{42181}
This section contains the following quotation.
\begin{quotation}\noindent
A {\tt condition} element can be used to place restrictions directly on the
bound variable. This allows MathML to define sets by rule, rather than
enumeration. The following markup, for instance, encodes the set $\{x | x <
1\}$: 
\end{quotation}
\begin{lstlisting}[language=MathML2]
<set>
  <bvar><ci> x </ci></bvar>
  <condition>
    <apply>
      <lt/>
      <ci> x </ci>
      <cn> 1 </cn>
    </apply>
  </condition>
  <ci> x </ci>
</set>
\end{lstlisting}
This can be converted into OpenMath by means of {\element{suchthat}}: here is an
OpenMath example from {\element{set1}} reworked to encode the same mathematics.
\begin{lstlisting}
<OMOBJ xmlns="http://www.openmath.org/OpenMath" version="2.0"
       cdbase="http://www.openmath.org/cd">
  <OMA>
    <OMS cd="set1" name="suchthat"/>
    <OMS cd="setname1" name="R"/>
    <OMBIND>
      <OMS cd="fns1" name="lambda"/>
      <OMBVAR> <OMV name="x"/> </OMBVAR>
      <OMA>
        <OMS cd="relation1" name="lt"/>
        <OMV name="x"/>
        <OMI> 1 </OMI>
      </OMA>
    </OMBIND>
  </OMA>
</OMOBJ>
\end{lstlisting}
We note that the OpenMath makes it explicit that it is $(-\infty,1)$ that is
meant, not, say, $[0,1)$, and equally that it is $\R$, not $\Z$ or some other
set, that is the base type.
\subsubsection{4.2.1.8(2) The use of qualifier elements}\label{42182}
This section contains the following quotation.
\begin{quotation}\noindent
Another typical use is the "lifting" of $n$-ary operators to "big operators",
for instance the $n$-ary {\tt union} operator to the {\tt union} operator over
sets, as the union of the $U$-complements over a family $F$ of sets in this
construction.
\end{quotation} 
\begin{lstlisting}[language=MathML2]
<apply>
  <union/>
  <bvar><ci>S</ci></bvar>
  <condition>
    <apply><in/><ci>S</ci><ci>F</ci></apply>
  </condition>
  <apply><setdiff/><ci>U</ci><ci>S</ci></apply>
</apply>
\end{lstlisting}
This falls foul of the ambiguity in MathML's {\element{union}}
constructor\footnote{At the end of section 4.2.3 of the MathML specification,
we read
\begin{quotation}\noindent
If qualifiers are used, they should be followed by a single child element
representing a function or an expression in the bound variables specified in
the {\tt bvar} qualifiers.
Mathematically the operation is then taken to be over the arguments generated
by this function ranging over the specified domain of application, rather than
over an explicit list of arguments as is the case when qualifier schemata are
not used. 
\end{quotation}
A purist might objection that the presence of the qualifier is changing the
fundamental semantics of the $n$-ary operator.}
highlighted in \cite[especially slide 14]{Davenportminor2008}. The best OpenMath
translation would seem to be on the following lines.
\begin{lstlisting}
<OMA>
  <OMS name="big_union" cd="set3"/>
  <OMA>
    <OMS name="map" cd="set1"/>
    <OMBIND>
      <OMS cd="fns1" name="lambda"/>
      <OMBVAR> <OMV name="S"/> </OMBVAR>
      <OMA>
        <OMS name="setdiff" cd="set1"/>
        <OMV name="U"/>
        <OMV name="S"/>
      </OMA>
    </OMBIND>
    <OMV name="F"/>
  </OMA>
</OMA>
\end{lstlisting}
\subsection{4.2.2.2 Constructors}\label{4222}
This contains the sentence
\begin{quotation}\noindent
For example, a {\tt bvar} and a {\tt condition} element can be used to define
lists where membership depends on satisfying certain conditions. 
\end{quotation}
No example is given here, but the example in 4.2.1.8 (our section \ref{42181})
could be regarded as typical.
\subsection{4.2.3.2 Operators taking Qualifiers}\label{4232}
This section lists {\tt condition} among the qualifiers, and the operators
taking qualifiers (not necessarily {\tt condition}) as follows (* indicates
that section 4.2.3 of the MathML specification states that they do not take
{\tt condition} as a qualifier).
\begin{description}
\item[operators]{\tt int}\footnote{Sections \ref{42321}, \ref{C2272},
\ref{C2215}, \ref{C251}.}, {\tt sum}\footnote{Sections \ref{4471} and \ref{C271}.}, {\tt
product}\footnote{Section \ref{C272}.}, {\tt root, diff}*, {\tt partialdiff}*,
{\tt limit}\footnote{Sections \ref{4473}, \ref{C258}, \ref{C273}, \ref{C211152}.}, {\tt log}*,
{\tt moment}*, {\tt forall}\footnote{Sections
\ref{C225}, \ref{C2214}, \ref{C231}, \ref{C232}, \ref{C233}, \ref{C237},
\ref{C238}, \ref{C2310}, \ref{C2318}, \ref{C2323}. \ref{C256}, \ref{C2113},
\ref{C2116}, \ref{C211151}.}, {\tt exists}.
\item[$n$-ary operators]{\tt plus, times, max}\footnote{Sections \ref{442723},
\ref{C234}.},
{\tt min}\footnote{Sections \ref{44341}, \ref{C235}.}, {\tt gcd, lcm, mean, sdev,
variance, median, mode, and, or, xor, union}\footnote{Section \ref{42182}.},
{\tt intersect, cartesianproduct, compose, eq, leq, lt, geq, gt}.
\item[user defined operators]{\tt csymbol, ci}.
\item[missing](or not regarded as operators in MathML) {\tt
set}\footnote{Sections \ref{42181}, \ref{C261}.}, {\tt
list}\footnote{Section \ref{C262}.}, {\tt interval}\footnote{Section
\ref{C224}} {\tt matrix}\footnote{Section \ref{C2102}.}.
\item[spuriously absent]{\tt subset}\footnote{Sections \ref{4467} and
\ref{C267} --- note that we can make no sense of this last example.} and in
theory all other $n$-ary relations --- see section \ref{4467}.
\end{description}
\begin{quotation}\noindent
The ({\tt lowlimit},{\tt uplimit}) pair, the {\tt interval} and the {\tt
condition} are all shorthand notations specifying a particular domain of
application and should not be used if {\tt domainofapplication} is used.
\end{quotation}
It is not clear to the current author how the example in section \ref{C224}
can be cast in this mould, though.
\subsubsection{4.2.3.2 Operators taking Qualifiers (1)}\label{42321}
{\tt condition} has the following example.
\begin{lstlisting}[language=MathML2]
<apply>
  <int/>
  <bvar><ci>x</ci></bvar>
  <condition>
    <apply><in/><ci>x</ci><ci type="set">C</ci></apply>
  </condition>
  <apply><sin/><ci>x</ci></apply>
</apply>
\end{lstlisting}
This is a special case of the example discussed in section \ref{C2272}
\subsubsection{4.2.3.2 Operators taking Qualifiers (2)}\label{42322}
It is also stated that
\begin{lstlisting}[language=MathML2]
<apply>
  <int/>
  <bvar><ci>x</ci></bvar>
  <lowlimit><cn>0</cn></lowlimit>
  <uplimit><cn>1</cn></uplimit>
  <apply><power/><ci>x</ci><cn>2</cn></apply>
</apply>
\end{lstlisting}
(whose OpenMath equivalent in terms of {\tt calculus1} is
\begin{lstlisting}
<OMA>
  <OMS cd="calculus1" name="defint"/>
  <OMA>
    <OMS name="interval_cc" cd="interval1"/>
    <OMI> 0 </OMI>
    <OMI> 1 </OMI>
  </OMA>
  <OMBIND>
    <OMS cd="fns1" name="lambda"/>
    <OMBVAR> <OMV name="x"/> </OMBVAR>
    <OMA>
      <OMS cd="arith1" name="power"/>
      <OMV name="x"/>
      <OMI> 2 </OMI>
    </OMA>
  </OMBIND>
</OMA>
\end{lstlisting}
) can be written as
\begin{lstlisting}[language=MathML2]
<apply>
  <int/>
  <bvar><ci>x</ci></bvar>
  <domainofapplication>
    <set>
      <bvar><ci>t</ci></bvar>
      <condition>
        <apply>
          <and/>
          <apply><leq/><cn>0</cn><ci>t</ci></apply>
          <apply><leq/><ci>t</ci><cn>1</cn></apply>
        </apply>
      </condition>
      <ci>t</ci>
    </set>
  </domainofapplication>
  <apply><power/><ci>x</ci><cn>2</cn></apply>
</apply>
\end{lstlisting} 
The OpenMath equivalent of this would probably be the following.
\begin{lstlisting}
<OMA>
  <OMS cd="calculus1" name="defint"/>
  <OMA>
    <OMS cd="set1" name="suchthat"/>
    <OMS name="R" cd="setname1"/>
    <OMBIND>
      <OMS cd="fns1" name="lambda"/>
      <OMBVAR> <OMV name="t"/> </OMBVAR>
      <OMA>
        <OMS name="and" cd="logic1"/>
        <OMA>
          <OMS cd="relation1" name="gt"/>
          <OMV name="t"/>
          <OMI> 0 </OMI>
        </OMA>
        <OMA>
          <OMS cd="relation1" name="lt"/>
          <OMV name="t"/>
          <OMI> 1 </OMI>
        </OMA>
      </OMA>
    </OMBIND>
  </OMA>
  <OMBIND>
    <OMS cd="fns1" name="lambda"/>
    <OMBVAR> <OMV name="x"/> </OMBVAR>
    <OMA>
      <OMS cd="arith1" name="power"/>
      <OMV name="x"/>
      <OMI> 2 </OMI>
    </OMA>
  </OMBIND>
</OMA>
\end{lstlisting}
\subsubsection{4.2.3.2 Operators taking Qualifiers (3)}\label{42323}
The example continues as follows.
\begin{quotation}\noindent
This use extends to multivariate domains by using extra bound variables and a
domain corresponding to a Cartesian product as in:
\end{quotation}
\begin{lstlisting}[language=MathML2]
<apply>
  <int/>
  <bvar><ci>x</ci></bvar>
  <bvar><ci>y</ci></bvar>
  <domainofapplication>
    <set>
      <bvar><ci>t</ci></bvar>
      <bvar><ci>u</ci></bvar>
      <condition>
        <apply>
          <and/>
          <apply><leq/><cn>0</cn><ci>t</ci></apply>
          <apply><leq/><ci>t</ci><cn>1</cn></apply>
          <apply><leq/><cn>0</cn><ci>u</ci></apply>
          <apply><leq/><ci>u</ci><cn>1</cn></apply>
        </apply>
      </condition>
      <list><ci>t</ci><ci>u</ci></list>
    </set>
  </domainofapplication>
  <apply>
    <times/>
    <apply><power/><ci>x</ci><cn>2</cn></apply>
    <apply><power/><ci>y</ci><cn>3</cn></apply>
  </apply>
</apply>
\end{lstlisting}
\begin{quotation}\noindent
Note that the order of bound variables of the integral must correspond to the
order in the {\tt list} used by the {\tt set} constructor in the {\tt
domainofapplication}. 
\end{quotation}
OpenMath 2 as it (and its CDs) exists has no immediate answer to this, since
{\tt defint} from {\tt calculus1} explicitly only integrates unary functions.
Obviously one could replace it by two nested unary integrations. Trying to
represent it directly would fall foul of the potential ambiguity referred to
in the quotation immediately above. It seems to the author that one really
wants some way of representing the following expression:
\begin{equation}
\int_{x=0}^1\int_{y=0}^1x^2y^3{\rm d}x{\rm d}y,
\end{equation}
i.e. explicitly linking the variables to the bounds.
\subsubsection{4.2.3.2 Operators taking Qualifiers (4)}\label{42324}
The text on {\element{forall}} gives the following example.
\begin{lstlisting}[language=MathML2]
<apply>
  <forall/>
  <bvar><ci> x </ci></bvar>
  <condition>
    <apply><lt/>
      <ci> x </ci><cn> 9 </cn>
    </apply>
  </condition>
  <apply><lt/>
    <ci> x </ci><cn> 10 </cn>
  </apply>
</apply>
\end{lstlisting}
This could be solved with the forall/implies encoding, as in
\begin{lstlisting}
<OMBIND>
  <OMS name="forall" cd="quant1"/>
  <OMBVAR> <OMV name="x"/> </OMBVAR>
  <OMA>
    <OMS name="implies" cd="logic1"/>
    <OMA>
      <OMS name="lt" cd="relation1"/>
      <OMV name="x"/> 
      <OMI> 9 </OMI>
    </OMA>
    <OMA>
      <OMS name="lt" cd="relation1"/>
      <OMV name="x"/> 
      <OMI> 10 </OMI>
    </OMA>
  </OMA>
</OMBIND>
\end{lstlisting}
Alternatively, we could use a new symbol {\element{forallin}},  as in the
following.
\begin{lstlisting}
<OMBIND>
  <OMA>
    <OMS name="forallin" cd="quant2"/>
    <OMA>
      <OMS name="suchthat" cd="set1"/>
      <OMS cd="setname1" name="R"/>
      <OMBIND>
        <OMS cd="fns1" name="lambda"/>
        <OMBVAR> <OMV name="x"/> </OMBVAR>
        <OMA>
          <OMS name="lt" cd="relation1"/>
          <OMV name="x"/> 
          <OMI> 9 </OMI>
        </OMA>
      </OMBIND>
    </OMA>
  </OMA>
  <OMA>
    <OMS name="lt" cd="relation1"/>
    <OMV name="x"/> 
    <OMI> 10 </OMI>
  </OMA>
</OMBIND>
\end{lstlisting}
\subsection{4.2.5 Conditions}\label{425}
\subsubsection{First example}\label{4251}
\begin{lstlisting}[language=MathML2]
<apply>
  <exists/>
  <bvar><ci> x </ci></bvar>
  <condition>
    <apply><lt/>
      <apply>
        <power/>
        <ci>x</ci>
        <cn>5</cn>
      </apply>
      <cn>3</cn>
    </apply>
    </condition>
    <true/>
</apply>
\end{lstlisting}
The author's first reaction was ``why a condition: isn't this MathML equivalent
(and shorter)?''
\begin{lstlisting}[language=MathML2]
<apply>
  <exists/>
  <bvar><ci> x </ci></bvar>
  <apply><lt/>
    <apply>
      <power/>
      <ci>x</ci>
      <cn>5</cn>
    </apply>
    <cn>3</cn>
  </apply>
</apply>
\end{lstlisting}
Certainly this would be the obvious OpenMath encoding.
\begin{lstlisting}
<OMBIND>
  <OMS name="exists" cd="quant1"/>
  <OMBVAR> <OMV name="x"/> </OMBVAR>
  <OMA>
    <OMS name="lt" cd="relation1"/>
    <OMA>
      <OMS name="power" cd="arith1"/>
      <OMV name="x"/>
      <OMI> 5 </OMI>
    </OMA>
    <OMI> 3 </OMI>
  </OMA>
</OMBIND>
\end{lstlisting}
\subsubsection{Second example}\label{4252}
Moved to OM2009 paper.
\subsubsection{Third example}\label{4253}
\begin{lstlisting}[language=MathML2]
<apply>
  <exists/>
  <bvar><ci> x </ci></bvar>
  <condition>
    <apply><lt/><ci>x</ci><cn>3</cn></apply>
  </condition>
  <apply>
    <eq/>
    <apply>
      <power/><ci>x</ci><cn>2</cn>
    </apply>
    <cn>4</cn>
  </apply>
</apply>
\end{lstlisting}
This works well with {\element{existsrestricted}}.
\begin{lstlisting}
<OMBIND>
  <OMA>
    <OMS name="existsin" cd="quant2"/>
    <OMA>
      <OMS name="suchthat" cd="set1"/>
      <OMS cd="setname1" name="R"/>
      <OMBIND>
        <OMS cd="fns1" name="lambda"/>
        <OMBVAR> <OMV name="x"/> </OMBVAR>
        <OMA>
          <OMS name="lt" cd="relation1"/>
          <OMV name="x"/> 
          <OMI> 3 </OMI>
        </OMA>
      <OMBIND>
    </OMA>
  </OMA>
  <OMA>
    <OMS name="eq" cd="relation1"/>
    <OMA>
      <OMS name="power" cd="arith1"/>
      <OMV name="x"/>
      <OMI> 2 </OMI>
    </OMA>
    <OMI> 4 </OMI>
  </OMA>
</OMBIND>
\end{lstlisting}
\subsection{4.4.2.7.2 Examples (of condition)}\label{44272}
\subsubsection{First example}\label{442721}
\begin{lstlisting}[language=MathML2]
<condition>
  <apply><in/><ci> x </ci><ci type="set"> A </ci></apply>
</condition>
\end{lstlisting}
As a free-standing piece of code, it is hard to see what this means.
\subsubsection{Second example}\label{442722}
\begin{lstlisting}[language=MathML2]
<condition>
  <apply>
    <and/>
    <apply><gt/><ci> x </ci><cn> 0 </cn></apply>
    <apply><lt/><ci> x </ci><cn> 1 </cn></apply>
  </apply>
</condition>
\end{lstlisting}
As a free-standing piece of code, it is hard to see what this means, but it
crops up as a fragment of the next example.
\subsubsection{Third example}\label{442723}
\begin{lstlisting}[language=MathML2]
<apply>
  <max/>
  <bvar><ci> x </ci></bvar>
  <condition>
    <apply> <and/>
      <apply><gt/><ci> x </ci><cn> 0 </cn></apply>
      <apply><lt/><ci> x </ci><cn> 1 </cn></apply>
    </apply>
  </condition>
  <apply>
    <minus/>
    <ci> x </ci>
    <apply>
      <sin/>
      <ci> x </ci>
    </apply>
  </apply>
</apply>
\end{lstlisting}
As OpenMath does not have a {\element{max}} operator acting on functions,
the nearest translation would seem to be the following.
\begin{lstlisting} 
<OMA>
  <OMS name="max" cd="minmax1"/>
  <OMA>
    <OMS name="map" cd="set1"/>
    <OMBIND>
      <OMS name="lambda" cd="fns1"/>
      <OMBVAR> <OMV name="x"/> </OMBVAR>
      <OMA>
        <OMS name="minus" cd="arith1"/>
        <OMV name="x"/>
        <OMA>
          <OMS name="sin" cd="transc1"/>
          <OMV name="x"/>
        </OMA>
      </OMA>
    </OMBIND>
    <OMA>
      <OMS name="interval_oo" cd="interval1"/>
      <OMI> 0 </OMI>
      <OMI> 1 </OMI>
    </OMA>
  </OMA>
</OMA>
\end{lstlisting}
\subsection{4.4.3.4 Maximum and minimum ({\tt max}, {\tt min})}\label{4434}
This contains the following examples.
\subsubsection{4.4.3.4 {\tt max}, {\tt min} (1)}\label{44341}
\begin{lstlisting}[language=MathML2]
<apply>
  <min/>
  <bvar><ci>x</ci></bvar>
  <condition>
    <apply><notin/><ci> x </ci><ci type="set"> B </ci></apply>
  </condition>
  <apply>
      <power/>
    <ci> x </ci>
    <cn> 2 </cn>
  </apply>
</apply>
\end{lstlisting}
This is somewhat hard to interpret: if $x\notin B$, what {\it can\/} we say
about $x$? In general terms, though, a solution such as in our section
\ref{C234} seems appropriate.
\subsubsection{4.4.3.4 {\tt max}, {\tt min} (2)}\label{44342}
\begin{lstlisting}[language=MathML2]
<apply>
  <max/>
  <bvar><ci>x</ci></bvar>
  <condition>
    <apply><and/>
      <apply><in/><ci>x</ci><ci type="set">B</ci></apply>
      <apply><notin/><ci>x</ci><ci type="set">C</ci></apply>
    </apply>
  </condition>
  <ci>x</ci>
</apply>
\end{lstlisting}
This is a clear case of {\element{suchthat}}.
\begin{lstlisting}
<OMA>
  <OMS name="max" cd="minmax1"/>
  <OMA>
    <OMS name="suchthat" cd="set1"/>
    <OMV name="B"/>
    <OMBIND>
      <OMS cd="fns1" name="lambda"/>
      <OMBVAR> <OMV name="x"/> </OMBVAR>
      <OMA>
        <OMS name="notin" cd="set1"/>
        <OMV name="x"/>
        <OMV name="C"/>
      </OMA>
    </OMBIND>
  </OMA>
</OMA>
\end{lstlisting}
\subsection{4.4.3.17 Universal quantifier ({\tt forall})}\label{44317}
This contains the following examples.
\subsubsection{4.4.3.17 {\tt forall} (1)}\label{443171}
\begin{lstlisting}[language=MathML2]
<apply>
  <forall/>
  <bvar><ci> p </ci></bvar>
  <bvar><ci> q </ci></bvar>
  <condition>
    <apply><and/>
      <apply><in/><ci> p </ci><rationals/></apply>
      <apply><in/><ci> q </ci><rationals/></apply>
      <apply><lt/><ci> p </ci><ci> q </ci></apply>
    </apply>
  </condition>
  <apply><lt/>
      <ci> p </ci>
      <apply>
          <power/>
        <ci> q </ci>
        <cn> 2 </cn>
      </apply>
  </apply>
</apply>
\end{lstlisting}
This will clearly succumb to the forall/implies encoding.
{\element{forallin}} will cope with the
{\element{<apply><in/><ci> p </ci><rationals/></apply>}} clauses, but not, as
currently suggested, with the $p<q$ clause. This would give the following.
\begin{lstlisting}
<OMBIND>
  <OMA>
    <OMS name="forallin" cd="quant2"/>
    <OMS name="Q" cd="setname1"/>
  </OMA>
  <OMBVAR> <OMV name="p"/> <OMV name="q"/> </OMBVAR>
  <OMA>
    <OMS name="implies" cd="logic1"/>
    <OMA>
      <OMS name="lt" cd="logic1"/>
      <OMV name="p"/>
      <OMV name="q"/> 
    </OMA>
    <OMA>
      <OMS name="lt" cd="logic1"/>
      <OMV name="p"/>
      <OMA>
        <OMS name="power" cd="arith1"/>
        <OMV name="q"/> 
        <OMI> 2 </OMI>
      </OMA>
    </OMA>
  </OMA>
</OMBIND>
\end{lstlisting}
\subsubsection{4.4.3.17 {\tt forall} (2)}\label{443172}
\begin{lstlisting}[language=MathML2]
<apply>
  <forall/>
  <bvar><ci> n </ci></bvar>
  <condition>
    <apply><and/>
      <apply><gt/><ci> n </ci><cn> 0 </cn></apply>
      <apply><in/><ci> n </ci><integers/></apply>
    </apply>
  </condition>
  <apply>
    <exists/>
    <bvar><ci> x </ci></bvar>
    <bvar><ci> y </ci></bvar>
    <bvar><ci> z </ci></bvar>
    <condition>
      <apply><and/>
        <apply><in/><ci> x </ci><integers/></apply>
        <apply><in/><ci> y </ci><integers/></apply>
        <apply><in/><ci> z </ci><integers/></apply>
      </apply>
    </condition>
    <apply>
      <eq/>
      <apply>
        <plus/>
        <apply><power/><ci> x </ci><ci> n </ci></apply>
        <apply><power/><ci> y </ci><ci> n </ci></apply>
      </apply>
      <apply><power/><ci> z </ci><ci> n </ci></apply>
    </apply>
  </apply>
</apply>
\end{lstlisting}
Again, the restricted quantifiers help a great deal. but not perfectly.
\begin{lstlisting}
<OMBIND>
  <OMA>
    <OMS name="forallin" cd="quant2"/>
    <OMS name="Z" cd="setname1"/>
  </OMA>
  <OMBVAR> <OMV name="n"/> </OMBVAR>
  <OMA>
  <OMS name="implies" cd="logic1"/>
    <OMA>
      <OMS name="gt" cd="relation1"/>
      <OMV name="n"/>
      <OMI> 0 </OMI>
    </OMA>
    <OMBIND>
      <OMA>
        <OMS name="existsin" cd="quant2"/>
        <OMS name="Z" cd="setname1"/>
      </OMA>
      <OMBVAR> <OMV name="x"/> <OMV name="y"/> <OMV name="z"/> </OMBVAR>
      <OMA>
        <OMS name="eq" cd="relation1"/>
        <OMA>
          <OMS name="plus" cd="arith1"/>
          <OMA>
            <OMS name="power" cd="arith1"/>
            <OMV name="x"/>
            <OMV name="n"/>
          </OMA>
          <OMA>
            <OMS name="power" cd="arith1"/>
            <OMV name="y"/>
            <OMV name="n"/>
          </OMA>
        </OMA>
        <OMA>
          <OMS name="power" cd="arith1"/>
          <OMV name="z"/>
          <OMV name="n"/>
        </OMA>
      </OMA>
    </OMBIND>
  </OMA>
</OMBIND>
\end{lstlisting}
\subsection{4.4.5.1 Integral ({\tt int})}\label{4451}
This contains the following example.
\begin{lstlisting}[language=MathML2]
<apply>
  <int/>
  <bvar><ci> x </ci></bvar>
  <condition>
    <apply><in/>
      <ci> x </ci>
      <ci type="set"> D </ci>
    </apply>
  </condition>
  <apply><ci type="function"> f </ci>
    <ci> x </ci>
  </apply>
</apply>
\end{lstlisting}
Up to renaming, this is identical to the example in section \ref{C2272}, and
the same comments apply.
\subsection{4.4.5.6.1 Bound variable ({\tt bvar})}\label{44561}
This contains the following example, which is mainly meant to illustrate the
use of the {\element{id=}} construct.
\begin{lstlisting}[language=MathML2]
<set>
  <bvar><ci id="var-x"> x </ci></bvar>
  <condition>
    <apply>
      <lt/>
      <ci definitionURL="#var-x"> x </ci>
      <cn> 1 </cn>
    </apply>
  </condition>
</set>
\end{lstlisting}
From our point of view, this looks like a {\element{suchthat}}, except that the
universe is unspecified in the MathML. Taking a guess for this, we end up with
the following OpenMath.
\begin{lstlisting}
<OMA>
  <OMS cd="set1" name="suchthat"/>
  <OMS cd="setname1" name="R"/>
  <OMBIND>
    <OMS cd="fns1" name="lambda"/>
    <OMBVAR> <OMV name="x"/> </OMBVAR>
    <OMA>
      <OMS cd="relation1" name="lt"/>
      <OMV name="x"/>
      <OMI> 1 </OMI>
    </OMA>
  </OMBIND>
</OMA>
\end{lstlisting}
XML {\element{id=}} tags could probably be used here.\ednote{Anyone wish to do so?}
This might also be an occasion for {\tt OMR}.
\subsection{4.4.5.6.2 Bound variable ({\tt bvar})}\label{44562}
Up to renaming, this is identical to the example in section \ref{C2272}, and
the same comments apply.  
\begin{lstlisting}[language=MathML2]
<apply>
  <int/>
  <bvar><ci> x </ci></bvar>
  <condition>
    <apply><in/><ci> x </ci><ci> D </ci></apply>
  </condition>
  <apply><ci type="function"> f </ci>
    <ci> x </ci>
  </apply>
</apply>
\end{lstlisting}
\subsection{4.4.6.1.2 Set ({\tt set})}\label{44612}
This contains the following example.
\begin{lstlisting}[language=MathML2]
<set>
  <bvar><ci> x </ci></bvar>
  <condition>
    <apply><and/>
      <apply><lt/>
        <ci> x </ci>
        <cn> 5 </cn>
      </apply>
      <apply><in/>
        <ci> x </ci>
        <naturalnumbers/>
      </apply>
    </apply>
  </condition>
  <ci> x </ci>
</set>
\end{lstlisting}
Again, this fits very naturally as a {\element{suchthat}}, indeed probably more
naturally than the original MathML.
\begin{lstlisting}
<OMA>
  <OMS cd="set1" name="suchthat"/>
  <OMS cd="setname1" name="N"/>
  <OMBIND>
    <OMS cd="fns1" name="lambda"/>
    <OMBVAR> <OMV name="x"/> </OMBVAR>
    <OMA>
      <OMS cd="relation1" name="lt"/>
      <OMV name="x"/>
      <OMI> 5 </OMI>
    </OMA>
  </OMBIND>
</OMA>
\end{lstlisting}
\subsection{4.4.6.2.2 List ({\tt list})}\label{44622}
This contains the following example.
\begin{lstlisting}[language=MathML2]
<list order="numeric">
  <bvar><ci> x </ci></bvar>
  <condition>
    <apply><lt/>
      <ci> x </ci>
      <cn> 5 </cn>
    </apply>
  </condition>
  <ci> x </ci>
</list>
\end{lstlisting}
This has no direct equivalent in OpenMath, not least because there is no
equivalent of {\element{order="numeric"}}, but also because one has to assume that
the list is being selected from $\N$, because if it were selected from $\Z$ or
$\R$ is would have no infimum.
\par
The best translation is probably the following.
\begin{lstlisting}
<OMA>
  <OMS name="integer_interval" cd="interval1"/>
  <OMI> 0 </OMI>
  <OMI> 4 </OMI>
</OMA>
\end{lstlisting}
The reader may protest that {\element{integer\_interval}} is ``special'', but surely no
more so than implicitly assuming selection from $\N$.

\subsection{4.4.6.7 Subset ({\tt subset})}\label{4467}
This contains the following sentence.
\begin{quotation}\noindent
The subset element is an $n$-ary set relation (see Section 4.2.4 Relations).
As an $n$-ary operator, its operands may also be generated as described in
[$n$-ary operators] Therefore it may take qualifiers.
\end{quotation} 
This appears to justify the following example.
\begin{lstlisting}[language=MathML2]
<apply>
  <subset/>
  <bvar><ci type="set">S</ci></bvar>
  <condition>
    <apply><in/>
      <ci>S</ci>
      <ci type="list">T</ci>
    </apply>
  </condition>
  <ci>S</ci>
</apply>
\end{lstlisting}
As far as the present author can see, this states that the list $T$ is
linearly ordered by $\subseteq$. OpenMath does not, and seems to have decided
that it will not\footnote{MK posed this question in e-mail of 21/9/2008,
tracked as {\tt http://wiki.openmath.org/?title=cd\%3Arelation1}.}, have
$n$-ary relations of this form.
\par
Any translation has therefore to be loose: we propose the
following\footnote{Which is conditional on assuming that this is the right way
to select elements from a {\tt list} --- a debate which we have had, but I
cannot remember the resolution.}.
\begin{lstlisting}
<OMBIND>
  <OMA>
    <OMS name="forallin" cd="quant2"/>
    <OMS name="N" cd="setname1"/>
  </OMA>
  <OMBVAR> <OMV name="i"/> <OMV name="j"/> </OMBVAR>
  <OMS name="implies" cd="logic1"/>
    <OMA>
      <OMS name="lt" cd="relation1"/>
      <OMV name="i"/>
      <OMV name="j"/>
    <OMA>
      <OMS name="subset" cd="set1"/>
      <OMA>
        <OMV name="T"/>
        <OMV name="i"/>
      </OMA>
      <OMA>
        <OMV name="T"/>
        <OMV name="j"/>
      </OMA>
    </OMA>
    </OMA>
</OMBIND> 
\end{lstlisting}
\subsection{4.4.7.1 Sum ({\tt sum}) }\label{4471}
This contains the following example.
\begin{lstlisting}[language=MathML2]
<apply>
  <sum/>
  <bvar><ci> x </ci></bvar>
  <condition>
    <apply> <in/>
      <ci> x </ci>
      <ci type="set"> B </ci>
    </apply>
  </condition>
  <apply><ci type="function"> f </ci>
    <ci> x </ci>
  </apply>
</apply>
\end{lstlisting}
This has an obvious translation into OpenMath, in which $x$ isn't even needed.
It could, of course, be supplied by replaced $f$ by $\lambda x.f(x)$.
\begin{lstlisting}
<OMA>
  <OMS name="sum" cd="arith1"/>
  <OMV name="B"/>
  <OMV name="f"/>
</OMA>
\end{lstlisting}
\subsection{4.4.7.2 Product ({\tt product}) }\label{4472}
This contains the following example.
\begin{lstlisting}[language=MathML2]
<apply>
  <product/>
  <bvar><ci> x </ci></bvar>
  <condition>
    <apply> <in/>
      <ci> x </ci>
      <ci type="set"> B </ci>
    </apply>
  </condition>
  <apply><ci type="function"> f </ci>
    <ci> x </ci>
  </apply>
</apply>
\end{lstlisting}
The same remarks as in the previous section apply.
\subsection{4.4.7.3 Limit ({\tt limit}) }\label{4473}
This contains the following example.
\begin{lstlisting}[language=MathML2]
<apply>
  <limit/>
  <bvar><ci> x </ci></bvar>
  <condition>
    <apply>
      <tendsto type="above"/>
      <ci> x </ci>
      <ci> a </ci>
    </apply>
  </condition>
  <apply><sin/>
     <ci> x </ci>
  </apply>
</apply>
\end{lstlisting}
This can be translated into OpenMath as follows.
\begin{lstlisting}
<OMA>
  <OMS cd="limit1" name="limit"/>
  <OMV name="a"/>
  <OMS cd="limit1" name="above"/>
  <OMBIND>
    <OMS cd="fns1" name="lambda"/>
    <OMBVAR> <OMV name="x"/> </OMBVAR>
    <OMA>
      <OMS name="sin" cd="transc1"/>
      <OMV name="x"/>
    </OMA>
  </OMBIND>
</OMA>
\end{lstlisting}
\section{Appendix C}
\subsection{C.2.2.4 MMLdefinition: {\tt interval}}\label{C224}
This contains the following example.
\begin{lstlisting}[language=MathML2]
<interval>
  <bvar><ci>x</ci></bvar>
  <condition>
    <apply><lt/><cn>0</cn><ci>x</ci></apply>
    </condition>
</interval>
\end{lstlisting}
Presumably this represents $(0,\infty)$. Equally, this is presumably intended
to close over {\element{<interval>}}, i.e. $x$ is bound in this expression, and
freely $\alpha$-convertible. However, it seems to JHD to be purely ``luck''
that this defines an interval. What about the following?
\begin{lstlisting}[language=MathML2]
<interval>
  <bvar><ci>x</ci></bvar>
  <condition>
    <apply><lt/><cn>1</cn>
       <apply> <power/>
         <ci>x</ci>
         <cn>2</cn>
         </apply>
      </apply>
    </condition>
</interval>
\end{lstlisting}
By the same logic, this is $(-\infty,-1)\cup(1,\infty)$.  As far as JHD can
see, this use of {\element{interval}} is basically a declaration of a set, coupled
with an assertion that that set is in fact an interval, which assertion is, in
fact, true in the first case, but not the second.
\par
The declaration of the set can be done with {\element{suchthat}}, the assertion is
a problem OpenMath has not really addressed.
\par
It is also not clear that this fragment is in fact legal. The specification
says elsewhere [4.4.2.4.1] that
\begin{quotation}\noindent
The interval element expects two child elements that evaluate to real numbers. 
\end{quotation} 
\subsection{C.2.2.5 MMLdefinition: {\tt inverse}}\label{C225}
This contains the following example (our formatting).
\begin{lstlisting}[language=MathML2]
<apply><forall/>
  <bvar><ci>y</ci></bvar>
  <bvar><ci type="function">f</ci></bvar>
  <condition>
    <apply><in/>
      <ci>y</ci>
      <apply>
        <csymbol definitionURL="domain">
          <mtext>Domain</mtext></csymbol>
        <apply><inverse/><ci type="function">f</ci></apply>
      </apply>
    </apply>
  </condition>
  <apply><eq/>
    <apply><ci type="function">f</ci>
      <apply><apply><inverse/><ci type="function">f</ci></apply>
        <ci>y</ci>
      </apply>
    </apply>
    <ci>y</ci>
  </apply>
</apply>
\end{lstlisting}
The associated textual description is 
\begin{lstlisting}
ForAll( y, such y in domain( f^(-1) ), f( f^(-1)(y) ) = y
\end{lstlisting}
which does not include the fact that $f$ is in the scope of $\forall$.
\par
This usage seems to be basically a ``typed quantifier'', and as such could
look like the following.
\begin{lstlisting}
<OMBIND>
  <OMA>
    <OMS name="forallin" cd="quant2"/>
    <OMA>
      <OMS name="domain" cd="fns1"/>
      <OMA>
        <OMS name="inverse" cd="fns1"/>
        <OMV name="f"/>
      </OMA>
    </OMA>
  </OMA>
  <OMBVAR> <OMV name="y"/> </OMBVAR>
  <OMA>
    <OMS name="eq" cd="relation1"/>
    <OMA>
      <OMV name= "f"/>
      <OMA>
        <OMA>
          <OMS name="inverse" cd="fns1"/>
          <OMV name= "f"/>
        </OMA>
        <OMV name= "y"/>
      </OMA>
    </OMA>
    <OMV name= "y"/>
  </OMA>
</OMBIND>
\end{lstlisting}
This resembles the ``vernacular'', and does not bind $f$. If we wanted to do
so, along the lines of the MathML, we would have to wrap the whole thing in
another {\element{forall}}. The two cannot be combined, as we want to be able to
$\alpha$-convert $f$ in the argument of {\element{forallin}}, and, as MK's
note of the tele-conference reads:
\begin{quotation}\noindent
In particular, we do not
      want to accept the occurrence of the bound variable in the
      (complex) binding operator. In particular, the OpenMath2 standard
      restricts alpha-conversion to the second and third children of the
      OMBIND, which is consistent with this view.
\end{quotation} 
See section \ref{sec:defintcond} below for an example of where this rule
bites.
\subsection{C.2.2.7 MMLdefinition: {\tt condition}}
This section contains two examples.
\subsubsection{C.2.2.7(1) MMLdefinition: {\tt condition}}\label{C2271}
\begin{lstlisting}[language=MathML2]
<condition>
  <apply><lt/>
    <apply><power/><ci>x</ci><cn>5</cn></apply>
    <cn>3</cn>
  </apply>
</condition>
\end{lstlisting}
It is hard to understand the meaning of this fragment in isolation (and
presumably the reader was not intended to do so). If it was intended to be
part of an encoding of a set (mathematically, representing the interval
$(-\infty,3^{1/5})$, or possibly $[0,3^{1/5})$), then {\element{suchthat}} from
{\element{set1}} is appropriate. Here is the example from {\element{set1}} reworked to
match the MathML example (as $(-\infty,3^{1/5})$: if one wanted $[0,3^{1/5})$
one would have to add $x\ge0$, or change from $\bf R$ as the base set).
\begin{lstlisting}
<OMOBJ xmlns="http://www.openmath.org/OpenMath" version="2.0"
       cdbase="http://www.openmath.org/cd">
  <OMA>
    <OMS cd="set1" name="suchthat"/>
    <OMS cd="setname1" name="R"/>
    <OMBIND>
      <OMS cd="fns1" name="lambda"/>
      <OMBVAR> <OMV name="x"/> </OMBVAR>
      <OMA>
        <OMS cd="relation1" name="lt"/>
        <OMA>
          <OMS cd="arith1" name="power"/>
          <OMV name="x"/>
          <OMI> 5 </OMI>
        </OMA>
        <OMI> 3 </OMI>
      </OMA>
    </OMBIND>
  </OMA>
</OMOBJ>
\end{lstlisting}
We should note that the OpenMath makes it clear that $x$ is bound by the
({\element{OMBIND}} whose first child is the)
{\element{lambda}}, whereas the MathML, being only a fragment, does not state the
scope.
\subsubsection{C.2.2.7(2) MMLdefinition: {\tt
condition}}\label{sec:defintcond}\label{C2272}
(Our formatting.)
\begin{lstlisting}[language=MathML2]
<apply><int/>
  <bvar><ci>x</ci></bvar>
  <condition>
    <apply><in/><ci>x</ci><ci type="set">C</ci></apply>
  </condition>
  <apply><ci type="function">f</ci><ci>x</ci></apply>
</apply>
\end{lstlisting}
This seems, to JHD, to be typical of the confusion that can arise over
{\element{condition}}s. Let us look at this expression in the vernacular: 
\begin{equation}\label{defintcond}
\int_{x\in C}f(x){\rm d} x,
\end{equation}
and observe that this is probably equivalent to $\int_{C}f(x){\rm d} x$.
\par
In terms of {\element{calculus1}} (calculus with functions in
\url{http://staff.bath.ac.uk/masjhd/differentiate.html})
this can be expressed as 
\begin{lstlisting}
<OMA>
  <OMS cd="calculus1" name="defint"/>
  <OMV name="C"/>
  <OMV name="f"/>
</OMA>
\end{lstlisting}
and $x$ doesn't appear at all. If one wants it to, then one writes
\begin{lstlisting}
<OMA>
  <OMS cd="calculus1" name="defint"/>
  <OMV name="C"/>
  <OMBIND>
    <OMS cd="fns1" name="lambda"/>
    <OMBVAR>
      <OMV name="x"/>
    </OMBVAR>
    <OMA>
      <OMV name="f"/>
      <OMV name="x"/>
    </OMA>
  </OMBIND>
</OMA>
\end{lstlisting}
as in the example in {\element{calculus1}}.
\par
MK's note \cite{Kohlhase2008} on {\element{calculus3}} (calculus with
expressions in \url{http://staff.bath.ac.uk/masjhd/differentiate.html}),
suggests (in MathML syntax)
\begin{lstlisting}[language=MathML2]
<bind>
  <apply>
    <csymbol cd="calculus3">defintset</csymbol>
    <ci>C</ci>
  </apply>
  <bvar><ci>x</ci></bvar>
  <apply><ci>f</ci><ci>x</ci></apply>
</bind>
\end{lstlisting}
which is a precise translation of the previous one from the language of
functions to expressions.
In OpenMath syntax it would be the following\footnote{In the circulated
version of {\tt calculus3}, there appear to be two symbols called {\tt
defint}. The second should presumably be {\tt defintset}.}.
\begin{lstlisting}
<OMBIND>
  <OMA>
    <OMS cd="calculus3" name="defintset"/>
    <OMV name="C"/>
  </OMA>
  <OMBVAR>
    <OMV name="x"/>
  </OMBVAR>
  <OMA>
    <OMV name="f"/>
    <OMV name="x"/>
  </OMA>
</bind>
\end{lstlisting}
MK's note also suggests an alternative way of expressing (\ref{defintcond}),
which in MathML syntax would be the following.
\begin{lstlisting}[language=MathML2]
<bind>
  <apply>
    <csymbol cd="calciulus3">defintcond</csymbol>
    <apply><in/>
      <ci>x</ci>
      <ci>C</ci>
    </apply>
  </apply>
  <bvar><ci>x</ci></bvar>
  <apply><ci>f</ci><ci>x</ci></apply>
</bind> 
\end{lstlisting}
In OpenMath syntax it would be the following.
\begin{lstlisting}
<OMBIND>
  <OMA>
    <OMS cd="calculus3" name="defintcond"/>
    <OMA>
      <OMS name="in" cd="set1"/>
      <OMV name="x"/>
      <OMV name="C"/>
    </OMA>
  </OMA>
  <OMBVAR>
    <OMV name="x"/>
  </OMBVAR>
  <OMA>
    <OMV name="f"/>
    <OMV name="x"/>
  </OMA>
</bind>
\end{lstlisting}
This, however, falls foul of %OpenMath's rule that 
\begin{quotation}\noindent\label{OMbound}
the OpenMath2 standard restricts alpha-conversion to the second and third
children of the OMBIND
\end{quotation}
and hence JHD does not see how {\element{defintcond}} as postulated can make its
way into OpenMath.
\subsection{C.2.2.14 MMLdefinition: {\tt image}\label{C2214}}
This section contains the following example.
\begin{lstlisting}[language=MathML2]
<apply><forall/>
  <bvar><ci>x</ci></bvar>
  <condition>
    <apply><in/>
      <ci>x</ci>
      <apply><image/><ci>f</ci></apply>
    </apply>
  </condition>
  <apply><in/>
    <ci>x</ci>
    <apply><codomain/><ci>f</ci></apply>
  </apply>
</apply>
\end{lstlisting}
The following remarks could be made.
\begin{enumerate}
\item This cries out for {\element{forallin}}.
\begin{lstlisting}
<OMBIND>
  <OMA>
    <OMS name="forallin" cd="quant2"/>
    <OMA>
      <OMS name="image" cd="fns1"/>
      <OMV name="f"/>
    </OMA>
  </OMA>
  <OMBVAR> <OMV name="x"/> </OMBVAR>
  <OMA>
    <OMS name="in" cd="set1"/>
    <OMV name="x"/> 
    <OMA>
      <OMS name="codomain" cd="fns1"/>
      <OMV name= "f"/>
    </OMA>
  </OMA>
</OMBIND>
\end{lstlisting}
\item It is a pretty good case for the ``forall with implies'' trick.
\begin{lstlisting}
<OMBIND>
  <OMS name="forall" cd="quant1"/>
  <OMBVAR> <OMV name="x"/> </OMBVAR>
  <OMA>
    <OMS name="implies" cd="logic1"/>
    <OMA>
      <OMS name="in" cd="set1"/>
      <OMV name="x"/> 
      <OMA>
        <OMS name="image" cd="fns1"/>
        <OMV name= "f"/>
      </OMA>
    </OMA>
    <OMA>
      <OMS name="in" cd="set1"/>
      <OMV name="x"/> 
      <OMA>
        <OMS name="codomain" cd="fns1"/>
        <OMV name= "f"/>
      </OMA>
    </OMA>
  </OMA>
</OMBIND>
\end{lstlisting}
\item Why use quantifiers at all?
\begin{lstlisting}
<OMA>
  <OMS name="subset" cd="set1"/>
  <OMA>
    <OMS name="image" cd="fns1"/>
    <OMV name= "f"/>
  </OMA>
  <OMA>
    <OMS name="codomain" cd="fns1"/>
    <OMV name= "f"/>
  </OMA>
</OMBIND>
\end{lstlisting}
Indeed there could be (possibly even ought to be, since in ZF it is the
{\it definition\/} of $\subset$) a FMP of {\element{subset}} that made this
equivalent to expression 2.
\end{enumerate}
\subsection{C.2.2.15 MMLdefinition: {\tt domainofapplication}}\label{C2215}
This contains the interesting statement
\begin{quotation}\noindent
Special cases of this qualifier can be abbreviated using one of interval
{\tt condition} or an ({\tt lowlimit},{\tt uplimit}) pair.
\end{quotation}
The examples given are
\begin{lstlisting}[language=MathML2]
<apply><int/>
  <domainofapplication><ci>C</ci></domainofapplication>
  <ci>f </ci>
</apply>
\end{lstlisting}
(which is a fairly straight-forward definite integral) and
\begin{lstlisting}[language=MathML2]
<apply><int/>
      <domainofapplication>
        <set>
          <bvar><ci>t</ci></bvar>
          <condition>
            <apply><in/>
              <ci>t</ci>
              <ci type="set">C</ci>
            </apply>
          </condition>
        </set>
      </domainofapplication>
      <ci>f</ci>
</apply>
\end{lstlisting}
which seems to this author to be a redundant variant. 
\subsection{C.2.3.1 MMLdefinition: {\tt quotient}}\label{C231}
This has the following property (presumably meant to be the equivalent of
OpenMath's FMP\footnote{OpenMath's FMP for {\tt quotient} in {\tt integer1} is
currently missing the proviso $b\ne0$.}).
\begin{lstlisting}[language=MathML2]
ForAll( [a,b], b != 0, a = b*quotient(a,b) + rem(a,b) )

<apply><forall/>
  <bvar><ci>a</ci></bvar>
  <bvar><ci>b</ci></bvar>
  <condition><apply><neq/><ci>b</ci><cn>0</cn></apply></condition>
  <apply><eq/>
    <ci>a</ci>
    <apply><plus/>
      <apply><times/>
          <ci>b</ci>
          <apply><quotient/><ci>a</ci><ci>b</ci></apply>
      </apply>
  <apply><rem/><ci>a</ci><ci>b</ci></apply>
    </apply>
  </apply>
</apply>
\end{lstlisting}
Again, this seems to be a pretty good case for the ``forall with implies''
trick, though {\element{forallin}} could be used.
\begin{lstlisting}
<OMBIND>
  <OMS name="forall" cd="quant1"/>
  <OMBVAR> <OMV name="a"/> <OMV name="b"/> </OMBVAR>
  <OMA>
    <OMS name="implies" cd="logic1"/>
    <OMA>
      <OMS name="neq" cd="relation1"/>
      <OMV name="b"/> 
      <OMS name="zero" cd="arith1"/>
    </OMA>
    <OMA>
      <OMS name="eq" cd="relation1"/>
      <OMV name="a"/> 
      <OMA>
        <OMS name="plus" cd="arith1"/>
        <OMA>
          <OMS name="times" cd="arith1"/>
          <OMV name="b"/> 
          <OMA>
            <OMS name="quotient" cd="integer1"/>
            <OMV name="a"/> 
            <OMV name="b"/> 
          </OMA>
        </OMA>
        <OMA>
          <OMS name="remainder" cd="integer1"/>
          <OMV name="a"/> 
          <OMV name="b"/> 
        </OMA>
      </OMA>
    </OMA>
  </OMA>
</OMBIND>
\end{lstlisting}
\subsection{C.2.3.2 MMLdefinition: {\tt factorial}}\label{C232}
This has the following property (presumably meant to be the equivalent of
OpenMath's FMP).
\begin{lstlisting}[language=MathML2]
ForAll( n, n \gt 0, n! = n*(n-1)! )

              
<apply><forall/>
  <bvar><ci>n</ci></bvar>
  <condition><apply><gt/><ci>n</ci><cn>0</cn></apply></condition>
  <apply><eq/>
    <apply><factorial/><ci>n</ci></apply>
    <apply><times/>
      <ci>n</ci>
      <apply><factorial/>
        <apply><minus/><ci>n</ci><cn>1</cn></apply>
      </apply>
    </apply>
  </apply>
</apply>
\end{lstlisting}
Again, this seems to be a pretty good case for the ``forall with implies''
trick, though {\element{forallin}} could be used.
\subsection{C.2.3.3 MMLdefinition: {\tt divide}}\label{C233}
This has the following property (presumably meant to be the equivalent of
OpenMath's FMP).
\begin{lstlisting}[language=MathML2]
ForAll( a, a!= 0, a/a = 1 ) 
          
<apply><forall/>
  <bvar><ci>a</ci></bvar>
  <condition><apply><neq/><ci>a</ci><cn>0</cn></apply></condition>
  <apply><eq/>
    <apply><divide/><ci>a</ci><ci>a</ci></apply>
    <cn>1</cn>
  </apply>
</apply>
\end{lstlisting}
Again, this seems to be a pretty good case for the ``forall with implies''
trick, though {\element{forallin}} could be used.
\subsection{C.2.3.4 MMLdefinition: {\tt max}}\label{C234}
This contains the interesting statement
\begin{quotation}\noindent
The elements may be listed explicitly or they may be described by a {\tt
domainofapplication}, for example, the maximum over all $x$ in the set $A$.
The {\tt domainofapplication} is often abbreviated by placing a {\tt
condition} directly on a bound variable. 
\end{quotation}
The example given is the following (our layout).
\begin{lstlisting}[language=MathML2]
<apply>
  <max/>
  <bvar><ci>y</ci></bvar>
  <condition>
    <apply>
      <in/>
      <ci>y</ci>
      <interval><cn>0</cn><cn>1</cn></interval>
    </apply>
  </condition>
  <apply><power/><ci> y</ci><cn>3</cn></apply>
</apply>
\end{lstlisting}
As OpenMath does not have a {\element{max}} operator acting on functions,
the nearest translation would seem to be the following.
\begin{lstlisting} 
<OMA>
  <OMS name="max" cd="minmax1"/>
  <OMA>
    <OMS name="map" cd="set1"/>
    <OMBIND>
      <OMS name="lambda" cd="fns1"/>
      <OMBVAR> <OMV name="y"/> </OMBVAR>
      <OMA>
        <OMS name="power" cd="arith1"/>
        <OMV name="y"/>
        <OMI> 3 </OMI>
      </OMA>
    </OMBIND>
    <OMA>
      <OMS name="interval_cc" cd="interval1"/>
      <OMI> 0 </OMI>
      <OMI> 1 </OMI>
    </OMA>
  </OMA>
</OMA>
\end{lstlisting}
We note that OpenMath seems to require us to be precise about the species of
interval we want to use.
\subsection{C.2.3.5 MMLdefinition: {\tt min}}\label{C235}
Nothing new need be said here.
\subsection{C.2.3.7 MMLdefinition: {\tt plus}}\label{C237}
The property here is the following.
\begin{lstlisting}[language=MathML2]
    Commutativity 
    <apply><forall/>
      <bvar><ci>a</ci></bvar>
      <bvar><ci>b</ci></bvar>
      <condition>
        <apply><and/>
          <apply><in/><ci>a</ci><reals/></apply>
          <apply><in/><ci>b</ci><reals/></apply>
        </apply>
      </condition>
      <apply><eq/>
        <apply><plus/><ci>a</ci><ci>b</ci></apply>
        <apply><plus/><ci>b</ci><ci>a</ci></apply>
      </apply>
    </apply>
\end{lstlisting}
Again, this seems to be a pretty good case for the ``forall with implies''
trick, though {\element{forallin}} could be used.
\subsection{C.2.3.8 MMLdefinition: {\tt power}}\label{C238}
The property here is the following.
\begin{lstlisting}[language=MathML2]
ForAll( a, a!=0, a^0=1 ) 
<apply><forall/>
  <bvar><ci>a</ci></bvar>
  <condition><apply><neq/><ci>a</ci><cn>0</cn></apply></condition>
  <apply><eq/>
    <apply><power/><ci>a</ci><cn>0</cn></apply>
    <cn>1</cn>
  </apply>
</apply>
\end{lstlisting}
Again, this seems to be a pretty good case for the ``forall with implies''
trick, though {\element{forallin}} could be used.
\subsection{C.2.3.9 MMLdefinition: {\tt rem}}\label{C239}
This has the same property, and solution, as {\tt quotient} (section
\ref{C231}).
\subsection{C.2.3.10 MMLdefinition: {\tt times}}\label{C2310}
The property here is the following.
\begin{lstlisting} 
ForAll( [a,b], condition(in({a,b}, Commutative)), a*b=b*a )
\end{lstlisting}
However, no formal translation is given, and it is not clear what one would
be.
\par
Later on we see the following property, to which the same remark applies as in
section \ref{C237}.
\begin{lstlisting}[language=MathML2]
<apply><forall/>
  <bvar><ci>a</ci></bvar>
  <bvar><ci>b</ci></bvar>
  <condition>
    <apply><and/>
      <apply><in/><ci>a</ci><reals/></apply>
      <apply><in/><ci>b</ci><reals/></apply>
    </apply>
  </condition>
  <apply><eq/>
    <apply><times/><ci>a</ci><ci>b</ci></apply>
    <apply><times/><ci>b</ci><ci>a</ci></apply>
  </apply>
</apply> 
\end{lstlisting}
\subsection{C.2.3.18 MMLdefinition: {\tt forall}}\label{C2318}
This contains the following example (our formatting).
\begin{lstlisting}[language=MathML2]
<apply>
  <forall/>
  <bvar><ci> x </ci></bvar>
  <condition>
    <apply><lt/><ci> x </ci><cn> 0 </cn></apply>
  </condition>
  <ci> x </ci>
</apply>
\end{lstlisting}
This seems to be $\forall x:x<0 x$. Since this last $x$ is not a Boolean, the
author respectfully submits that this example is ill-typed. In any case
{\element{forallin}} would solve the issue. 
\subsection{C.2.3.23 MMLdefinition: {\tt real}}\label{C2323}
This contains the following property,
\begin{lstlisting}[language=MathML2]
<apply><forall/>
  <bvar><ci>x</ci></bvar>
  <bvar><ci>y</ci></bvar>
  <condition>
    <apply><and/>
      <apply><in/><ci>x</ci><reals/></apply>
      <apply><in/><ci>y</ci><reals/></apply>
    </apply>
  </condition>
  <apply><eq/>
    <apply><real/>
      <apply><plus/>
        <ci> x </ci>
        <apply><times/><imaginaryi/><ci>y</ci></apply>
      </apply>
    </apply>
    <ci> x </ci>
  </apply>
</apply>
\end{lstlisting}
Again, this seems to be a pretty good case for the ``forall with implies''
trick, though {\element{forallin}} could be used.
\subsection{C.2.5.1 MMLdefinition: {\tt int}}\label{C251}
This contains the following example.
\begin{lstlisting}[language=MathML2]
<apply><int/>
  <bvar><ci> x </ci></bvar>
  <condition>
    <apply><in/><ci> x </ci><ci type="set"> D </ci></apply>
  </condition>
  <apply><ci type="function"> f </ci><ci> x </ci></apply>
</apply>
\end{lstlisting}
The discussion in section \ref{C2272} is appropriate here.
\subsection{C.2.5.6 MMLdefinition: {\tt bvar}}\label{C256}
This contains the following example (our formatting).
\begin{lstlisting}[language=MathML2]
<apply><forall/><bvar><ci>x</ci></bvar>
  <condition><apply><in/><ci>x</ci><reals/></apply></condition>
  <apply>
    <eq/>
    <apply><minus/><ci>x</ci><ci>x</ci></apply>
    <cn>0</cn>
  </apply>
</apply>
\end{lstlisting}
Again, this seems to be a pretty good case for the ``forall with implies''
trick, though c{\element{forallin}} could be used.
\subsection{C.2.5.8 MMLdefinition: {\tt divergence}}\label{C258}
This contains the following example.
\begin{lstlisting}[language=MathML2]
<apply>
  <eq/>
  <apply><divergence/><ci type="vectorfield">a</ci></apply>
  <apply>
    <limit/>
    <bvar><ci> V </ci></bvar>
    <condition>
      <apply>
        <tendsto/>
        <ci> V </ci>
        <cn> 0 </cn>
      </apply>
    </condition>
    <apply>
      <divide/>
      <apply>
        <int encoding="text" definitionURL="SurfaceIntegrals.htm"/>
        <bvar><ci> S</ci></bvar>
        <ci> a </ci>
      </apply>
      <ci> V </ci>
    </apply>
  </apply>
</apply>
\end{lstlisting}
Here the relevant part is the limit, which could be expressed (as it is in the
{\element{limit1}} CD) as the following.
\begin{lstlisting}
<OMA>
  <OMS cd="limit1" name="limit"/>
  <OMI> 0 </OMI>
  <OMS cd="limit1" name="both_sides"/>
  <OMBIND>
    <OMS cd="fns1" name="lambda"/>
      <OMBVAR>
      <OMV name="V"/>
      </OMBVAR>
      ...
  </OMBIND>
</OMA>
\end{lstlisting}
\subsection{C.2.6.1 MMLdefinition: {\tt set}}\label{C261}
This contains the following example.
\begin{lstlisting}[language=MathML2]
<set>
  <bvar><ci> x </ci></bvar>
  <condition>
    <apply><lt/>
      <ci> x </ci>
      <cn> 5 </cn>
    </apply>
  </condition>
  <ci>x</ci>
</set>
\end{lstlisting}
It is not clear what this means, but a plausible stab would seem to be the
following.
\begin{lstlisting}
<OMA>
  <OMS cd="set1" name="suchthat"/>
  <OMS cd="setname1" name="N"/>
  <OMBIND>
    <OMS cd="fns1" name="lambda"/>
    <OMBVAR>
      <OMV name="x"/>
    </OMBVAR>
    <OMA>
      <OMS cd="relation1" name="lt"/>
    <OMV name="x"/>
      <OMI> 5 </OMI>
    </OMA>
  </OMBIND>
</OMA>
\end{lstlisting}
\subsection{C.2.6.2 MMLdefinition: {\tt list}}\label{C262}
This contains the following example.
\begin{lstlisting}[language=MathML2]
<list order="numeric">
  <bvar><ci> x </ci></bvar>
  <condition>
    <apply><lt/>
      <ci> x </ci>
      <cn> 5 </cn>
    </apply>
  </condition>
</list>
\end{lstlisting}
There is no direct translation into OpenMath for reasons other than
{\element{condition}}, but {\element{suchthat}} in {\element{list1}} seems an obvious tool to
use.
\subsection{C.2.6.7 MMLdefinition: {\tt subset}}\label{C267}
This contains the following example.
\begin{lstlisting}[language=MathML2]
<apply>
  <subset/>
  <subset/>
  <bvar><ci type="set">S</ci></bvar>
  <condition>
    <apply><in/>
      <ci>S</ci>
      <ci type="list">T</ci>
    </apply>
  </condition>
  <ci>S</ci>
</apply>
\end{lstlisting}
Even assuming the second {\element{<subset/>}} to be a mistake, the present author
can make no sense of this.
\subsection{C.2.7.1 MMLdefinition: {\tt sum}}\label{C271}
This contains the following example (our formatting).
\begin{lstlisting}[language=MathML2]
<apply><sum/>
  <bvar><ci> x </ci></bvar>
  <condition>
    <apply> <in/><ci> x </ci><ci type="set">B</ci></apply>
  </condition>
  <apply><ci type="function"> f </ci><ci> x </ci></apply>
</apply>
\end{lstlisting}
This translates straightforwardly.
\begin{lstlisting}
<OMA>
  <OMS name="sum" cd="arith1"/>
  <OMV name="S"/>
  <OMV name="f"/>
</OMA>
\end{lstlisting}
We note that there is no need to name the dummy variable at all.
\subsection{C.2.7.2 MMLdefinition: {\tt product}}\label{C272}
The example, and its OpenMath translation, are essentially identical to the
previous section.
\subsection{C.2.7.3 MMLdefinition: {\tt limit}}\label{C273}
This contains the following example (our formatting).
\begin{lstlisting}[language=MathML2]
<apply><limit/>
  <bvar><ci>x</ci></bvar>
  <condition>
    <apply><tendsto/><ci>x</ci><cn>0</cn></apply>
  </condition>
  <apply><sin/><ci>x</ci></apply>
</apply>
\end{lstlisting}
The equivalent OpenMath would be the following.
\begin{lstlisting}
<OMA>
  <OMS cd="limit1" name="limit"/>
  <OMI> 0 </OMI>
  <OMS cd="limit1" name="both_sides"/>
  <OMS cd="transc1" name="sin"/>
  </OMBIND>
</OMA>
\end{lstlisting}
We again note that there is no need to name the dummy variable at all.
\subsection{C.2.10.2 MMLdefinition: {\tt matrix}}\label{C2102}
This contains the following example (our formatting).
\begin{lstlisting}[language=MathML2]
<matrix>
  <bvar><ci type="integer">i</ci></bvar>
  <bvar><ci type="integer">j</ci></bvar>
  <condition>
    <apply><and/>
      <apply><in/>
        <ci>i</ci>
        <interval><ci>1</ci><ci>5</ci></interval>
      </apply>
      <apply><in/>
        <ci>j</ci>
        <interval><ci>5</ci><ci>9</ci></interval>
      </apply>
    </apply>
  </condition>
  <apply><power/>
    <ci>i</ci>
    <ci>j</ci>
  </apply>
</vector>
\end{lstlisting}
We can assume that this should end {\element{</matrix>}} instead of
{\element{</vector>}}, but this has no equivalent in OpenMath.
\subsection{C.2.11.3 MMLdefinition: {\tt rational}}\label{C2113}
This contains the following property (our formatting).
\begin{lstlisting}[language=MathML2]
    for all z where z is a rational, there exists 
      integers p and q with p/q = z
<apply><forall/>
  <bvar><ci>z</ci></bvar>
  <condition>
    <apply><in/><ci>z</ci><rationals/></apply>
    </condition>
  <apply><exists/>
    <bvar><ci>p</ci></bvar>
    <bvar><ci>q</ci></bvar>
    <apply><and/>
      <apply><in/><ci>p</ci><integers/></apply>
      <apply><in/><ci>q</ci><integers/></apply>
      <apply><eq/>
        <apply><divide/><ci>p</ci><ci>q</ci></apply>
        <ci>z</ci>
      </apply>
    </apply>
  </apply> 
\end{lstlisting}
{\element{forallin}} seems the obvious solution, though the implies trick
could also be used.
\subsection{C.2.11.6 MMLdefinition: {\tt primes}}\label{C2116}
This contains the following property (our formatting).
\begin{lstlisting}[language=MathML2]
ForAll( [d,p], p is prime, Implies( d | p , d=1 or d=p ) ) 
<apply><forall/>
  <bvar><ci>d</ci></bvar>
  <bvar><ci>p</ci></bvar>
  <condition>
    <apply><and/>
    <apply><in/><ci>p</ci><primes/></apply>
    <apply><in/><ci>d</ci><naturalnumbers/></apply>
    </apply>
  </condition>
  <apply><implies/>
    <apply><factorof/><ci>d</ci><ci>p</ci></apply>
    <apply><or/>
      <apply><eq/><ci>d</ci><cn>1</cn></apply>
      <apply><eq/><ci>d</ci><ci>p</ci></apply>
    </apply>
  </apply>
</apply>
\end{lstlisting}
This could be encoded with the forall/implies trick, except that the result
would have two implication signs --- perfectly correct, but possibly harder to
read. We would need to nest {\element{forallin}}, as in the following.
\begin{lstlisting}
<OMBIND>
  <OMA>
    <OMS name="forallin" cd="quant2"/>
    <OMS name="N" cd="setname1"/>
  </OMA>
  <OMBVAR> <OMV name="d"/> </OMBVAR>
  <OMBIND>
    <OMA>
      <OMS name="forallin" cd="quant2"/>
      <OMS name="P" cd="setname1"/>
    </OMA>
    <OMBVAR> <OMV name="p"/> </OMBVAR>
    ...
  </OMBIND>
</OMBIND> 
\end{lstlisting}
\subsection{C.2.11.15 MMLdefinition: {\tt infinity}}\label{C21115}
This contains the following property and example (our formatting).
\subsubsection{C.2.11.15(1) MMLdefinition: {\tt infinity}}\label{C211151}
\begin{lstlisting}[language=MathML2] 
    for all reals x, x \lt infinity
<apply><forall/>
  <bvar><ci>n</ci></bvar>
  <condition><apply><in/><ci>n</ci><reals/></apply></condition>
  <apply><lt/><ci>n</ci><infinity/></apply>
</apply>
\end{lstlisting}
{\element{forallin}} seems the obvious solution, though the implies trick
could also be used.
\subsubsection{C.2.11.15(2) MMLdefinition: {\tt infinity}}\label{C211152}
\begin{lstlisting}[language=MathML2]
<apply><eq/>
  <apply><limit/>
    <bvar><ci>x</ci></bvar>
    <condition>
      <apply><tendsto/><ci>x</ci><infinity/></apply>
      </condition>
    <apply><divide/><cn>1</cn><ci>x</ci></apply>
  </apply>
  <cn>0</cn>
</apply>
\end{lstlisting}
From OpenMath's point of view, this is a straightforward limit.
\begin{lstlisting} 
<OMA>
  <OMA>
    <OMS cd="limit1" name="limit"/>
    <OMS name="infinity" cd="nums1"/>
    <OMS cd="limit1" name="below"/>
    <OMBIND>
      <OMS cd="fns1" name="lambda"/>
      <OMBVAR> <OMV name="x"/> </OMBVAR>
      <OMA>
        <OMS name="divide" cd="arith1"/>
        <OMI> 1 </OMI>
        <OMV name="x"/>
      </OMA>
    </OMBIND>
  </OMA>
  <OMS cd="alg1" name="zero"/>
</OMA>
\end{lstlisting}
\fi
\bibliography{jhd}
\iffull
\appendix
\section{OpenMath Standard changes}
%As an extension of $\alpha$-conversion (backwards) to the binding operator in OpenMath
%seems more exotic, we propose to extend OpenMath binding objects to allow more than one ``argument''.
Concretely we propose to liberalize the definition of a ``OpenMath Binding
Object'' in~\cite[section 2.1.3]{OpenMath2004a} from the current

\begin{quote}
  \begin{itemize}
  \item[($iv$)] If $B$ and $C$ are OpenMath objects, and $v_1$, \ldots, $v_n$ ($n>0$)
    are OpenMath variables or attributed variables, then
    \begin{center}
      $\mathbf{binding}(B, v_1,\ldots,C)$
    \end{center}
    is an {\emph{OpenMath binding object}}.
  \end{itemize}
\end{quote}
to the following form:
\begin{quote}
  \begin{itemize}
  \item[($iv$)] If $B$, $C_1$, \ldots, and $C_m$ are OpenMath objects, and $v_1$, \ldots,
    $v_n$ ($n>0$) are OpenMath variables or attributed variables, then
    \begin{center}
      $\mathbf{binding}(B, v_1,\ldots,C_1,\ldots,C_m)$
    \end{center}
    is an {\emph{OpenMath binding object}}.
  \end{itemize}
\end{quote}
And correspondingly change the following sentences in~\cite[section 2.2]{OpenMath2004a}
\begin{quote}
  Phrasebooks are allowed to use $\alpha$ conversion in order to avoid clashes of variable
  names. Suppose an object $\Omega$ contains an occurrence of the object
  $\mathbf{binding}(B,v,C)$.  This object $\mathbf{binding}(B,v,C)$ can be replaced in
  $\Omega$ by $\mathbf{binding}(B,z,C')$ where $z$ is a variable not occurring free in $C$
  and $C'$ is obtained from $C$ by replacing each free (i.e., not bound by any
  intermediate binding construct) occurrence of $v$ by $z$.  This operation preserves the
  semantics of the object $\Omega$.
\end{quote}
to 
\begin{quote}
  Phrasebooks are allowed to use $\alpha$ conversion in order to avoid clashes of variable
  names. Suppose an object $\Omega$ contains an occurrence of the object
  $\Theta\colon=\mathbf{binding}(B,v_1,\dots,v_n,C_1,\ldots,C_m)$.  Then $\Theta$ can be
  replaced in $\Omega$ by $\mathbf{binding}(B,v'_1,\ldots,v'_{n} C'_1,\ldots,C'_m)$ where
  \begin{itemize}
  \item $z$ is a variable or attributed variable not occurring in $\Theta$, and
  \item $v'_k=z$ for some $k>0$,
  \item $v'_i$ is obtained from $v_i$ by replacing each free (i.e., not bound by any
    intermediate binding construct) occurrence of $v_k$ by $z$ for all $i$ with $k\leq
    i\leq n$.
  \item $C'_j$ is obtained from $C_j$ by replacing each free occurrence of $v_k$ by $z$
    for all $1\leq j\leq m$.
  \end{itemize}
  This operation preserves the semantics of the object $\Omega$.
\end{quote}
Note that while we were at it, we also cleaned up the imprecision in OpenMath2 that only
allowed $\alpha$-renaming in the presence of a single bound variable and clarified that
variables in attributions must also be renamed. 

Compared to these necessary clarifications, our proposed change is minor, but as we have
seen, it allows us to represent (\ref{eq:bounds2}) and (\ref{eq:bounds1}) naturally. As we
will see below, it also simplifies the remaining alignment issues considerably. 

Note that this extension of the OpenMath Objects is backwards compatible to OpenMath2
objects, since the case of unary binding operators is a special case. Furthermore, all
OpenMath2 content dictionaries only describe unary binding operators which will stay unary
as guaranteed by the OpenMath rules for CD management. Finally note that our extension
proposal is different from the earlier proposal by one of the authors to introduce a
``condition-like element'' into OpenMath3. That proposal was rejected on the grounds that
it is not clear that the meaning of the MathML {\element{condition}} element is
sufficiently independent of the particular binding operator, as will become clear in the section on conditions.

%\begin{thebibliography}{9}
%\bibitem{JHD-JEM}
%Davenport,J.H.,
%OpenMath in a (Semantic) Web. 
%Presentation at third Joining Educational Mathematics Workshop
%\url{http://www.jem-thematic.net/files_private/Barcelona.pdf},
%February 1, 2008.
%\bibitem{JHD}
%Davenport,J.H.,
%OpenMath and MathML:
%Differentiating between analysis and algebra.
%\url{http://staff.bath.ac.uk/masjhd/differentiate.html},
%October 4, 2008.
%\bibitem{MK}
%Kohlhase,M.,
%OpenMath3 without conditions: A Proposal for a
%MathML3/OM3 Calculus Content Dictionary.
%\url{http://svn.openmath.org/OpenMath3/doc/blue/noconds/note.tex},
%September 6, 2008.
%\end{thebibliography}
\fi
\end{document}
$$
2\pi\phi=\left\{\int_\delta^{2\pi-\delta}+\int_{-\delta}^0+\int_0^\delta\right\}
\frac{a^2-r^2}{a^2-2ar\cos \vartheta+r^2}f(\theta+\vartheta)d\vartheta
$$
\cite[(8) p.~435]{JeffreysJeffreys1956}
\par
\begin{quotation}
If $Q=[0,1]\times[0,\pi/2]$, evaluate
${\int\int}_midsub Q (x\sin y -ye^x)dxdy$. \cite[p. 360, Example
1]{Apostol1967}
\pause
\begin{itemize}
\end{itemize}
\begin{lstlisting}
<OMA>
</OMA>
\end{lstlisting}
\begin{enumerate}
\end{enumerate}

% LocalWords:  InvisibleTimes MathML's csymbol definitionURL OMA OMBIND bvar cd
% LocalWords:  connotin notin JHD existsin forallin suchthat eq
% LocalWords:  fns OMBVAR OMV arith OMI transc dx dy partialdiff af Df lowlimit
% LocalWords:  uplimit domainofapplication defint lr dt dxdydz MK's nassoc fn
% LocalWords:  lcccc llll jhd vvv FlashMeeting withcond vvvwithcond vvvcond FMP
% LocalWords:  forallwithcond forallcond existscond DPC cond genericwithcond oo
% LocalWords:  genericcond quant OMOBJ xmlns cdbase setname setdiff gcd lcm leq
% LocalWords:  sdev cartesianproduct geq minmax OMR naturalnumbers infimum tele
% LocalWords:  Arelation tendsto MMLdefinition mtext ForAll defintset calciulus
% LocalWords:  defintcond ZF neq imaginaryi vectorfield SurfaceIntegrals htm ar
% LocalWords:  factorof nums alg midsub dxdy pre ci lt cn Apostol tripleintcond
% LocalWords:  mathescape forall lcccccc lor bigvee vee subscripted idempotence
% LocalWords:  reln Carlisle Caprotti Lange Libbrecht Rowley tripleint Ctable
